\phantomsection\mtcaddchapter[Anexo A: abreviaturas]
\FormatoAnexo

%Agrego las marcas laterales
\AddLabelsAxUno

\chapter*{Anexo A: abreviaturas}

	\noindent Lista de abreviaturas, acrónimos y siglas recurrentes utilizadas en este trabajo, listadas por orden alfabético-

    \begin{acronym}[\hspace*{3cm}] 
    \setlength{\itemsep}{0.2\parsep} 
    \small
	\acro{ESC}{Electrodo Saturado de Calomel}
    \acro{FIB}{Focused Ion Beam}
    \acro{SEM}{Scanning Electron Microscopy} %O pongo en castellano MEB
    \acro{EPA}{Elipsoporosimetría Ambiental}
    \acro{HQ}{hidroquinona}
    \acro{Fc}{ferroceno metanol}
    \acro{FcCN}{ferro/ferri cianuro de potasio}
    \acro{AR}{hexaaminorutenio}
    \acro{PDM}{Película Delgada Mesoporosa}
    \acro{PDM@Si}{Película Delgada Mesoporosas de óxido de Silicio}
    \acro{PDM@Si-CTAB}{Película Delgada Mesoporosas de óxido de Silicio estructura con CTAB}
    \acro{PDM@Si-F127}{Película Delgada Mesoporosas de óxido de Silicio estructura con F127}
    \acro{INTI}{Instituto Nacional de Tecnología Industrial}
    \acro{INTI-CMNB}{INTI-Centro de Micro y Nano Electrónica del Bicentenario}
    \acro{INTI-CIEPS}{INTI-Centro de Investigaciones en Procesos Superficiales}
    \acro{CAC-CNEA}{Centro Atómico Constituyentes - Comisión de Energía Atómica}
    \acro{FTIR}{Fourier Transform Infrared Spectroscopy}
    \acro{EQ}{Electroquimica}
    \acro{XPS}{X-ray Photoelectron Spectroscopy}
    \acro{CAD}{Computer-Aided Design}
    \acro{ET}{Electrodo de Trabajo}
    \acro{CE}{Contraeletrodo}
    \acro{ER}{Electrodo de Referencia}
    \acro{INIFTA}{Instituto de Investigaciones Fisico Químicas Teóricas y Aplicadas}
    \acro{MPTMS}{3-mercaptopropil trimetoxisilano}
    \acro{VC}{Voltametría Cíclica}
    \end{acronym}
   
    PMMA\\
    PAI\\
    cms=concentracion miscelar critica\\
    polibutileno de tereftalato (PBT)\\
    polietileno de tereftalato (PET),\\
    Digitalizacion de graficos: http://arohatgi.info/WebPlotDigitizer/\\
    Vidrio ITO\\
    Vidrio FTO\\

\thispagestyle{backmatter}