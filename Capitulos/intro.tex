En su conferencia " " en 1959 el físico Richard Feymann, considerado por muchos el padre de la nanotecnología, sugiere que se podría escribir toda la enciclopedia británica en al cabeza de un alfiler, incluso antes del uso generalizado de las técnicas de microscopia electrónica. [cita: “There’s Plenty of Room at the Bottom”
(Richard Feynman, Pasadena, 29 December 1959)] El termino nano-tecnologia fueé incorporado por primera por Taniguchi para describir procesos de microfabricación como deposito de películas delgadas o <<ion millling>>, y lo define como aquellos procesos de separación, consolidación y deformación de los materiales átomo por átomo o molécula por molécula. [cita: N. Taniguchi, "On the Basic Concept of 'Nano-Technology'," Proc. Intl. Conf. Prod. Eng. Tokyo, Part II, Japan Society of Precision Engineering, 1974.] Fue Drexler quien finalmente popularizo el termino nonotecnologia en su libro Engines of Creation: The Coming Era of Nanotechnology. El fuerte impacto que tiene hoy en día la nonotecnologia no es sólo por el hecho de miniaturizar, sino por las propiedades y características nuevas y/o diferenciales que surgen de dicha miniturización, ya sean cambio en las propiedades ópticas, eléctricas, magnéticas o mecánicas. Es por este aporte novedoso y por la alta posibilidad de miniturización que esta disciplina no deja de crecer y sus contribuciones a la vida cotidiana se encuentran en medicina, electrónica, telecomunicaciones, sensores, materiales aeroespaciales, envases, pinturas, cosmética, atomoviles, aceros y la lista no parece tener fin.

La nanotecnologia se caracteriza fundamentalmente por ser mutidisciplinaria. Ciencia de materiales, biotecnología y electronica interactuando para desarrollar nanobiosenseres, por ejemplo. Esta tesis, que tiene por objetivo principal fabricar sensores, combina tres componentes. Materiales autoensamblados, microfabricación y electroquimica. 
El diagrama de conjuntos de la figura 1, muestra las caracteristicas que 

Aca va el diagrama para explicar como aporta cada area.

 
pongo estas secciones en la intro:
Se expone breve,ente a cotinuacion los fundamentos teoricos de cada una de las areas etmaticas tratados a lo largo de la tesis que nos llevaran finalmente, en su conjunto a sentar las bases para fabricar microsensores electroqquimicos a base de materiales mesoporoso.

1) mesoporoso

2) eq

3) microfa

Citas:

@article{576,
    author = {Taniguchi, N.},
    booktitle = {Bulletin of the Japan Society of Precision Engineering},
    citeulike-article-id = {9148416},
    keywords = {nanotechnologie, wissenschaftsgeschichte},
    pages = {18-23+},
    posted-at = {2011-04-12 16:01:31},
    priority = {0},
    title = {{On the Basic Concept of 'Nano-Technology'}},
    year = {1974}
}

@book{P_S-R:408,
    author = {Drexler, K. Eric},
    citeulike-article-id = {9148371},
    citeulike-linkout-0 = {http://www.foresight.org/EOC/index.html},
    isbn = {0-385-19973-2},
    keywords = {nanotechnologie, zukunftsforschung},
    location = {New York},
    posted-at = {2011-04-12 16:01:29},
    priority = {0},
    publisher = {Anchor Press},
    title = {{Engines of Creation. The Coming Era of Nanotechnology}},
    url = {http://www.foresight.org/EOC/index.html},
    year = {1987}
}

This is the transcript of a talk presented by Richard P. Feynman to the American Physical Society
in Pasadena on December 1959
