%Linea Para poder completar automaticamente las citas con el Sublime
%No hace el documento, se puede borrar esta linea si no se usa el Sublime
%------------------------------------------------------------------------------
 \newcommand{\NoBiblioConc}[1]{
 \ifthenelse{\equal{#1}{verdadero}}{}{\bibliography{Referencias/base_bibliografica}}
 \NoBiblioConc{verdadero}}
 %-----------------------------------------------------------------------------

%Formato (Nombre de capitulo largo o corto), nombre del capitulo y estilo de la
%Portada del Capitulo
%------------------------------------------------------------------------------
 	
 %Formato en si, titulo en un solo renglon
 \FormatoCapituloUnaLinea
 
 %Nombre y etiquete para referir
 \chapter{Conclusiones}
 \label{chap:Conclusiones}

 %Para que no salga el numero de pagina en la portada del capitulo
 \thispagestyle{empty}
	
 %Resumen del Capitulo en Italica


 %Indice de capitulo alineada al borde inferior de la pagina, nueva pagina
 \vfill
 \minitoc
 \newpage
 %-------------------------------------------------------------------------------

A lo largo de este trabajo de tesis todo la concentración y el esfuerzo estuvieron dedicados a la fabricación escalable de sensores electroquímicos permeoselectivos. 

Las primeras etapas tuvieron como eje la compatibilidad entre los procesos de fabricación \textit{bottom-up} y \textit{top-down}, los primeros para sinterizar las películas delgadas mesoporosas y los segundos para fabricar los electrodos de los sensores. Las problemáticas que allí surgieron estuvieron vinculadas a la pobre adherencia de los electrodos de oro a las películas y a las temperaturas de calcinación usadas tradicionalmente para condensar y extraer el surfactante en películas delgadas mesoporosas. Esa temperatura (>\SI{350}{\celsius}) favorece los procesos difusivos de impurezas hacia la superficie de los electrodos, afectando la detección Electroquímica de moléculas. Las soluciones a los problemas de adherencia se basaron en modificar químicamente los electrodos utilizando moléculas de anclaje químicamente compatible con las películas de SiO$_2$. Para minimizar los procesos difusivos se siguieron dos estrategias: aumentar la pureza del Au pulverizado para formar los electrodos y bajar la temperatura de condensación de las películas mesoporosas de SiO$_2$ hasta \SI{130}{\celsius}. La primera, sin duda fue la mas sencilla y la que primero se llevo a cabo obteniéndose buenos resultados. Sin embargo la segundo era la que presentaba mayores desafíos, estabilizar la estructura del cristal líquido, condensar el óxido seguido de un proceso de extracción del surfactante sin aumentar la temperatura por encima de \SI{130}{\celsius}. Basándose en la literatura especializadas se llevaron a cabo métodos novedosos para la condensación y extracción de las películas. Se diseñaron y experimentaron cinco procesos posdepósito para condensar películas mesoporosas: simplificado, prolongado, ácido, alcalino y alto vacío. Todo el capítulo 3 está dedicado a la comparación de las películas mesoporosas obtenidas por estos procesos con las calcinadas. las caracterizaciones incluyeron técnicas como elipsoporosimetría ambiental, microscopía óptica, de barrido y de iones focalizados de galio, espectroscopía IR, ángulo de contacto y voltametría cíclica. Se evaluaron características como espesor, grado de condensación, grado de extracción del surfactante, accesibilidad, porosidad, distribución de tamaño de poros y cuellos. Todos los métodos llevaron a películas porosas con carácteristicas distintivas que se detallan en dicho capítulo. A pesar de esto se consideró utilizar el método de lato vacío para continuar con la elaboración de los sensores. Esta elección esta basada en que fué el método que mostró películas cuasi equivalentes a las calcinadas y por ser el proceso que no utilizó reactivos extra en la sintesis, por lo que se considero el que podría estar libre de interferencias a la hora de medir. El desarrollo de estos métodos no solo permitió minimizar los procesos difusivos sino que también abre el camino para depositar óxidos puros o mixtos de Si$_x$Zr$_{1-x}$O$_2$ sobre sustratos termicamente lábiles, como acrílicos, PET o polimeros en general.

Durante todo el trabajo de tesis las mediciones EQ fueron una tarea que se llevó en forma trasversal y constante. Los resultados de las mismas está distribuidas a lo largo de todos lo capítulos de la tesis.  Estas tuvieorn mucha importancia y muchos propositos, caracetrizar las PDM, detedctar las sondas en fucniona de la permeoselectividad y por ultimo estudiar los fenomenos de rtasnsporte.
muchos datos, K, hopping, difusion, selectividad, concentracion, etc. Simulacion, redox etc. Finalmente disolucion de la silice y zirconio
todos los datos que se sacarosn. LLevaron a una conclusion sumamente importante solubilidad estabilidad y peliculas de Zr

Despues otras de las cosas que se hizo a lo largo de la tesos es la fabricaciones de llos electrodos.


Metodos de sintesis alternativos, estudio sistematico con diferentes surfactantes y tecnicas para estudiar la accesibilidad el grado de condensacion y la extraccion de surfactante.

Estudio de las propiedades de trasnporte, demostracion de la adsorcion de la concnetracion dentro de los poros coeficientes de disfusion y de la estabilida de las pelicuas.



FUTURO

recien con la introduccion de Zr se piuede abrir la puerta a experimentos mas reproducibles, medtodo de aprendizaje, funcionalizaciones diferenciales. distintas proporciones con Zr. distintas funcionalizaciones.
redes neuronales

\newpage

