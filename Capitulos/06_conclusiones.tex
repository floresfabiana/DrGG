%Linea Para poder completar automaticamente las citas con el Sublime
%No hace el documento, se puede borrar esta linea si no se usa el Sublime
%------------------------------------------------------------------------------
 \newcommand{\NoBiblioConc}[1]{
 \ifthenelse{\equal{#1}{verdadero}}{}{\bibliography{Referencias/base_bibliografica}}
 \NoBiblioConc{verdadero}}
 %-----------------------------------------------------------------------------

%Formato (Nombre de capitulo largo o corto), nombre del capitulo y estilo de la
%Portada del Capitulo
%------------------------------------------------------------------------------
 	
 %Formato en si, titulo en un solo renglon
 \FormatoCapituloUnaLinea
 
 %Nombre y etiquete para referir
 \chapter{Conclusiones}
 \label{chap:Conclusiones}

 %Para que no salga el numero de pagina en la portada del capitulo
 \thispagestyle{empty}
	
 %Resumen del Capitulo en Italica


 %Indice de capitulo alineada al borde inferior de la pagina, nueva pagina
 \vfill
 \minitoc
 \newpage
 %-------------------------------------------------------------------------------

GALA

A lo largo de este trabajo de tesis todo la concentración y el esfuerzo estuvieron dedicados a la fabricación escalable de sensores electroquímicos permeoselectivos. Las primeras etapas tuvieron como eje la compatibilidad entre los procesos de fabricación \textit{bottom-up} y \textit{top-down}, los primeros para sinterizar las peliculas delgadas mesporosas y los segundos para fabricar los electrodos de los sensores. Las problemáticas en allí surgieron estuvieron vinculadas a la pobre adherencia de las peliclas sobre los electrodos y a las temperaturas de calcinacion de las pelicuasl, que favores los procesos difusivos de impurezaas hacia la suoperfie de los electrodos. Las soluciones vienen de la mano de la quimica para el casoi de la adhrencia. Para el caso de la remeprar fue el desarrollo de todo el capitulo 3 donde se expusiron los resultados de desarrollar metrodoos de sintesos de peiluicas delgadas mesoporosas disminuyendo la temperatura de sisnteas a 130.  ... y la resolucion llevo a metodos novedosos . Las caracterizasciones incluyeron microscopias de todo tipo, espectrocopia, etp etc.

Una etapa trasnversal que se hizo durante todo el trabajo fue las mediciones EQ. Estas tuvieorn mucha importancia y muchos propositos, caracetrizar las PDM, detedctar las sondas en fucniona de la permeoselectividad y por ultimo estudiar los fenomenos de rtasnsporte.
todos los datos que se sacarosn. LLevaron a una conclusion sumamente importante solubilidad estabilidad y peliculas de Zr

Despues otras de las cosas que se hizo a lo largo de la tesos es la fabricaciones de llos electrodos.


Metodos de sintesis alternativos, estudio sistematico con diferentes surfactantes y tecnicas para estudiar la accesibilidad el grado de condensacion y la extraccion de surfactante.

Estudio de las propiedades de trasnporte, demostracion de la adsorcion de la concnetracion dentro de los poros coeficientes de disfusion y de la estabilida de las pelicuas.



FUTURO

recien con la introduccion de Zr se piuede abrir la puerta a experimentos mas reproducibles, medtodo de aprendizaje, funcionalizaciones diferenciales. distintas proporciones con Zr. distintas funcionalizaciones.
redes neuronales

\newpage

