%Linea Para poder completar automaticamente las citas con el Sublime
%No hace el documento, se puede borrar esta linea si no se usa el Sublime
%------------------------------------------------------------------------------
 \newcommand{\NoBiblioConc}[1]{
 \ifthenelse{\equal{#1}{verdadero}}{}{\bibliography{Referencias/base_bibliografica}}
 \NoBiblioConc{verdadero}}
 %-----------------------------------------------------------------------------

%Formato (Nombre de capitulo largo o corto), nombre del capitulo y estilo de la
%Portada del Capitulo
%------------------------------------------------------------------------------
 	
 %Formato en si, titulo en un solo renglon
 \FormatoCapituloUnaLinea
 
 %Nombre y etiquete para referir
 \chapter{Conclusiones}
 \label{chap:Conclusiones}

 %Para que no salga el numero de pagina en la portada del capitulo
 \thispagestyle{empty}
	
 %Resumen del Capitulo en Italica


 %Indice de capitulo alineada al borde inferior de la pagina, nueva pagina
 \vfill
 \minitoc
 \newpage
 %-------------------------------------------------------------------------------

\section*{Conclusiones}

A lo largo de la tesis toda la concentración y el esfuerzo estuvieron dedicados a la fabricación escalable de sensores electroquímicos permeoselectivos. 

Las primeras etapas tuvieron como eje la compatibilidad entre los procesos de fabricación \textit{bottom-up} y \textit{top-down}, los primeros para sintetizar las películas delgadas mesoporosas de diferentes tamaños de poros utilizando diferentes surfactante (Pluronic F127, Brij58 y CTAB) y los segundos para fabricar los electrodos de Au para los sensores. Las problemáticas que allí surgieron estuvieron vinculadas a la pobre adherencia de los electrodos de oro a las películas y a las temperaturas de calcinación tradicionales para condensar y extraer el surfactante en películas delgadas mesoporosas. La solución a los problemas de adherencia se basó en modificar químicamente los electrodos utilizando moléculas de anclaje químicamente compatible con las películas de SiO$_2$.  La temperatura de calcinación, típicamente $\geq$\SI{350}{\celsius} favorece los procesos difusivos de impurezas hacia la superficie de los electrodos, afectando el desempeño electroquímico de los sensores. Para minimizar dichos procesos difusivos se plantearon dos estrategias: aumentar la pureza del Au pulverizado para componer los electrodos y bajar la temperatura de condensación de las películas mesoporosas de SiO$_2$ hasta \SI{130}{\celsius}. La primera, más simple de implementar, fue la que primero se llevo a cabo obteniéndose resultados satisfactorios. Sin embargo la segundo era la que presentaba mayores desafíos tanto científicos como tecnológicos: estabilizar la estructura del cristal líquido, condensar el óxido seguido de un proceso de extracción del surfactante manteniendo la temperatura pode debajo de \SI{130}{\celsius}. 

Tomando algunos aspectos de la literatura especializadas se llevaron a cabo métodos novedosos para la condensación y extracción de las películas. Para este fin se diseñaron y experimentaron cinco procesos posdepósito: simplificado, prolongado, ácido, alcalino y alto vacío. Todo el capítulo 3 está dedicado a la comparación de las películas mesoporosas obtenidas por estos procesos con las calcinadas. Las caracterizaciones incluyeron técnicas como elipsoporosimetría ambiental, microscopía óptica, de barrido electrónico y de iones focalizados de galio, espectroscopía IR, ángulo de contacto y voltametría cíclica. Se evaluaron características como indice de refracción, espesor, grado de condensación, grado de extracción del surfactante, accesibilidad, porosidad y distribución de tamaño de poros y cuellos. Todos los métodos llevaron a películas porosas con características distintivas que se detallan en dicho capítulo. A pesar de que se podría haber usado cualquiera de los procesos desarrollados, se consideró el método de alto vacío el más adecuado para avanzar hacía la elaboración de sensores amperométricos selectivos. Esta elección esta fundamentada en que el método de alto vacío fue el proceso que mostró películas con propiedades equivalentes a las calcinadas y, además, no utiliza reactivos extra en la síntesis, lo que redunda en una síntesis limpia, libre de interferencias, productos de reacciones secundarias o moléculas sensibles de ser adsorbidas. El desarrollo de estos métodos no solo permitió minimizar los procesos difusivos, obtener respuestas electroquímicas de calidad sino que también abre el camino para depositar películas mesoporosas de óxidos puros o mixtos sobre sustratos termicamente lábiles, como acrílicos, PET o polímeros en general.

Durante todo el trabajo de tesis las mediciones electroquímicas fueron una labor que se llevó en forma trasversal y constante. Los resultados de las mismas está distribuidas a lo largo de todos lo capítulos de la tesis. Estas tuvieron gran relevancia y muchos propósitos: 1) evaluar el desempeño electroquímico de las películas delgadas de (Cr,Ti)\textbar Au destinadas a usar como electrodos, 2) comprobar la accesibilidad de moléculas dentro de las películas delgadas mesoporosas, 3) estudiar los mecanismos de transporte y obtener parámetros fisicoquímicos de los sistemas porosos, y 4) medir analiticamente la respuesta de las sondas en un sensor formado por electrodos de diferentes caracteristicas.

El análisis minucioso y reiterativo sobre los voltagramas, tanto de corriente continua como de corriente alterna, llevaron a conclusiones generales sobre el transporte de moléculas dentro de los poros. Se utilizaron sondas electroactivas de distinta carga: \ferroferri\space de carga negativa, \aminorutenio\space de carga positiva y \ferroceno\space de carga neutra. Los resultados obtenidos permitieron evaluar las propiedades permeoselectivas de las membranas y, a su vez, establecer la capacidad de preconcentración, estimar la concentración de sonda adsorbida, proponer un mecanismo para el transporte de carga dentro de las películas y calcular coeficiente de difusión tanto para la permeación ($D$) como para la trasferencia de carga mediante \textit{electron hopping} $D_e$.
 Se realizaron simulaciones y experimentos con el ánimo de establecer bajo qué condiciones de contorno se pueden manifestar fenómenos de mediación rédox entre la película y un analito en solución. No fue posible reproducir experimentalmente dichas condiciones, sin embargo se realizaron simulaciones con posibles escenarios y a partir de estos experimentos, realizó un análisis profundo de cómo influyen la constante de difusión $D_e$ y la constante de equilibrio $K$ en demerito de dicho proceso de mediación. 

Un resultado destacable fue la demostración de la disolución de las películas de SiO$_2$ catalizada por el ciclado electroquímico. Se depositaron soles agregando un 10\% de  ZrCl$_4$ en su formulación, seguidas por el tratamiento en alto vacío lo que llevó a películas mesoporosasa homogéneas de composición general Si$_{0.9}$Zr$_{0.1}$O$_2$. La adición de Zr permitió aumentar la estabilidad química y mecánica de las películas manteniendo las propiedades permeoselectivas y sin perder capacidad de adsorción de \aminorutenio. Con la intensión de regular las propiedades de transporte se funcionalizaron estas películas, de gran estabilidad, con dihexadecilfosfato (DHDP) y 3-aminopropil trietoxisilano (APTES). Se discutieron e interpretaron como se afecta el transporte debido a estas funcionaliciones llegando a la conclusión de que se afectan significativamente la velocidad y la capacidad de adsorción para \ru.

Se fabricaciones electrodos recurrentemente a lo largo de todo el periodo que llevó el trabajo. Se hizo primero un diseño que luego fue reemplazado por uno más compacto y especialmente optimizado para usar en procesos electroquímicos. En esta últmima etapa se volcó todo la experiencia adquirida para fabricar electrodos de Ti\textbar Au recubiertos con películas de mixtas Si$_{0.9}$Zr$_{0.1}$O$_2$ con poros de \SI{\approx 9}{\nm} de diámetro, sintetizadas por el método de alto vacío para mantener la calidad analítica de los electrodos. Finalmente se funcionalizaron sobre algunos de ellos con DHDP y APTES a fin de obtener un sensor prototipo con cuatro electrodos de características diferentes. Sobre este prototipo se llevaron a cabo mediciones electroquímica y se obtuvieron respuestas gobernadas por patrones específicos cada una de las sondas modelos que se utilizaron: \ferroferri, \ferroceno\space y \aminorutenio.   



\section*{Perspectivas para futuros trabajos}


Avanzando un poco mas allá de los objetivos de la tesis y aprovechando los métodos de baja temperaturas desarrollados se realizaron pruebas en las cuales se pretende generar patrones arbitrarios con las películas de mesoporoso. De esta forma se podrían realizar capas con películas mesoporosas de distintos óxidos específicamente sobre el área de cada electrodo. Las pruebas en este sentido, si bien preliminares, arrojaron resultados prometedores

Sin duda, los novedosos métodos posdepósito y el refuerzo por la adición de circonio, sumado al meticulosos estudio electroquímico que se presentó en este trabajo abrirán caminos poco explorados hasta el momento.

Se presentan al menos dos líneas claras para continuar con posibles investigaciones. La primera relacionada al estudio de la fisicoquímica en espacios confinados, transporte, cinética de adsorción/desorción, efecto de las funcionalizaciones, por citar algunos. Y la segunda, relacionada con las aplicaciones en sensores, de forma de establecer patrones de respuesta para analitos de interés debido a múltiples funcionalizaciones sobre los electrodos y parametrizar dichos patrones en función de variables como el potencial, la intensidad en cada de electrodos, la evolución con los ciclos, el pH y demás condiciones de contorno.



























