%Linea Para poder completar automaticamente las citas con el Sublime
%No hace el documento, se puede borrar esta linea si no se usa el Sublime
%------------------------------------------------------------------------------
 \newcommand{\NoBiblioIntro}[1]{
 \ifthenelse{\equal{#1}{verdadero}}{}{\bibliography{Referencias/base_bibliografica}}
 \NoBiblioIntro{verdadero}}
 %-----------------------------------------------------------------------------

%Formato (Nombre de capitulo largo o corto), nombre del capitulo y estilo de la
%Portada del Capitulo
%------------------------------------------------------------------------------

 %Formato en si, titulo en un solo renglon
 \FormatoCapituloUnaLinea

 %Nombre y etiquete para referir
 \chapter{Introducción}\label{chap:Introduccion}
 %\label{chap:Introduccion}

 %Para que no salga el numero de pagina en la portada del capitulo
 \thispagestyle{empty}
	
 %Resumen del Capitulo en Italica
 \noindent\textit{la Intro}

 
 %Indice de capitulo alineada al borde inferior de la pagina, nueva pagina
 \vfill
 \minitoc
 \newpage
 %-------------------------------------------------------------------------------

\section{Breve reseña sobre nanotecnología}

	En su conferencia, \textit{<<There's Plenty of Room at the Bottom>>} del 29 de diciembre de 1959, el físico Richard Feynman, considerado por muchos el padre de la nanotecnología, sugirió que se podría escribir toda la enciclopedia británica en la cabeza de un alfiler. Dicha presentación fue sin duda inspiradora para muchos de los presentes y anterior al uso generalizado de las técnicas de microscopia electrónica y a pocos años de la fabricación del primer transistor, el cual estaba muy lejos de convertirse en la unidad básica de la calculo computacional, de unas decenas de nanométros cuadrados.\cite{Feynman1959}  Años más tarde Taniguchi incorpora por primera vez el termino nanotecnología para describir procesos de microfabricación como deposito de películas delgadas o <<ion millling>>, y lo define como aquellos procesos de separación, consolidación y deformación de los materiales átomo por átomo o molécula por molécula. \cite{taniguchi1974} Fue Drexler quien finalmente popularizó el termino en su libro \textit{<<Engines of Creation: The Coming Era of Nanotechnology>>}\cite{drexler1987}. Una definición más amplia y actual para nanotecnología podría ser el desarrollo tecnológico de estructuras y sistemas en una escala nanométrica (entre 1 y 100 nanómetros). 


	La nanotecnologia se caracteriza fundamentalmente por ser multidisciplinaria y abarcar muchas áreas del conocimiento, ciencia de materiales, química, física, biotecnología, microelectrónica, etc. interactuando entre sí para generar más conocimiento básico y aplicado y favorecer  desarrollos de prototipos para, eventualmente acabar en productos de consumo. 

	Existen dos enfoques posibles para obtener estructuras y objetos en la nanoescala. El primero se trata de realizar estructuras por grabado o maquinado de un material, para llevarlo a las dimensiones nano. Esta aproximación se denomina de <<arriba hacía abajo>> o más conocida como \textit{top-down}. Pertenecen a este grupo las técnicas de fotolitografía, grabado por vía húmeda o seca y gran parte de la tecnología del silicio se basa en esta aproximación (es decir toda la electrónica actual de consumo masivo incluyendo computadoras y dispositivos móviles). El segundo enfoque es aquel denominado de <<abajo hacía arriba>> o \textit{bottom-up} el cual consiste en la construcción de objetos a partir de bloques fundamentales, los cuales pueden ser átomos o moléculas. A este grupo pertenecen las técnicas químicas de síntesis de nanopartículas, nanobarras y películas delgadas, métodos de atoensamblado y química supramolecular; también técnicas de crecimiento en fase vapor: epitaxial, \textit{physical vapour deposition (PVD)}, \textit{chemical vapour deposition} y \textit{atomic layer deposition (ALD)}.

	El fuerte impacto que tiene hoy esta disciplina no es sólo por el hecho de miniaturizar en sí, sino por las propiedades y características nuevas y/o diferenciales que surgen de dicha miniturización, ya sean cambio en las propiedades ópticas, eléctricas, magnéticas o mecánicas. Faraday fue uno de los primeros científicos en sugerir que, en la escala nanométrica, el cambio en las propiedades de la materia está ligado al tamaño, estudiando el cambio de color en coloides de Au\cite{faraday1857}. Es por este aporte novedoso y por la alta grado de miniturización e integración que esta disciplina no deja de crecer y sus contribuciones a la vida cotidiana se encuentran en medicina, electrónica, telecomunicaciones, sensores, materiales aeroespaciales, envases, pinturas, cosmética, atomoviles, aceros y la lista no parece tener fin. 


	convergencia bottom, - top para cualquier desarrollo nanotecnologico.

	Nanociencia o nanotecnologia?

	Los paises que estan a la cabeza de la nanociencia es de esperar, por supuesto, que esten a la cabeza de la nanotecnologua. 
	Graficos de scopus

	Productos anotados.

% 	Esta tesis, que tiene por objetivo principal fabricar sensores, . Materiales autoensamblados, microfabricación y electroquimica. 
% 	El diagrama de conjuntos de la figura 1, muestra las caracteristicas que 

% 	Dos enfoques clasicos de arriba hacia abajo ....... de abajo hacia arriba.

% 	Aca va el diagrama para explicar como aporta cada area.


% 		\begin{center}
% 			\includegraphics[width=0.75\textwidth]{Esquemas/concepto-interdiplinario.pdf}
% 		    \end{center}

 
% pongo estas secciones en la intro:
% Se expone breve,ente a cotinuacion los fundamentos teoricos de cada una de las areas etmaticas tratados a lo largo de la tesis que nos llevaran finalmente, en su conjunto a sentar las bases para fabricar microsensores electroqquimicos a base de materiales mesoporoso.

% 1) mesoporoso

% 2) eq

% 3) microfa

% Citas:


% in Pasadena on December 1959
% 1) de complejo a ser algo sencillo \\
% 2) trabajo en INTI, aplicado....
% 3) Nanotecnologia.... combiancaion microfab con sol-gel... tema central
% 4) conjunto interseccion entre las 3 patas, EQ, microfab y sol-gel

\section{Materiales Mesoporososos}\label{sec:mesoporosos}

				EL porque se elijo oxido de silicio, porque F127.... muy importante!
				
				Esta etapa del trabajo involucró la síntesis por sol-gel de películas delgadas de óxido de silicio mesoporoso. La película de oxido es la base de cual partimos para construir la <<palicula activa>>, es por ello que es de suma importancia escoger los elementos fundacionales de esta película. \cite{Soler-Illia2002a,Brinker1999,Soler-Illia2006,Grosso2004,Innocenzi2013}


				\begin{enumerate}
					\item El óxido que define las propiedades estructurales (Morfología, cristalinidad, espesor), ópticas y eléctricas.
					\item El tamaño, estructura y caracteristicas de los poros.
				\end{enumerate}

				Vamos a repasar porque elegimos el óxido de silicio, y no otros oxido de metales tales como Ti, Zr, Al, los cuales se han demostrado que son propicios para hacer estructuras mesoporoosas. El SiO$_2$ es aislante eléctrico y no absorbe en el rango UV/VIS. Estas dos características son fundamentales para los sensores, si bien la primera es común a la mayoritaria de los óxidos de transición, algunos de ellos presentan propiedades de semiconductores, 

				Descripcion de como se sintetizan las peliculas, preparacion de los soles, porque no se usa CTAB ni brij, ni Titanio oxido (quimica del silicio mas rica, mas economica, no absorbe en el UV/VSI)

				Descripcion de Spin - Coating (una sola cara, facilmente escalable, integrable a la industria microelectronica, ventajas de utilizar spin en lugar de dip, (pero tambien se puede utilizar dip, sobre todo para piezas no planas y de mayor volumen)

				Proceso de calcinacion. Despoito sobre Au, Vidrio y Solicio.

				Porque: gran area superficial, escablables, bajos costos, tuneable como filtro por tamaño de poro, quimicamente facil de modicar la superfie, optica adsocion en el UV (por esto no se puede de Ti02, pero si de ZrO4) despoitable por inkjet\cite{Lian2013,Mougenot2006a}, spin (microelectronica), dip (superficies de dificil geometria.)
				Aplicaciones en microfluidica \cite{schmuhl2005,Martinez2009}
				
	\subsection{sol-gel}	

\section{Sensores electroquimicos}

Portabilidad

\section{Microfabricacion}\label{sec:microfabricacion}
apilamiento de conocimientos, de cosas complejas a cosas simples en cuanto a la funcion. barato, de uso facil.
fenomenos complejo y caro..... Au, sondas caras, silicio... etc -> comprendimientos, comprension,comportamientos -> papel, impresion, sol-gel, \cite{Whitesides2015,Burdass2010}microfabricacion.

``''Todo deberia hacerse tan simple como sea posible, pero no mas que eso``'' Einsten.

`` There's Plenty of Room at the Bottom''

\url{https://www.ted.com/talks/george_whitesides_toward_a_science_of_simplicity?language=es}\cite{ted_whitesides2010}

Porque se elijio Au, Electroquimica, etc.

Sputt: explicar sputt, fotolito porque microelectronica, MEMS, sensores.
En los casos que se depositó la capa dieléctrica de SiO$_2$ se hizo con la fuente de radiofrecuencia (RF), mientras que los depósitos de las películas metálicas se realizaron con la fuente de corriente directa (DC), ambas configuradass a potencia constante, a P=\SI{400}{\watt}.  De esta forma se deja libre la tensión y la corriente, parámeros que dependen a su vez del vacío en la cámara, de la distancia entre el cátodo y el ánodo y el caudal de argón.\cite{sigmund1968}. 

\subsection{Fotolitografia}\label{sec:intro_fotolito}

\subsection{sputtering}

\section{implementacion tecnologica}
nanotecnologia\cite{Gimenez2017}
Intergrar bottom-up, top-down y hacer un dispositivo intergrado, miniaturizado, escalado, industrializable, tecnologicamente compatible, IC, logica, sensores MEM
Integracion, todo en argentina, valor agregado del proceso sol-gel.\cite{Volksen2010}

\section{Aplicaciones}

\section{Objetivos}