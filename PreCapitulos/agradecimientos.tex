\vfill
\cleardoublepage

\phantomsection\mtcaddchapter[Agradecimientos]	

\section*{\centering Agradecimientos} 

 \noindent Esta es la última página que escribo de este documento. Aquí concluyen muchos años de trabajo dedicado a elaborar esta tesis. Ahora me toca agradecer a todos los que aportaron para culminarlo, ¡espero no olvidarme de ninguno!

	 \begin{sangria_pers}{0.5cm}

	 A Galo y Gabriel, los directores de este trabajo. A Galo por todas las reuniones las cuales siempre resultaron inspiradoras, por su amplia visión científica, por su mirada integradora del trabajo y por sus correcciones. \linebreak A Gabi por el apoyo incondicional, la gran dedicación, su avidez para profundizar y discutir resultados, la capacidad de diseñar experimentos concluyentes y por sus comentarios siempre acertados.

	 Quiero hacer un agradecimiento muy especial a Lili a quien considero un engranaje fundamental de este trabajo. Me dio todas las libertades para hacerlo, las herramientas, me apoyó, me incentivó y me exigió resultados ¡espero que le guste como quedó!

	 A Mario Tagliazucchi por el gran y excelente aporte que hizo con las simulaciones electroquímicas y por las mediciones a altas velocidades de barrido, ambas contribuciones cruciales para desatar más de un nudo.  

	 A Cecilia Fuertes y Paula Angelomé porque me iniciaron en la síntesis de películas mesoporosas y por las largas jornadas de mediciones en el elipso.

	 A Alejandro Wolosiuk por las sugerencias y recomendaciones sobre cómo funcionalizar los electrodos.

	 A Paula Steinberg y Sebastian Alberti por las comparaciones de resultados y las colaboraciones mutuas.

	 A Claudia Marchi por las imágenes SEM.

	 A Matías Calderon por los espectros XPS.

	 A Alejandra que siempre me sacó todas las dudas, me mando los documentos, resoluciones, comunicados, me orientó con los papeles, etc, etc. ¡Una genia!

	 A toda la gente y compañeros del INTI, que en más de un sentido contribuyeron a este trabajo, especialmente con los que comparto laboratorio cotidianamente desde hace años y siempre están en el día a día, Mariano, Alex, Laura, Omar, Miji, Pipi, Chelo, Ani, Lea, Lea T. (¡sigue siendo del CMNB!), Mati, Charly, Nehuen, Luchito, Kuo, Pao, Sandri, María, Hernán, Lucho, Juli, Eli, Pablo, Fabi, Brunoloti, Bruno, Salva, Diego y Rodrigo.

	 Al Instituto Nacional de Tecnología Industrial, en el cual trabajo hace más de 10 años, que me brindó los medios materiales, el espacio y que apostó a este proyecto. A este INTI en el que siempre me sentí cómodo y me gustó trabajar, que siempre me ofreció proyectos y me recibió apenas recibido de químico, este INTI que esta atravesando un momento triste de su historia, vapuleado, donde se despide personal sin causa, desfinanciado, desmantelado... a este INTI le quiero agradecer profundamente y espero que sepa sobrevivir mas allá de todos los esfuerzo que están haciendo por llevarlo a su mínima expresión.  
	 

\pagebreak\thispagestyle{empty}	

  A la educación pública y gratuita, en la cual me formé desde el jardín de infantes hasta el doctorado. Sin lugar a dudas uno de las mejores baluartes de este país.\medskip

  También quiero agradecer a los lectores de esta tesis, espero que les sea una lectura inspiradora y de utilidad.\bigskip

     \end{sangria_pers}

 \noindent Quiero también hacer algunos agradecimientos en el plano de mi vida personal:\medskip 

		\begin{sangria_pers}{0.5cm}

		A la banda de amigos con los que hice toda la carrera, Esteban, Tomy, Dami, Coni, Cris, Ani (¡repetida!), Gloria y Ani W.\medskip

		A Nico ¡compañero de toda la vida!\medskip 

		A mis tíos Julio y Olga, Horacio y Silvia y a mis queridos primos Emilio y Guille.\medskip

	    A mi mamá Mirta y a mi papá Enrique que siempre me dieron lo mejor.\medskip

		A mis hermanas Vero y Pau y a mi hermano Agus que son el apoyo incondicional en todo momento.\medskip 
		
		A mis sobrinas y sobrinos ¡More, Juli, Manu, Cande y Rafa!\medskip

		A ese espacio maravilloso y lleno de magia que es la Quinta.\medskip

		Y por último quiero dedicarle especialmente este trabajo a Mariana, Gala a Lucía que me bancan siempre y me llenan de alegría día a día.

		\end{sangria_pers}

\begin{figure}[b]
\tikz[remember picture,overlay] \node[opacity=0.5,scale=0.6] at (6,4.5){\includegraphics{Imagenes/inti-corazon.png}};
\end{figure}

\cleardoublepage
