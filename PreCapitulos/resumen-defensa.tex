Este trabajo de tesis tuvo dos objetivos fundamentales. El primero fue sintetizar y estudiar películas delgadas mesoporososas (PDM) basadas en oxido de silicio y oxidos mixtos de silicio y circonio para utilizar como elemento permeoselectivo para un analito o grupo de analitos. El segundo fue la integración de dichas películas en sensores electroquímicos fabricados con tecnologías de microfabricación, compatibilizando de esta forma los procesos \textit{bottom-up} con los \textit{top-down}.

A lo largo de este desarrollo surgieron incógnitas y preguntas que se fueron a analizando y discutiendo a lo largo de la tesis: ¿es posible combinar procesos ``bottom-up'' con ``top-down''? ¿Qué fenómenos de transportes son los que tienen lugar dentro de las películas? ¿que elementos se pueden incorporar para modificar la permeoselectividad? ¿Cómo se puede mejorar la estabilidad química de los sensores? ¿Se puede generar una marca sensorial por analito sin recurrir a moléculas de reconocimiento específico?

Los resultados obtenidos a partir de estos interrogantes permitieron escalar la fabricación de sensores basados en electrodos de oro recubiertos con PDM potencialmente selectivas, químicamente estables, reproducibles y de fácil funcionalización. El conocimiento adquirido es la base que permitirá avanzar en el desarrollo y diseño de multisensores electroquímicos integrados basados en películas delgadas mesoporosas permeoselectivas.

