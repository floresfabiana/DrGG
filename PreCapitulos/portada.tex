\begin{titlepage}
		Portada
		% %Universidad
		% \begin{center}
		% \sc{\LARGE{Universidad de Buenos Aires\\}%\vspace*{0.5cm}
		% \Large{Facultad de Ciencias Exactas y Naturales\\}%\vspace*{0.5cm}
		% \large{Departamento de Química Inorgánica, Analítica y Química Física\\}}%\vspace*{1cm}}
		% \end{center}

		% %Logo Facultad
		% \begin{figure}[ht!]
		% \centering
		% \includegraphics[width=0.4\textwidth]{Imagenes/logo-exactas.jpg}%\vspace*{0.5cm}
	 % 	\end{figure}
	 % 	%Titulo

	 % 	\begin{center}
	 % 	\large{\bfseries{Estudio de propiedades fisicoquímicas y de transporte en películas delgadas de SiO$_2$: integración y bases para su uso en sensores electroquímicos}} \\ %\vspace*{0.3cm}
	 % 	\small{Tesis presentada para optar al título de Doctor de la Universidad de Buenos Aires en el área de química inorgánica, analítica y química física} \\ %\vspace*{0.3cm}
	 % 	\large{\bfseries{Gustavo Giménez}} \\ %\vspace*{1cm}
	 % 	\end{center}
		
		% %Director
		% \setlength\tabcolsep{1.5pt}
		% \noindent\begin{tabular}{@{}ll}
		% Director de tesis: & Dr. Juan de Ávila Arturo Galor Soler Illia \\ 
		% 				   & Dr. Gabriel Ybarra \\						
		% \end{tabular} \\
		% \noindent\begin{tabular}{@{}ll}
		% Consejero de Estudios: Dr. Darío Estrin \\
		% \end{tabular} \\
		% \noindent\begin{tabular}{@{}ll}
		% Lugar de trabajo: &Centro de Micro y Nanoelectrónica del Bicentenario \\
		% 				  & Instituto Nacional de Tecnología Industrial (INTI-CMNB) \\ 
		% \end{tabular} 
		% \setlength\tabcolsep{6pt}

		% %Fecha contra abajo				  
		% \vspace*{\fill}
		% \begin{center}
		% \small{Ciudad Autónoma de Buenos Aires, 2017}
	 % 	\end{center}

	 % 	%Logo de INTI?????

\end{titlepage}