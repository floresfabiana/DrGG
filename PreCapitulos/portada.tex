\begin{titlepage}

%Portada

		%Logo Facultad
		\begin{figure}[t!]
		\centering
		\includegraphics[width=0.4\textwidth]{Imagenes/Logo-Exactas.pdf}%\vspace*{0.5cm}
	 	\end{figure}
	 	%Titulo

		%Universidad
		\begin{center}
		\sc{\LARGE{Universidad de Buenos Aires\\}\vspace*{4mm}
		\Large{Facultad de Ciencias Exactas y Naturales\\}\vspace*{2mm}
		\large{Departamento de Química Inorgánica, \\ Analítica y Química Física\\}} 
		\vspace*{1cm}
		\end{center}

		

	 	\begin{center}
	 	\large{\bfseries{Fabricación y caracterización de arreglos de electrodos recubiertos con películas delgadas mesoporosas de óxido de silicio y óxidos mixtos de silicio y circonio}} \\ \vspace*{0.5cm}
	 	\normalsize{Tesis presentada para optar al título de Doctor de la Universidad de Buenos Aires en el Área de Química Inorgánica, Química Analítica y Química Física} \\ \vspace*{1.2cm}
	 	
	 	\Large{\bfseries{Gustavo Giménez}} \\ 
	 	\end{center}
		
		\vspace*{2cm}

		%Director
		\setlength\tabcolsep{1.5pt}
		\noindent\begin{tabular}{@{}ll}
		Directores de tesis:&Dr. Gabriel Ybarra\\  %Abreviar nombres
		 &Dr. Juan de Ávila Arturo Galo Soler Illia\\						%Falta segundo nombre
		\end{tabular} \\
		\noindent\begin{tabular}{@{}ll}
		Consejero de Estudios: Dr. Darío Estrin \\ %Falta segundo NOmbre!!!!
		\end{tabular} \\ 

		\noindent\begin{tabular}{@{}ll}
		Lugar de trabajo: &Centro de Micro y Nanoelectrónica del Bicentenario \\
						  & Instituto Nacional de Tecnología Industrial (INTI-CMNB) \\ 
		\end{tabular} 
		\setlength\tabcolsep{6pt}

		%Fecha contra abajo				  
		\vspace*{\fill}
		
		\noindent\small{Buenos Aires, 2018}
	 	

	 	%Logo de INTI?????

\end{titlepage}