\phantomsection\mtcaddchapter[Resumen]

\section*{\centering Resumen}

%noindent Este trabajo tiene por objet
Los sensores químicos han cobrado cobrado importancia en las últimas décadas, obteniendo una mayor fracción de mercado. En particular, el subgrupo de sensores electroquímicos es una de los más ha crecido. Dicho éxito se debe a la combinación de varios factores. Obtener directamente una señal eléctrica la cual no requiere de transductores, se pueden fabricar mediante procesos de la industria electrónica intengrandolos facimelmente en dispositivos electrónicos, poseen un limite de detección bajo (sasas), no requieren alineación y con el uso racional de intermediarios o una adecuada funcionalización de los electrodos se pueden conseguir sensores altamente específicos, por ejemplo, mediante el empleo de ADN o proteínas.

El siguiente trabajo de tesis tuvo dos objetivos fundamentales. El primero fue la síntesis  y el estudio de películas delgadas mesoporososa (PDM) basadas en oxido de silicio obtenidas por sol-gel para utilizar como elemento de selección especifico de un analito o grupo de analitos; el segundo fue la integración de dichas películas en sensores electroquímicos fabricados con tecnologías de microfabricación, compatibilizando de esta forma los procesos \textit{bottom-up} con los \textit{top-down}.

Durante la primera etapa se sintetizaron PDM sobre electrodos de oro, empleando diferentes surfactantes para regular el tamaño de los poros entre 3 y \SI{10}{\nm}. Se empleando procesos alternativos a la calcinación, permitieron condensar el SiO$_2$ por debajo de los \SI{130}{\celsius}, disminuyendo los efectos difusivos y ampliando el espectro de sustratos compatibles. Las PDM se caracterizaron utilizando elipsoporosimetría ambiental, espectroscopia IR, microscopía óptica, de barrido electrónico y de iones de galio.

En una segunda etapa se estudiaron los fenómenos de transporte que ocurren dentro de las PDM mediante técnicas electroquimicas y de simulación por elementos finitos. 

\vfill

\noindent\textbf{falabras clave:}Peliculas delgadas, oxidos mesoporosos, microfabricacion, sensores

\cleardoublepage
