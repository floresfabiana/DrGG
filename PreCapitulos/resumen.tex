\cleardoublepage

\phantomsection\mtcaddchapter[Resumen]

%\section*{\centering\large{\bfseries{Estudio de propiedades fisicoquímicas y de transporte en películas delgadas mesoporosas basadas en SiO$_2$. Integración y aplicación en sensores electroquímicos}}}
\section*{\centering\large{\bfseries{Fabricación y caracterización de arreglos de electrodos recubiertos con películas delgadas mesoporosas de óxido de silicio y óxidos mixtos de silicio y circonio}}}

%\vspace*{\fill}
%noindent Este trabajo tiene por objet
\vspace*{\fill}

Los sensores químicos han cobrado importancia en las últimas décadas. En particular, el subgrupo de sensores electroquímicos es uno de los campos que más ha crecido. 

Este trabajo de tesis tuvo dos objetivos fundamentales. El primero fue sintetizar y estudiar películas delgadas mesoporososas (PDM) basadas en SiO$_2$ para utilizar como elemento permeoselectivo para un analito o grupo de analitos. El segundo fue la integración de dichas películas en sensores electroquímicos fabricados con tecnologías de microfabricación, compatibilizando de esta forma los procesos \textit{bottom-up} con los \textit{top-down}.

Durante la primera etapa se sintetizaron PDM sobre electrodos de oro, empleando diferentes surfactantes para regular el tamaño de los poros entre 3 y \SI{10}{\nm}. Dado que la remoción del molde de poros mediante calcinación deteriora los electrodos, se desarrollaron procesos alternativos a la calcinación, permitiendo condensar el SiO$_2$ por debajo de los \SI{130}{\celsius}, minimizando efectos difusivos y ampliando el espectro de sustratos compatibles. Las PDM se caracterizaron mediante elipsoporosimetría ambiental, espectroscopía IR, microscopía óptica, de barrido electrónico y de iones de galio.

En una segunda etapa se estudiaron los fenómenos de transporte que ocurren a través de las PDM mediante técnicas electroquímicas: voltametría cíclica, voltametría de corriente alterna y simulación por elementos finitos. Los resultaron permitieron obtener información significativa de los sistemas, tanto cualitativa (casos de exclusión, permeación y preconcentración) como cuantitativa (coeficientes de difusión, capacidad de adsorción, distancias entre sitios rédox, etc.). Sin embargo, se observó que las PDM de sílice se disuelven en el término de horas durante las medidas electroquímicas, lo que limita su aplicabilidad en el caso de sensores para uso continuo.

%La motivación principal de la tesis estuvo dirigida a la fabricación de multisensores electroquímicos. Es por ello que 

Se fabricaron en forma continua y constante electrodos de Au, optimizando diseños, condiciones de depósito, sustratos y funcionalizaciones superficiales, de forma de mejorar el desempeño electroquímico y aumentar la compatibilidad de las PDM con los electrodos. Ya en una etapa avanzada se incorporó circonio a los soles lo que redunda en un aumento significativo de la estabilidad química de las PDM de sílice frente a la disolución. Finalmente se fabricó un multisensor prototipo funcionalizando las PDM en forma localizada específicamente sobre el área de cada electrodo con la intención de darle una característica distintiva a cada elemento del sensor. Sobre este prototipo se realizó un análisis multivarible de la respuesta electroquímica de cada electrodo de forma de obtener marcas sensoriales especificas para cada sonda utilizada.

Los resultados obtenidos permiten escalar la fabricación de sensores basados en electrodos de oro recubiertos con PDM potencialmente selectivas, químicamente estables, reproducibles y de fácil funcionalización. El conocimiento adquirido es la base que permitirá avanzar en el desarrollo y diseño de multisensores electroquímicos basados en películas delgadas mesoporosas permeoselectivas. 
\vspace*{\fill}

\vfill
%\vspace*{3mm}
\noindent\textbf{Palabras claves:} películas delgadas mesoporosas, SiO$_2$, ZrO$_2$, electrodos de oro, microfabricacion, sensores electroquímicos, multisensores.

\cleardoublepage
