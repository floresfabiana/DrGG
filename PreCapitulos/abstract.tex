\phantomsection\mtcaddchapter[Abstract]

\section*{\centering\large{\bfseries{Fabrication and characterization of electrodes array with a mesoporous thin film coating based on silicon and zirconium/silicon oxide}}}

\vspace*{\fill}

Chemical sensors have gained relevance in recent decades, particularly the subgroup of electrochemical sensors is one with the bigger growing rate.

The following thesis work had two fundamental goals. The first one, was to synthesize and study mesoporous thin films (MTF) based on silicon oxide to be used as permeoselective element for an analyte or group of analytes. The second was the integration of these films in electrochemical sensors manufactured with microfabrication technologies, thus compatibilizing the bottom-up and top-down processes strategies.

During the first stage, MTF was synthesized upon gold electrodes, using different surfactants to tailored the pore size between 3 and \SI{10}{\nm}. Alternative processes were developed for the calcination, allowing to condense the SiO$_2$ below the \SI{130}{\celsius}, minimizing diffusive effects and expanding the spectrum of compatible substrates. The TMFs were characterized by environmental ellipsoporosimetry, IR spectroscopy, optical microscopy, scanning electron microscopy and focused ion beam microscopy.

In a second stage, the transport phenomena that have place through the MTP were studied using electrochemical techniques: cyclic voltammetry, alter current voltammetry and simulation by finite elements. The results allowed to obtain significant information of the systems, qualitative (cases of exclusion, permeation and preconcentration) and quantitative (diffusion coefficients, adsorption capacity, distances between redox sites, etc.).

The main motivation of the thesis was centered in the manufacture of sensors. For this reason, Au electrodes were continuously and constantly deposited, optimizing severals designs, spinning conditions, substrates and functionalized surface, in order to improve the electrochemical performance and improve the adhesion of the MTF to the electrodes. At an advanced stage, zirconium was incorporated into the precursor solutions to increase the chemical stability of the silicon MTFs against dissolution, and finally these MTFs were functionalized with the intention of giving a differential  characteristic to each of the electrodes inside into the same sensor.

The results obtained allow to scale the manufacturing of sensors based on gold electrodes coated with MTF, obtaining potentially selective films, chemically stable, reproducible and easy to functionalize. The acquired knowledge is fundamental and will be allow us to advance in the development and design of electrochemical sensors based on selective mesoporous thin films.
\vspace*{\fill}

\vfill

\noindent\textbf{Keywords:} thin film mesoporous, SiO$_2$, ZrO$_2$, gold electrode array, microfabrication, electrochemical sensors.

\cleardoublepage

