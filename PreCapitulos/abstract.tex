\cleardoublepage

\phantomsection\mtcaddchapter[Abstract]
\section*{\centering\large{\bfseries{Fabrication and characterization of electrodes array with a mesoporous thin film coating based on silicon and zirconium/silicon oxide}}}

\vspace*{\fill}

Chemical sensors have gained relevance in the last decades, particularly the subgroup of electrochemical sensors is one with the larger growth rate.

This work had two fundamental goals. The first one was to synthesize and study mesoporous thin films (MTF) based on SiO$_2$ to be used as perm-selective elements for an analyte or group of analytes. The second was the integration of these films in electrochemical sensors manufactured with microfabrication technologies, thus compatibilizing the bottom-up and top-down processes strategies.

During the first stage, MTF were deposited on gold electrodes, using different surfactants to tailor the pore size between  3 and \SI{10}{\nm} . Thermal treatments used for template removal negatively affect the electrode response. Therefore, alternative processes were developed for the calcination, allowing to condense the SiO$_2$ below \SI{130}{\celsius}, minimizing diffusive effects and expanding the spectrum of compatible substrates. The TMFs were characterized by ellipsoporosimetry, IR spectroscopy, optical microscopy, scanning electron microscopy and focused ion beam microscopy.

In a second stage, the transport phenomena that take place through the MTP were studied using electrochemical techniques: cyclic voltammetry, alternating-current voltammetry and simulation by finite elements. The results allowed obtaining significant information of the systems, qualitatively (cases of exclusion, permeation and preconcentration) and quantitatively (diffusion coefficients, adsorption capacity, distances between redox sites, etc.). However, silica MTF dissolve after several electrochemical cycles, thus limiting the use of these systems as continuous sensors.

The main motivation of the thesis was centered in the manufacture of electrochemically based sensors. For this reason, Au electrodes were deposited, optimizing several designs, spinning conditions, substrates and functionalized surface, in order to improve the electrochemical performance and to optimize the adhesion of the MTF to the electrodes. At an advanced stage, Zr(IV) was incorporated into the precursor solutions, which significantly increased the chemical stability of the silicon MTFs against dissolution. Finally, a prototype multisensor was manufactured functionalizing the MTF specifically over the area of each electrode with the intention of giving a distinctive feature to each element of the sensor. A multivariate analysis of the electrochemical response of each electrode was performed for this prototype in order to obtain an specific sensorial sign for each one of used probes.

The results obtained allow scaling the fabrication of sensors based on gold electrodes coated with PDM potentially selective, chemically stable, reproducible and easy to functionalize. The knowledge acquired is the basis for advancing the development and design of electrochemical multisensors based on permeoselective mesoporous thin films.

\vspace*{\fill}

\vfill

\noindent\textbf{Keywords:} mesoporous thin film, SiO$_2$, ZrO$_2$, gold electrode array, microfabrication, electrochemical sensors, multisensors.

\cleardoublepage

