%Linea Para poder completar automaticamente las citas con el Sublime
%No hace el documento, se puede borrar esta linea si no se usa el Sublime
%------------------------------------------------------------------------------
 \newcommand{\NoBiblioMat}[1]{
 \ifthenelse{\equal{#1}{verdadero}}{}{\bibliography{Referencias/base_bibliografica}}
 \NoBiblioMat{verdadero}}
 %-----------------------------------------------------------------------------

%Formato (Nombre de capitulo largo o corto), nombre del capitulo y estilo de la
%Portada del Capitulo
%------------------------------------------------------------------------------

 %Formato en si, titulo en un solo renglon
 \FormatoCapituloUnaLinea

 %Nombre y etiquete para referir
 \chapter{Materiales, Métodos y Procesos}\label{chap:Materiales}

 %Para que no salga el numero de pagina en la portada del capitulo
 \thispagestyle{empty}
	
 %Resumen del Capitulo en Italica 
  \noindent\textit{En este capitulo se presenta la descripción experimental de todos los materiales, instrumental y procesos involucrados en la tesis. En la primera sección se detallan los procesos de síntesis de las películas mesoporosas, desde la preparación de los soles\index{sol} hasta la caracterización de las mismas; la segunda muestra los procesos de fabricación de los electrodos, desde el diseño a las técnicas de transferencia; la tercera sección describe las técnicas de microscopia\index{microscopía}s utilizadas y la última sección detalla como se llevaron a cabo las mediciones electroquímica\index{electroquimico}s\index{electroquimico}.}


 %Indice de capitulo alineada al borde inferior de la pagina, nueva pagina
 \vfill
 \minitoc
 \newpage

 %-------------------------------------------------------------------------------

\section{Síntesis de películas delgadas mesoporosas}\label{sec:sintesis_mesoporosos}	
	
	 Las consideraciones teóricas sobre la química sol-gel\index{sol-gel} y el autoensamblado inducido por evaporación\index{autoensamblado inducido por evaporación} ya fueron mencionadas en el capítulo \ref{chap:Introduccion}. También fueron mencionadas las razones por las cuales fue elegido el óxido de silicio\index{silicio!oxido de}\index{silicio} y el Pluronic F127\index{Pluronic F127} como estructura para las películas delgadas mesoporosas y como agente moldeante\index{agente moldeante} respectivamente. Los procedimientos, métodos y proporciones molares para la preparación de los soles\index{sol} fueron en su mayoría adaptaciones de los utilizados por Angelomé\index{Angelomé}\cite{Angelome2008} y Fuertes\index{Fuertes}\cite{Fuertes2009}. El esquema \ref{esq:peliculas_meso} resume cada etapa de síntesis de las películas, que se explican con detalles en las próximas secciones.
		  \begin{figure}[ht]
			  \begin{center}
			  \includegraphics[width=\textwidth]{Esquemas/Resumen_sintesis_meso.pdf}
			  \caption[Síntesis de películas delgadas mesoporosas]{Diagrama de flujo para la síntesis de películas delgadas mesoporosas.}
			  \label{esq:peliculas_meso}
			  \end{center}
			  \end{figure}

	\subsection{Preparación de los soles, reactivos y nomenclatura}
		
			La síntesis y depósito\index{depósito} de las películas delgadas mesoporosas comienzan con la preparación de las soluciones, las cuales deben contener los precursor\index{precursor}es del óxido (o de los óxidos en el caso de películas mixtas), el agente moldeante\index{agente moldeante} de los poros, solvente adecuado, H$_2$O y HCl\index{acido@ácido!clohídrico}\cite{Brinker1990}. Cada uno de ellos cumple una función especifica. El precursor\index{precursor} del óxido es el que da la estructura a la película. Para este fin se utilizó tetraetoxisilano\index{tetraetoxisilano} (TEOS, \textit{Merck}) para las películas de sílice pura, y TEOS combinado con cloruro de circonio(IV) (ZrCl$_4$, \textit{Aldrich}) para las películas mixtas de silicio/circonio. Las condiciones de hidrólisis\index{hidrolisis@hidrólisis} y condensación\index{condensación} para estos dos reactivos (ya sean solos o combinados) son bien conocidas y llevan a la formación películas delgadas estables y reproducibles\cite{Soler-Illia2004,Crepaldi2002a,Angelome2008}. El surfactante\index{surfactante} es el agente que establece el tamaño de los poros, se utilizó para ello el copolímero de bloque Pluronic F127\index{Pluronic F127} (F127, \textit{Aldrich}) y bromuro de hexadeciltrimetilamonio\index{bromuro de hexadeciltrimetilamonio} (CTAB, \textit{Aldrich}). Como solvente se utilizó etanol\index{etanol} absoluto (EtOH, \textit{Sigma}). El H$_2$O es el reactivo para la formación del óxido, y por último, el HCl\index{acido@ácido!clohídrico} es el encargado de generar el medio ácido\index{acido@ácido} que cataliza la hidrólisis\index{hidrolisis@hidrólisis} del TEOS y/o del Zr\index{circonio}Cl$_4$. Los reactivos utilizados fueron de calidad proanalisis o superior y el H$_2$O de \SI{18}{\mega\ohm\per\cm} fue obtenida con un equipo \textit{Ultra Clear TWF} de la marca \textit{Siemmens}. La nomenclatura, pesos moleculares y estructura química de los reactivos utilizados se pueden consultar en la tabla \ref{tabla:reactivos}.
					
			El preparado de las soluciones se realizó agregando cada reactivo por pesada en balanza analítica. Cada lote de solución fue de aproximadamente \SI{100}{\ml}. Para llegar a este volumen se agregaron, en este orden, \SI{10.417}{\gram} de TEOS, \SI{6.911}{\gram} de etanol\index{etanol} y \SI{0.902}{\gram} de HCl\index{acido@ácido!clohídrico} \SI{2,77e-3}{\Molar}. En el caso de los soles\index{sol} mixtos, se pesaron \SI{9.375}{\gram} de TEOS y \SI{1.165}{\gram} de Zr\index{circonio}Cl$_4$. Esta primera solución, denominada solución de prehidrólisis, se deja envejecer bajo agitación constante durante \SI{48}{\hour} a \SI{25}{\celsius}, con el objetivo de hidrolizar los precursor\index{precursor}es metálicos y mantener un bajo grado de condensación\index{condensación}.\cite{Grosso2001}

				\begin{table}[ht!] 
						  \caption[Reactivos para los soles]{Nomenclatura, estructura, peso molecular y función de las moléculas utilizadas en las soluciones para la síntesis de películas delgadas mesoporosas.} 
				  		  \begin{tabular}{>{\raggedright\arraybackslash}m{2.40cm}>{\centering\arraybackslash}m{4cm}>{\centering\arraybackslash}m{2.35cm}>{\raggedright\arraybackslash}m{1.7cm}} 
				  		  \toprule
						  Nombre Nomenclatura    & Estructura & Peso molecular \si{g.mol^{-1}} & Función\\ \midrule
				      	  tetraetoxisilano\index{tetraetoxisilano} TEOS & \includegraphics[scale=0.5]{Esquemas/teos.pdf} & $208,33$ & precursor\index{precursor} del óxido  \\ \midrule
				      	  \mbox{cloruro de circonio(IV)}  Zr\index{circonio}Cl$_4$ & \includegraphics[scale=0.8]{Esquemas/zrcl4.pdf} & $233.04$ & precursor\index{precursor} del óxido  \\ \midrule
				  		  Pluronic F127\index{Pluronic F127} F127    & \hspace*{-10px} \includegraphics[scale=0.5]{Esquemas/f127.pdf} & \multirow{1}{*}{$13800$}	 & agente moldeante\index{agente moldeante}	 \\ \midrule
				  		  bromuro de hexadeciltrimetilamonio\index{bromuro de hexadeciltrimetilamonio}  CTAB   & \hspace*{1cm} \includegraphics[scale=0.6]{Esquemas/ctab.pdf} & \multirow{1}{*}{$364.48$}	 & agente moldeante\index{agente moldeante}	 \\ \midrule
				  		  ácido\index{acido@ácido} clohídrico\index{acido@ácido!clohídrico} HCl\index{acido@ácido!clohídrico}& \includegraphics[scale=0.75]{Esquemas/hcl.pdf}  & \multirow{1}{*}{$36,46$}   & cataliza la hidrólisis\index{hidrolisis@hidrólisis} \\ \midrule
				  		  agua \hspace{2cm} H$_2$O  &  \includegraphics[scale=0.75]{Esquemas/agua.pdf}  & \multirow{1}{*}{$18,02$}   & reactivo de hidrólisis\index{hidrolisis@hidrólisis} \\ \midrule
				  		  etanol\index{etanol} \hspace{2cm} EtOH\index{etanol}  & \includegraphics[scale=0.75]{Esquemas/etanol.pdf}  & \multirow{1}{*}{$46,07$}   & solvente \\ 
				  		  \bottomrule
				    	  \end{tabular}
				   		  \label{tabla:reactivos}
					      \end{table}

			Una vez envejecida la solución de prehidrólisis\index{prehidrólisis} (ya sea de SiO$_2$ pura o mixta) se agregan a \SI{17.146}{\gram} de esta, \SI{80.184}{\gram} de EtOH\index{etanol}, \SI{3.246}{\gram} de F127 o \SI{1.822}{\gram} de CTAB y \SI{7.630}{\gram} de HCl\index{acido@ácido!clohídrico} \SI{2,5e-2}{\Molar}. De esta forma se obtienen unos \SI{100}{\ml} de un sol\index{sol} con las relaciones molares de la tabla \ref{tabla:soles}. Se conservan en \textit{frezeer} a \SI{-18}{\celsius} y solo se saca de allí a la hora de depositarlo. 

			Para facilitar la lectura se utilizará la siguiente nomenclatura, tanto para los soles\index{sol} como para las películas delgadas mesoporosas que se fabriquen con ellos: 

				\begin{itemize}
			 			\item \pdm\space para películas delgadas mesoporosas en general.
			 			\item \pdmF\space para \pdm\space de óxido de silicio\index{silicio!oxido de}\index{silicio} estructuradas con F127 
			 			\item \pdmC\space para \pdm\space de óxido de silicio\index{silicio!oxido de}\index{silicio} estructuradas con CTAB.
			 			\item \pdmZ\space para las \pdm\space mixtas de óxido de circonio\index{circonio} y silicio\index{silicio} en relación molar $1\!:\!9$ y estructuradas con F127. 
					    \end{itemize}	
			
				\begin{table}[ht]
			  		  \caption[Relación molares de los soles]{Relaciones molares para las soluciones utilizadas.} 
			  		  \begin{tabular}{>{\raggedright\arraybackslash}m{2.2cm}>{\centering\arraybackslash}m{2.2cm}>{\centering\arraybackslash}m{1.875cm}>{\centering\arraybackslash}m{1.875cm}>{\centering\arraybackslash}m{1.875cm}} 
			  		  \toprule
					  Componente & Prehidrólisis  & \pdmF   & \pdmC  & \pdmZ \\  \midrule
			      	  TEOS 		  & 1/0,9$^*$	  & 1   	& 1		 & 0,9   \\ \midrule
			      	  Zr\index{circonio}Cl$_4$	  & -/0,1$^*$	  &	-		& - 	 & 0,1   \\ \midrule	
			      	  EtOH\index{etanol} 		  & 3			  & 40   	& 40	 & 40    \\ \midrule
			      	  F127 		  & -		 	  & 0,0075  & -		 & 0,0075\\ \midrule
			      	  CTAB 		  & -             & -		& 0,1	 & 0,1   \\ \midrule
			      	  H$_2$O	  & 1			  & 9	  	& 9	     & 9     \\ \midrule
			      	  HCl\index{acido@ácido!clohídrico}    	  & 0,00005		  & 0,01   	& 0,01	 & 0,01   \\ 
			      	  \bottomrule
			    	  \end{tabular}\vspace*{2pt}
		    	  	  \footnotesize{$^*$Los números después de la barra son los utilizados en soluciones de prehidrolisis para películas mixtas de silicio/circonio.}
			    	  \label{tabla:soles}
			   		  \end{table}
			Todas las soluciones fueron preparadas indistintamente en el Centro de Micro y Nanoelectrónica del Bicentenario del Instituto Nacional de Tecnología Industrial (INTI-CMNB) o en la Gerencia Química, Centro Atómico Constituyentes Comisión Nacional de Energía Atómica (CAC-CNEA). 
				
	\subsection{Depósitos de las películas delgadas mesoporosas}\label{sec:deposito_pdm}

			Las películas mesoporosas utilizadas en esta tesis fueron depositadas en el Laboratorio de Fotolitografía del INTI\index{INTI}-CMNB por la técnica de \textit{spin-coating}\index{spin@\textit{spin-coating}}. El equipo utilizado fue un \textit{Suss MicroTec Delta 20BM},  el cual consiste en un cabezal rotatorio con control de aceleración de 0 a  \SI{1000}{\minute^{-1}.\second^{-1}} y de velocidad variable de 0 a \SI{10000}{\minute^{-1}}; tiene varios portamuestras para sustratos de diferentes tamaños con entrada de vacío para sujetar las muestras (figura \ref{fig:spin}). 
			
			Se utilizaron como sustrato\index{sustrato} para depositar las \pdm, vidrio, silicio, oro\index{oro} sobre silicio, microelectrodo\index{electrodo!microelectrodo}s y sustratos poliméricos como  polimetilmetacrilato (PMMA) y poliestireno de alto impacto (PAI). Cada uno de ellos fue escogido para una función particular (p. ej. sustrato\index{sustrato} para reacciones electroquímica\index{electroquimico}s\index{electroquimico}) o por alguna característica distintiva (p. ej. transparente en el IR). En la tabla \ref{tabla:sustratos}, pág. \pageref{tabla:sustratos}, se agrupan los sustratos utilizados y se resumen algunas características y funciones destacadas.

			Las dimensiones de las muestras fueron típicamente de \SI{1x1}{\cm} a \linebreak \SI{2x2}{\cm}, aunque la técnica permite depositar películas continuas de hasta \SI{15}{cm} de diámetro. En algunos casos, para obtener un lote de sensores\index{sensor} de más cantidad, se utilizaron obleas de silicio\index{silicio} de \SI{10}{\cm} de diámetro. Antes de hacer el depósito, el sol\index{sol} se pasa a través de un filtro de jeringa\index{jeringa} de \SI{0.45}{\um} para evitar discontinuidades y/o <<cometas>> en los depositos\cite{Franssila2004}. Luego, para dispensar el sol\index{sol} en el sustrato, se utilizaron pipetas tipo Pasteur\index{Pasteur} o pipetas automáticas dependiendo del volumen requerido, el cual varió de 80 a \SI{100}{\uL.\cm^{-2}}. Las condiciones del laboratorio durante el depósito\index{depósito} se mantuvieron en \SI{25}{\celsius} y a una HR entre 30\% y 50\%. Una vez dispensado el sol, se da comienzo a la rotación que dispersa la solución de manera homogénea sobre el sustrato\index{sustrato} y, a su vez, la evaporación del solvente promueve la formación del cristal líquido\index{cristal líquido} por el mecanismo de autoensamblado molecular inducido por evaporación (EISA, del inglés \textit{evaporation induced self-assembly}) \cite{Brinker1999}.

					\begin{figure}[ht!]
					  \begin{center}
					  \includegraphics[width=\textwidth]{Imagenes/Spin.jpg}
					  \caption[Equipo para el depósito\index{depósito} de películas delgadas, \textit{spin-coater}]{\textit{Spin-coater} ubicado en el Laboratorio de Fotolitografía del INTI\index{INTI}-CMNB utilizado para el deposito de las películas delgadas mesoporosas, Marca \textit{Suss MicroTec}, modelo \textit{Delta 20BM}.}
					  \label{fig:spin}
					  \end{center}
					  \end{figure}

			El espesor\index{espesor} de la película, ($t$), es inversamente proporcional a la raíz cuadrada de la velocidad angular\index{velocidad!angular} ($\omega$), es decir $t\propto \omega ^{-1/2}$, por lo que las rampas de velocidad y aceleración utilizadas fueron optimizadas para lograr espesor\index{espesor}es entre 150 y \SI{300}{\nm}\cite{Meyerhofer1978,Hall1998,Brinker1990}. Los esquemas aplicados se muestran en gráfico de la figura \ref{fig:rampa-spin}. 
				
					\begin{figure}[!ht]
						 \begin{center}
						 \includegraphics[width=0.70\textwidth]{Graficos/rotacion_meso.pdf}
						 \caption[Parámetros de depósito\index{depósito} para las \pdm]{Esquema con las rampas más frecuentes de aceleración, velocidad y tiempo utilizadas para el depósito\index{depósito} de \pdm.}
						 \label{fig:rampa-spin}
						 \end{center}
						 \end{figure}

			 	    \begin{table}[ht!]
			  		   \caption[Sustratos utilizados para el depósito\index{depósito} de \pdm]{Sustratos utilizados para el depósito\index{depósito} de \pdm.} 
			  		   \begin{tabular}{>{\raggedright\arraybackslash}m{2.4cm}>{\raggedright\arraybackslash}m{2.5cm}>{\raggedright\arraybackslash}m{2cm}>{\raggedright\arraybackslash}m{3.55cm}} 
			  		   \toprule
					   Sustrato Nomenclatura   & Observaciones  & Limpieza previa$^*$ & Función \\ \midrule
			       	   vidrio\index{vidrio} \hspace{2cm} Vi  &	portaobjetos \textit{BioTraza} & inmersión KOH 40\% & económico para pruebas preliminares de deposito \\ \midrule
			       	   silicio\hspace{2cm} Si  & Si[100] pulido dopado tipo n  \textit{Addison}& inmersión HF\index{acido@ácido!fluohídrico} 48\% & FTIR, SEM\index{SEM}, FIB\index{FIB}, EPA \\ \midrule
			       	   Au\index{oro} sobre silicio\hspace{2cm} Si$|$Au & depositado por pulverizacion catódica$^\dagger$  & ultrasonido\index{ultrasonido}en H$_2$O  & transporte, EQ\\ \midrule
			      	   microelectrodo\index{electrodo!microelectrodo}s \hspace{2cm} $\mu Elec$ & sensores, diseño transferido por fotolitografía\index{fotolitografía}$^\mathsection$  	  &  ultrasonido\index{ultrasonido}en H$_2$O  & multisensado, EQ \\ \midrule
			      	   polimericos         &  PMMA y PAI		  &  ultrasonido\index{ultrasonido}en H$_2$O &  demostrador métodos suaves de síntesis\\ 
			      	   \bottomrule
			    	   \end{tabular}\vspace*{2pt}
			    	   \footnotesize{$^*$Ver la sección <<\nameref{sec:limpieza}>>, tabla \ref{tabla:limpieza}, pág. \pageref{sec:limpieza}.}\\
			    	   \footnotesize{$^\dagger$Ver la sección <<\nameref{sec:sputt}>>, pág.\pageref{sec:sputt}.} \\
			    	   \footnotesize{$^\mathsection$Ver la sección <<\nameref{sec:fotolito}>>, pág. \pageref{sec:sputt}.}
			    	   \label{tabla:sustratos}
			   		   \end{table}
			
	\subsection{Eliminación del surfactante\index{surfactante}}\label{sec:cond_y_extr}

		Una vez realizado el depósito, se debe conservar la estructura del cristal líquido\index{cristal líquido} obtenido, y evitar el deterioro durante la eliminación\index{eliminación} del surfactante\index{surfactante}. Para ello se estabiliza la película durante \SI{1}{\hour} en cámara de humedad controlada a una HR constante de 50\%. Para mantener dicha humedad se utilizó una solución saturada de Ca(NO$_3$)$_2$.5H$_2$O (\textit{Biopack}). El ingreso de H$_2$O permite de esta forma aumentar el grado de polimerización\index{polimerización} del óxido y ayudar a la separación de fases entre el agente moldeante\index{agente moldeante} y el óxido\cite{Crepaldi2003}. El proceso de estabilización y condensación\index{condensación} del óxido continua con un calentamiento en plancha calefactora, (\textit{Cimarec}) una hora a \SI{60}{\celsius} y una hora más a \SI{130}{\celsius}\cite{Crepaldi2003,Crepaldi2002a}. 
				
		Posteriormente a la estabilización de la película se experimentaron varios tratamientos para completar el proceso de condensación\index{condensación} de la fase inorgánica y extraer el surfactante\index{surfactante} para dar lugar a la película nanoporosa, a saber:

				\begin{itemize}

				\item \textit{Calcinacion.} Este es el proceso clásico en el cual se somete a la película a una temperatura de \SI{350}{\celsius} durante \SI{2}{\hour} con una rampa de \SI{1}{\celsius.\minute^{-1}} (Horno \textit{Indef 337}). De esta forma se condensa el óxido, se elimina el surfactante\index{surfactante} y se minimiza el daño de la estructura tridimensional de la red nanoporosa\index{película!nanoporosa} \cite{Crepaldi2003}.

				\item \textit{Condensación ácida.} En este método se busca promover la condensación\index{condensación} de la matriz inorgánica mediante la exposición de las películas a una atmósfera de vapores de HCl\index{acido@ácido!clohídrico} \cite{Doshi2000a}. El arreglo para tal fin consiste en sujetar las muestras al fondo de un vaso precipitados y colocarlo invertido sobre un cristalizador con HCl\index{acido@ácido!clohídrico} concentrado (\textit{Biopack}) durante \SI{10}{\minute}. 

				\item \textit{Condensación alcalina.} Al\index{aluminio} igual que el método anterior, se busca promover la condensación\index{condensación} del óxido cambiando las condiciones del entorno químico, en este caso someter las películas a una atmósfera de pH\index{pH} extremadamente básico\index{básico} generada con vapores de NH\index{amoniaco}$_3$ (\textit{Biopack}) \cite{Soler-Illia2012,Soler-Illia2011}. El armado experimental fue igual que el descripto para el método ácido.

				\item \textit{Tratamiento a \SI{130}{\celsius}.} Esta estrategia de síntesis involucró dejar las muestras en estufa a \SI{130}{\celsius} durante 7 días con el objetivo de promover la condensación\index{condensación} del óxido.

				\item \textit{Alto vacío.} Este tratamiento consiste en dejar las muestras en una cámara de alto vacio\index{alto@alto vacío} a \SI{1e-5}{\milli\bar} y \SI{130}{\celsius} durante 7 días. Para calentar y llegar al vacío necesario se utilizó la cámara de una soldadura de obleas (\textit{EVG 501 Manual Wafer Bonding System}) la cual fue evacuada por una bomba mecánica y una turbomolecular secuencialmente.

				\end{itemize}
					
		En los casos donde fue necesario realizar la extracción\index{extracción} del surfactante\index{surfactante} sin calcinar, las muestras fueron sometidas a un reflujo de isopropanol\index{propanol@2-propanol} a punto de ebullición (\textit{Biopack}) durante \SI{15}{\minute}. Luego se enjuagaron con H$_2$O acidificada con HCl\index{acido@ácido!clohídrico} a $\text{pH}=2$. El siguiente diagrama de flujo resume y agrupa los tratamientos realizados sobre las \pdm, desde el depósito\index{depósito} hasta la extracción\index{extracción} del surfactante\index{surfactante}.
		
				\begin{figure}[ht!]
						  \begin{center}
						  \includegraphics[width=\textwidth]{Esquemas/Resumen_extraccion.pdf}
						  \caption[Tratamientos pos-depósito de \pdm]{Etapas de estabilización y diferentes tratamientos pos-depósito utilizados para hacer las \pdm, tanto de óxidos puros como las mixtas.}
						  \label{esq:peliculas_meso_tratamientos}
						  \end{center}
						  \end{figure}

	\subsection{Espectroscopia IR}\label{sec:IR}

		El segmento infrarrojo (IR) del  espectro electromagnético\index{espectro electromagnético} puede ser divido en tres zonas, según su longitud de onda\index{longitud de onda}\index{longitud de onda}, IR cercano (400 a \SI{10}{\cm^{-1}}), IR medio (4000 a \SI{400}{\cm^{-1}}), e IR lejano (14000 a \SI{4000}{\cm^{-1}}). El infrarrojo medio puede ser usado para estudiar las vibraciones\index{vibración} fundamentales y la estructura roto-vibracional; brinda información acerca de los grupos orgánicos e inorgánicos  presentes a través del análisis de las vibraciones\index{vibración} moleculares.\cite{Atkins2006,Barrow1962,Stuart2004} 
		
		A lo largo de este trabajo se uso esta porción del espectro IR para analizar los resultados de la extracción\index{extracción} de surfactante\index{surfactante} y estructura inorgánica de las \pdm. Las mediciones se llevaron a cabo en la Unidad Técnica de Nanomateriales del Centro de Investigaciones en Procesos Superficiales del INTI\index{INTI} (INTI-CIEPS). El equipo es un \textit{Thermo Scientific Nicolet 6700 FTIR} que cuenta con un microscopio para poder focalizar el haz en un área de aproximadamente \SI{0.5x0.5}{\mm}. Se utilizó la técnica de espectroscopia\index{espectroscopia} infrarroja por transformadas de Fourier (FTIR) tanto en trasmisión como en reflexión y los espectros fueron tomados con el detector MCT/B (\textit{Wide Band mercury cadmium telluride}) que es de 4 a 10 veces más sensible que los detectores estándar para equipos de espectroscopia\index{espectroscopia} FTIR.\cite{Nicholet2007} Las películas destinada a ser caracterizadas por FTIR\index{FTIR} fueron depositadas sobre Si, por ser este trasparente en el IR medio.

	\subsection{Ángulo de contacto}

		La medición del angulo de contacto surge como una descripción teórica para el equilibrio entre tres fases; la fase líquida de la gota, la fase gaseosa del aire y la sólida del sustrato. El valor del ángulo de contacto\index{angulo@ángulo de contacto} depende principalmente de la relación que existe entre las fuerzas adhesivas entre el líquido y el sólido y las fuerzas cohesivas del líquido. Se puede, así, cuantificar la mojabilidad\index{mojabilidad} de un líquido en aire, en una determinada superficie\index{superficie}.\cite{findenegg1997} Tomando dos caso extremos, cuando la superficie\index{superficie} interactúa fuertemente con el líquido y se moja, el angulo de contacto se aproxima a $0^{\circ}$, en cambio si la superficie\index{superficie} y el líquido se repelen, el angulo tenderá a $180^{\circ}$. En términos de equilibrio termodinámico, el potencial químico de las tres fases  debe ser igual. Quien dió la primera descripción en términos de energías interfasiales fue Young\index{Young} en 1805\cite{young1805}, donde postuló que la energía\index{energía} superficial líquido-vapor ($\gamma$) por el coseno del angulo de contacto($\theta$) es igual a diferencia de las energías superficiales sólido-líquido $\gamma_{_{SL}}$ y sólido-vapor  $\gamma_{_{SV}}$s. Tal relación se la conoce como ecuación de Young\index{Young} (ecuación \ref{eq:young}).

			\begin{equation}
				\gamma\, cos(\theta) = \gamma_{_{SL}} - \gamma_{_{SV}}
				\label{eq:young} 
				\end{equation}

		En este trabajo se utilizaron las medidas de angulo de contacto entre agua y las superficie\index{superficie}s de las \pdm, para calcular la distribución de los tamaños de poro\index{poro} y cuello\index{cuello de poro} de los sistemas porosos aplicando la ecuación de Kelvin\index{Kelvin}.\cite{Boissiere2005} En la próxima sección se explica en detalle como se estiman dichas distribuciones.
		Las medidas de ángulo de contacto\index{angulo@ángulo de contacto} se realizaron en la Gerencia Química, CAC-CNEA con un equipo \textit{Ramé-Hart 290} y los datos fueron recogido con el software \textit{DROPImage}.

	\subsection{Elipsometría}\label{sec:elipso}

		La elipsometría es una técnica de análisis óptico que se basa en el cambio del estado de polarización\index{polarización} de la luz que incide sobre una o más películas delgadas soportadas sobre un material reflectivo. Dicho análisis es no destructivo y es útil para la determinación de espesor\index{espesor}es y constantes ópticas (índices de refracción y constante de absorción) de dichas películas.\cite{TompkinsHarlandG.1999,Rothen1945} Las mediciones obtenidas nos devuelven los parámetros elipsométricos $\Delta(\lambda\index{longitud de onda})$ y $\Psi(\lambda\index{longitud de onda})$. 

		Mediante un modelo matemático (en el cual se proponen valores iniciales para las constantes ópticas y el espesor\index{espesor} de la muestra) se ajusta por cuadrados mínimos hasta minimizar, por sucesivas iteraciones, la diferencia con los datos experimentales. De esta forma se extrae del modelo el espesor\index{espesor} y el índice de refracción\index{indice@índice de refracción} de la película. Cuando se adapta una cámara al equipo, donde se puede variar la presión parcial de H$_2$O, es posible medir los cambio de las propiedades ópticas de las \pdm durante la adsorción\index{adsorción} y desorción\index{desorción} de H$_2$O. A esta técnica se la conoce con el nombre de porosimetría elipsométrica ambiental (PEA) \cite{Boissiere2005}. La figura \ref{fig:elipso} esquematiza el funcionamiento del elipsómetro y la figura \ref{fig:elipsofoto}, pág \pageref{fig:elipsofoto}, es una fotografía del equipo utilizado un \textit{SOPRA} modelo \textit{GES 5E}. 

			  \begin{figure}[t]
				\begin{center}
				\includegraphics[width=\textwidth]{Esquemas/Elipso.pdf}
			  	\caption[Esquema de la técncia de elipsoporosimetría\index{elipsoporosimetría ambiental} ambiental]{Esquema de los componentes principales del equipos de elipsometría utilizado para determinar las constantes elipsométricas, $\Delta(\lambda\index{longitud de onda})$ y $\Psi(\lambda\index{longitud de onda})$, de las cuales se obtienen el espesor\index{espesor}, indice de refracción, coeficiente de absorción\index{coeficiente de absorción},  distribución y tamaño de poro\index{distribución!de poro}s y cuello\index{cuello de poro}s de las \pdm.}
			  	\label{fig:elipso}
			  	\end{center}
			  	\end{figure}
		
		El volumen de vapor adsorbido dentro de los poros se determina a partir de dicha variación utilizando aproximaciones de medio efectivo como la de Bruggeman\index{Bruggeman}\cite{Bruggeman1935} o la de Maxwell-Garnett\index{Maxwell-Garnett}\cite{Garnett1906} que son simplificaciones de la ecuación general de Lorentz-Lorentz\index{Lorentz-Lorentz} \cite{TompkinsHarlandG.1999}.
		La aproximacion de Bruggeman\index{Bruggeman} considera dos componentes mezclados al azar cuyas fracciones en volumen ($f_i$) y constante dielectrica ($\mathcal{E}_i$) deben cumplir con la ecuación \ref{eq:bruggeman} donde $\mathcal{E}_e$ es la constante dieléctrica\index{constante dieléctrica} del material compuesto, la cual se determina experimentalmente.
							\begin{equation}
					 		   	 f_1\left(\frac{\mathcal{E}_1-\mathcal{E}_e}{\mathcal{E}_1+2\mathcal{E}_e}\right)+
					 		   	 f_2\left(\frac{\mathcal{E}_2-\mathcal{E}_e}{\mathcal{E}_2+2\mathcal{E}_e}\right)=0
					 		     \label{eq:bruggeman}
								\end{equation}
		La aproximación de Maxwell-Garnett\index{Maxwell-Garnett} considera al material compuesto por al menos dos especies, la matriz y la inclusión. En el caso de los óxidos porosos, la matriz es el óxido y el aire el surfactante\index{surfactante} la inclusión. Se deben satisfacer en este caso las ecuaciones \ref{eq:maxwall1} y \ref{eq:maxwall2}.
							\begin{equation}
					 		   	 f_1\left(\frac{\mathcal{E}_1-\mathcal{E}_2}{\mathcal{E}_1+1\mathcal{E}_2}\right)-
					 		   	 \left(\frac{\mathcal{E}_e-\mathcal{E}_2}{\mathcal{E}_e+2\mathcal{E}_2}\right)=0
					 		     \label{eq:maxwall1}
								\end{equation}
								\begin{equation}
					 		   	 f_2\left(\frac{\mathcal{E}_2-\mathcal{E}_1}{\mathcal{E}_1+2\mathcal{E}_1}\right)-
					 		   	 \left(\frac{\mathcal{E}_e-\mathcal{E}_1}{\mathcal{E}_e+2\mathcal{E}_1}\right)=0
					 		     \label{eq:maxwall2}
								\end{equation}
		El volumen total ocupado por los poros, V$_p$, y el volumen de agua adsorbido para cada HR, V$_{ads}$, se calcularon aplicando indistintamente dichas aproximaciones (ya que para \pdm\space dan resultados equivalentes) a las constantes dieléctricas\index{constante dieléctrica} medidas del film seco y lleno de agua, luego de la condensación\index{condensación} capilar.\cite{Angelome2008,Fuertes2009,Nano-compuestas2013}. Se construye de esta forma una isoterma\index{isoterma} de adsorción\index{adsorción}/desorción de H$_2$O en función del índice de refracción\index{indice@índice de refracción} de la o las películas porosas.
		El tamaño de los poros y los cuello\index{cuello de poro}s, que forman la red porosa tridimensional, se puede calcular a partir de la rama de adsorción\index{adsorción} y de la de desorción\index{desorción} de la isoterma\index{isoterma} respectivamente. Para ello debemos recurrir a la ecuación de Kelvin\index{Kelvin} (ec. \ref{eq:kelvin}), que describe el equilibrio líquido-vapor considerando tamaño de la esfera y energía\index{energía} superficial. Donde R es la constante de los gases, T es la temperatura, P es la presión de vapor\index{presión de vapor}, P$_s$ es la presión de vapor\index{presión de vapor} de saturación, $\gamma$ es la tensión superficial\index{tensión superficial} del líquido, V$_m$ es el volumen molar del líquido y $\theta$ es el ángulo de contacto\index{angulo@ángulo de contacto} sólido-líquido. \cite{Baklanov2000,Boissiere2005,Sing1985} Para poros esféricos la relación $\partial S/ \partial dV$ es proporcional al radio de la esfera, llamado radio de Kelvin\index{Kelvin}.\cite{FernandezPrini2005}
			\begin{equation}
			  	 \ln \left(\frac{P}{P_s}\right)=\frac{2\gamma V_m}{RT} \cos{\theta}\frac{\partial S}{\partial V}
			     \label{eq:kelvin}
			 	 \end{equation}					
		Todas las medidas fueron tomadas en la Gerencia Química, CAC-CNEA con un elipsómetro espectroscópico marca \textit{SOPRA}, modelo \textit{GES 5E}. El rango espectral del equipo va de 190 a \SI{900}{\nm}, pose una cámara para realizar las mediciones en condiciones de humedad controladas y también permite configuración en modo \textit{micro-spot} que permite reducir el área de medición a una región de aproximadamente \SI{1}{\mm^2}. El modelado de los parámetros se hizo mediante el \textit{software Winelli II} también de la marca \textit{SOPRA}.
					\begin{figure}[ht]
							  \begin{center}
							  \includegraphics[width=\textwidth]{Imagenes/elipsometro.jpg}
							  \caption[Elipsómetro]{Foto del equipos elipsómetro espectroscópico marca \textit{SOPRA}, modelo \textit{GES 5E} ubicado en la Gerencia Química, CAC-CNEA utilizado para la caracterización de las \pdm.}
							  \label{fig:elipsofoto}
							  \end{center}
							  \end{figure}

\section{Microfabricación de los electrodos}
		
	 En esta sección se dará cuenta de los detalles experimentales para la fabricación de los electrodos, los cuales son una parte fundamental de los sensores. Es en la superficie\index{superficie} de los electrodos donde se llevan a cabo las reacciones de óxido-reducción de los analitos de interés y donde se depositan la película delgada \index{película!delgada}mesoporosa. Por estos motivos resulta fundamental contar con un diseño funcional y compacto y, además, controlar los aspectos superficiales tales como la rugosidad\index{rugosidad}, control de impurezas, espesor\index{espesor}, y funcionalización en los caso que sea necesario.

	 Los electrodos fueron enteramente diseñados y fabricados en los laboratorios del CMNB-INTI. 
		
	 Las herramientas y técnicas empleadas para la fabricación son propias del sector de la microelectrónica\index{microelectrónica}; herramientas de \textit{software}\index{software@\textit{software}} tipo CAD, fotolitografía\index{fotolitografía}, pulverización catódica\index{pulverización catódica}, grabado por vía húmeda, \textit{lift-off}\index{lift@\textit{lift-off}}, corte y encapsulado, etc.\cite{Franssila2004,Jaeger2001} Cada uno de estos procesos y metodologías se explicarán en las secciones subsiguientes. 

	 El flujo general de trabajo para la transferencias de diseños en una o más capas se presenta en la figura \ref{esq:micro}.

			\begin{figure}[ht]
			  \begin{center}
			  \includegraphics[width=\textwidth]{Esquemas/Resumen_micro.pdf}
			  \caption[Esquema para la transferencia de los diseños\index{transferencia!de los diseños}\index{transferencia!de los diseños}]{Diagrama general para la transferencias y fabricaciones de diseños de una o más capas. Este esquema contempla el uso de las técnicas de \textit{lift-off }o grabado según se requiera dependiendo de las características de los materiales empleados para esa capa.}
			  \label{esq:micro}
			  \end{center}
			  \end{figure}
			  
	\subsection{Diseño e impresión de las máscara\index{máscara}s}\label{sec:impresion_mascaras}

		El primer paso necesario en la fabricación de los sensores\index{sensor} es el diseño. Como todo diseño en microelectrónica\index{microelectrónica}, se pensó en función de las tecnologías disponibles, de la calidad de las máscara\index{máscara}s y de la aplicación final en la cual se emplearán. Todos estos aspectos ya fueron expuestos en el capitulo \ref{chap:Introduccion}, por lo que aquí nos remitiremos a describir los detalles técnicos.

		Los diseño fueron pensaron para obleas de \SI{10}{\cm} de diámetro. El primer diseño se mandó a imprimir en filmina de \SI{13x13}{\cm} en una filmadora de películas \textit{Agfa Accuset 1000}, a una resolución de \SI{3600}{dpi}, perteneciente a la firma $Imacrom$. Esto ha permitido obtener resoluciones de linea de \SI{50}{\um}, muy por encima de la resolución de la tecnología de la cual disponemos (transferencia por UV, $\lambda\index{longitud de onda}=365nm$). El segundo diseño, mas completo e integrado, también fue diagramado para obleas \SI{10}{\cm} de diámetro. Este contempló la integración del contraelectrodo\index{electrodo!contraelectrodo} y el electrodo de referencia\index{electrodo!de referencia}, además de incluir 6 electrodos de trabajo\index{electrodo!de trabajo}. Las máscara\index{máscara}s correspondientes a este diseños se mandaron a imprimir en filminas de \SI{13x13}{\cm} a la empresa \textit{International Phototool Company} a una resolución de \SI{48000}{dpi}, logrando mejor resolución y lineas más definidas que en el primer diseño, hasta de \SI{7}{\um}. Todos los diseños se llevaron a cabo con el \textit{software CAD electric}\index{software@\textit{software}!\textit{electric}} (\url{http://www.staticfreesoft.com/productsFree.html}) de licencia pública general de GNU, \url{https://www.gnu.org/licenses/gpl.html}. 
				
	\subsection{Limpieza de los sustratos}\label{sec:limpieza}
			
			Una vez terminado el diseño, comienza la etapa de transferencia del mismo. El primer paso es la limpieza de los sustratos para evitar problemas de falta de adhesión y eliminar impurezas superficiales adsorbidas. 

			\begin{table}[!ht]
					  \caption[Soluciones para la limpieza de los sustratos]{Soluciones utilizadas para hacer la limpieza antes de realizar cualquier proceso de fotolitografía\index{fotolitografía} o pulverización catódica\index{pulverización catódica}.\cite{Franssila2004,Kern1990}}
			  		  \begin{tabular}{>{\raggedright\arraybackslash}m{1.02cm}>{\centering\arraybackslash}m{2.8cm}>{\centering\arraybackslash}m{1.9cm}>{\centering\arraybackslash}m{1.9cm}>{\raggedright\arraybackslash}m{2.4cm}} 
			  		  \toprule
					  Nombre  & Composición &  Proporciones & Condiciones & Blanco \\ \midrule
			      	  KOH$^*$ & KOH:H$_2$O 	&    40\%p/v    &  \SI{25}{\celsius}/\SI{10}{\minute}  &  residuos orgánicos \\  \midrule
			      	  SC1$^\dagger$ &	H$_2$O:H$_2$O$_2$:NH\index{amoniaco}$_4$OH & 5:1:1 & \SI{80}{\celsius}/\SI{10}{\minute} & residuos orgánicos  \\ \midrule
			      	  SC2 &	H$_2$O:H$_2$O$_2$:HCl\index{acido@ácido!clohídrico} & 6:1:1 & \SI{80}{\celsius}/\SI{10}{\minute}   &  residuos iónico\index{iónico}s y metálicos \\ \midrule
			      	  HF\index{acido@ácido!fluohídrico}  &	H$_2$O:HF & 50:1 & \SI{25}{\celsius}/\SI{2}{\minute} & óxido de silicio\index{silicio!oxido de}\index{silicio} \\ \midrule
			      	  iPOH    &	  (CH$_3)_2$CHOH &  puro$^\mathsection$      &  enjuague & residuos grasos \\ \midrule
			      	  H$_2$O & H$_2$O desionizada & puro$^\ddagger$  &  enjuague  & desorción\index{desorción} de partículas \\ \midrule
			      	  Piraña &  H$_2$SO$_4$:H$_2$O$_2$ & 2:1 & \SI{25}{\celsius}/\SI{10}{\minute}  & residuos orgánicos  \\
			      	  \bottomrule
			    	  \end{tabular}
			    	  \footnotesize{$^*$}No apta para silicio, racciona con el mismo para formar Si(OH)$_4$ e H$_2$. \\
				      \footnotesize{$^\dagger$}Crece una capa de SiO$_2$ de 10 a \SI{15}{\angstrom} de espesor\index{espesor}. \\
				      \footnotesize{$^\mathsection$}Grado analítico o superior. \\
			    	  \footnotesize{$^\ddagger$}Resistividad de o \SI{18}{\mega\ohm\per\cm} o mayor.
			    	  \label{tabla:limpieza}
			   		  \end{table}
			
							
			La tabla \ref{tabla:limpieza} resume cuales fueron las soluciones utilizadas para limpieza, su composición y cuál es la finalidad de cada una. Al\index{aluminio} finalizar cada etapa de limpieza siempre se hace un lavado con H$_2$O DI seguido de un secado con aire o N$_2$. El porqué de los materiales elegidos para usar de sustratos ya fueron discutidos en el capitulo \ref{chap:Introduccion}, aquí solo se mencionan los protocolos de limpieza\cite{Franssila2004,Kern1990} utilizados para cada uno de ellos:

				\begin{itemize}
					\item{Vidrio: KOH}
					\item{Silicio: SC1, SC2, HF\index{acido@ácido!fluohídrico} o piraña\index{piraña} según el caso}
					\item{Sustratos poliméricos: ipOH}
				\end{itemize}

    \subsection{Transferencia de los diseños por fotolitografía\index{fotolitografía}}\label{sec:fotolito}

		La transferencia de los diseños\index{transferencia!de los diseños} se realizó por fotolitografía\index{fotolitografía}, técnica que también se conoce con los nombres de litografía\index{litografía} óptica o litografía\index{litografía} ultravioleta\index{UV} (UV). La técnica consiste en depositar una resina fotosensible sobre un sustrato, irradiar con luz UV\index{UV} de $\lambda\index{longitud de onda}\!=$\SI{365}{nm} a través de una máscara\index{máscara} y por último revelar la fotorresina\index{fotorresina}. Dependiendo si esta es negativa, positiva o de doble exposición, se disolverá la parte expuesta (positiva) o la no expuesta a la luz (negativa). \cite{Jaeger2001,Franssila2004,Mack2007,Mack2006}
	
		Antes de depositar la fotorresina\index{fotorresina} se calienta el sustrato\index{sustrato} hasta \SI{120}{\celsius} con el objetivo de desorber H$_2$O. El depósito\index{depósito} se realizó con el equipo descrito en la sección <<\nameref{sec:deposito_pdm}>>, pág. \pageref{sec:deposito_pdm}. Para cubrir una oblea\index{oblea} completa de \SI{10}{\cm} de diámetro se necesitan colocar un mínimo de \SI{5}{\ml} de fotorresina\index{fotorresina} \textit{TI35E image reversal} de la marca \textit{Microchemicals}, la cual es de doble exposición, especialmente elegida por formar un perfil negativo, particularmente útil para el proceso \textit{lift-off}\index{lift@\textit{lift-off}}, explicado mas adelante.\cite{MicrochemicalsTeam2009} 
			  \begin{figure}[ht]
			  \begin{center}
			  \includegraphics[width=0.60\textwidth]{Esquemas/fotolito.pdf}
			  \caption[Esquema fotolitografía\index{fotolitografía}]{Proceso de fotolitografía\index{fotolitografía} para una resina de doble exposición\index{resina!de doble exposición}. 1) Deposito de la resina, 2) Calentamiento suave, mejora la adherencia\index{adherencia} y evapora solventes, 3) 1$a$ exposición, 4) Calentamiento para invertir la polaridad de la resina, 5) 2$a$ exposición sin máscara\index{máscara}, 6) Revelado, notese el perfil invertido, especialmente útil para aplicar en procesos de\textit{ lift-off}.}
			  \label{esq:fotolito}
			  \end{center}
			  \end{figure}			  
		El deposito se hizo por \textit{spin-coating}\index{spin@\textit{spin-coating}}, a una velocidad final de \SI{4000}{\minute^{-1}} durante \SI{40}{\second}, con una aceleración de \SI{400}{\minute^{-1}.\second^{-1}} para obtener un espesor\index{espesor} final de \SI{4}{\um}. Luego se realiza un calentamiento durante \SI{2}{\minute} a \SI{95}{\celsius} para evaporar el exceso de solvente y promover la adhesión de la resina al sustrato. Seguidamente se cargan el sustrato\index{sustrato} y la máscara\index{máscara} en la alineadora\index{alineadora} de máscara\index{máscara}s (\textit{EVG 620}, figura \ref{fig:alineadora}), la cual cuenta con un microscopio incorporado para hacer la alineación máscara\index{máscara}/sustrato y una lámpara de Hg para el sistema de irradiación UV. 

		Después de alinear, se realiza la primera exposición con un densidad de energía\index{energía!densidad de}\index{energía} de \SI{140}{mJ.\cm^{-2}} y se deja reposar \SI{10}{\minute} para dar tiempo a la difusión\index{difusión} de N$_2$ liberado durante la reacción. Se realiza ahora el calentamiento para invertir el perfil (las zonas expuestas polimerizan volviéndose inerte al solvente) de la resina a una temperatura de \SI{120}{\celsius} durante \SI{2}{\minute}  y se expone por segunda vez a una densidad de energía\index{energía!densidad de}\index{energía} de \SI{540}{mJ.cm^{-2}},
		
		esta vez sin máscara\index{máscara}. En esta segunda exposición las partes polimerizadas no se afectan, mientras las no expuestas en la primera iluminación se vuelven solubles en el medio revelador. Para finalizar, se hace el revelado\index{revelado} surmegiendo la oblea\index{oblea} en un cristalizador con una solución de revelador\index{revelador} \textit{AZ General} (\textit{Microchemicals}) y H$_2$O 1:1. La evolución del revelado\index{revelado} se siguió mediante microscopia\index{microscopía}\index{microscopía!óptica} y se determinó en aproximadamente unos \SI{7}{\minute}, dependiendo del espesor\index{espesor} de la fotoresina. De esta forma quedan transferidos los diseños. El flujo de trabajo se sintetiza en el esquema \ref{esq:fotolito}.

			\begin{figure}[ht]
			  \begin{center}
			  \includegraphics[width=\textwidth]{Imagenes/alineadora.jpg}
			  \caption[Alineadora de máscara\index{máscara}s]{Alineadora de máscara\index{máscara}s \textit{EVG 620} semiautomática de doble cara, con lámpara de Hg de \SI{350}{W}  y capacidad para obleas de hasta \SI{150}{\mm} .}
			  \label{fig:alineadora}
			  \end{center}
			  \end{figure}	

	\subsection{Depósito películas delgadas metálicas}\label{sec:sputt}

			En esta sección se describe el modo en que fueron fabricadas las películas delgadas de Au\index{oro} cuya función es ser usadas como electrodos en los sensores. 
				  \begin{figure}[t!]
				  \begin{center}
				  \includegraphics[width=\textwidth]{Imagenes/sputt.jpg}
				  \caption[Equipo para depósito\index{depósito} de películas delgadas, \textit{sputtering}\index{sputtering@\textit{sputtering}}]{Foto del instrumental utilizado para realizar los depósitos bicapa Ti\index{titanio}\textbar Au\index{oro} o Cr\index{cromo}\textbar Au. A)El equipo \textit{BOC Edwards} completo donde se ve el gabinete de control y la cámara de vacío, B)Foto a través de la ventana al momento de realizar un deposito de Cr\index{cromo} y C)Foto a través de la ventana al momento de realizar un deposito de Au.}
				  \label{fig:sputt}
				  \end{center}
				  \end{figure}	
			
			Para su fabricación se utilizó la técnica de pulverización catódica\index{pulverización catódica}, la cual es comúnmente conocida por su nombre en ingles, \textit{sputtering}\index{sputtering@\textit{sputtering}}\cite{sigmund1968}. Los fundamentos básicos de la técnica se discutieron en el capitulo \ref{chap:Introduccion}, pág. \pageref{sec:microfabricacion}.

			Se utilizaron como sustratos de los electrodos principalmente obleas de silicio\index{silicio} monocristalinas (vírgenes o fotolitografiadas) y portaobjetos de vidrio.  Estos soporte fueron escogidos debido la baja rugosidad\index{rugosidad} de su superficie\index{superficie} y por ser materiales que pueden ser sometidos a temperaturas altas, en particular \SI{350}{\celsius}, que es la temperatura de calcinación\index{calcinación} para la ruta de síntesis clásica de óxidos mesoporosos. Previo a realizar el depósito, los sustratos fueron tratados con los procesos de limpieza descritos en la sección \ref{sec:limpieza}, pág. \pageref{sec:limpieza} y una vez dentro de la cámara se realizó una limpieza por plasma para promover una mayor adherencia\index{adherencia} del depósito\index{depósito} al sustrato.

			Cabe destacar que si se trabaja sobre obleas de silicio, estas tienen que estar recubiertas con una capa dieléctrica para que no haya fugas eléctricas a través del silicio. A lo largo de este tesis se utilizó indistintamente obleas que ya venían con capa aislante u obleas a las cuales se le depositó una película delgada \index{película!delgada}de SiO$_2$, también por pulverización catódica\index{pulverización catódica}.

			Para promover la adherencia\index{adherencia} del Au, se deposita una capa de al menos \SI{20}{\nm} de espesor\index{espesor}, la misma puede ser indistintamente de Ti\index{titanio} o Cr\index{cromo}. Sin esta capa el Au\index{oro} no adhiere sobre superficie\index{superficie}s no metálicas\cite{Hieber1976}. Una vez depositada esta capa de adherente y sin romper el vacío de la cámara del equipo, se depositan un mínimo \SI{150}{\nm} de Au, para lograr un electrodo mecánicamente robusto y con buenas propiedades de conducción eléctrica. En los casos que se depositó una capa dieléctrica de SiO$_2$ se utilizó la fuente de radiofrecuencia\index{radiofrecuencia} (RF) a potencia constante, P=\SI{400}{W}. Mientras que los depósitos de las películas metálicas se realizaron todos con la fuente de corriente directa (DC) también configurada a P=\SI{400}{W}, dejando la tensión y la corriente libre, parámetros que dependen a su vez del vacío en la cámara, de la distancia entre el cátodo\index{cátodo} y el ánodo\index{anodo @ ánodo} y el caudal de argón\index{argón}. Para cada caso, en condiciones constantes, se puede realizar una curva de calibración. La misma se construye graficando el espesor\index{espesor} de las películas depositadas en función del tiempo de depósito, con el objetivo de establecer la tasa de depósito\index{depósito} para cada material. 

			Las condiciones de depósito\index{depósito} de cada una de las sucesivas capas se detallan en la tabla \ref{tabla:sputt1} para las películas metálicas y en la tabla  \ref{tabla:sputt2} para el SiO$_2$ y el plasma previo al deposito.

			%Tabla con los parámetros de deposito de la películas
		  		\begin{table}[ht]
		  		\caption[Parámetros de depósito\index{depósito} películas metálicas]{Parámetros de depósito\index{depósito} de las distintas películas delgadas metálicas para su uso como electrodos de trabajo\index{electrodo!de trabajo}.}
		  		\begin{tabular}{lcccccc} 
		  		\toprule
		    	 Depósito&$P_{_{\text{DC}}}$(W) & $T$(V)  &  $I$(A)   & $p$(mbar) & $Q_{Ar}$(sccm)   & $T$(nm/min) \\
		    	 		\midrule
		  		 Ti\index{titanio} 	 & $400$ & $750$ & $0.53$ & \num{1.70e-3} & $5$ & $50$ \\
		  		 Cr\index{cromo} 	 & $400$ & $453$ & $0.84$ & \num{1.70e-3} & $5$ & $55$ \\
		  		 Au\index{oro} 	 & $400$ & $679$ & $0.56$ & \num{1.35e-3} & $5$ & $44$ \\
		    	 \bottomrule
		    	 \end{tabular}
		   		\label{tabla:sputt1}
		   		\end{table}
		   		
		  		\begin{table}[ht]
		  		\caption[Parámetros de depósito\index{depósito} películas dieléctricas]{Parámetros de depósito\index{depósito} utilizado para el depósito\index{depósito} de $SiO_2$.}
		  		\begin{tabular}{lccccc} 
		  		 		\toprule
		       	Depósito&$P_{_{\text{RF}}}$(W)  &$P_{ref}$(W)  &$p$(mbar) & $Q_{Ar}$(sccm) &$T$(nm/min)\\
		    	 		\midrule
		  		 $SiO_2$  & $400$ & $23$ & \num{1.23e-2} & $80$ & $1.18$ \\
		  		 Limpieza & 150   & 3    & \num{2.04e-3} & 10   & -      \\
		  		\bottomrule
		  		\end{tabular}
		   		\label{tabla:sputt2}
		   		\end{table}
		   	
		   	Todos los depósitos fueron realizados en el equipo de \textit{sputtering}\index{sputtering@\textit{sputtering}} del INTI\index{INTI}-CMNB. El mismo cuenta, entre sus principales capacidades, con una fuente DC (hasta \SI{1.5}{\kW}) una fuente de RF (\SI{600}{W} a \SI{13.56}{\MHz}), posibilidad de depositar 3 materiales consecutivamente y capacidad para colocar sustratos de hasta \SI{250}{\nm}. El mismo es de la marca \textit{Boc Edwards}, en la figura \ref{fig:sputt} se muestra el equipo y un detalle al momento de hacer los depósitos.

	\subsection{Proceso de\textit{ lift-off}}
			Una vez finalizados los procesos de fotolitografía\index{fotolitografía} y pulverización catódica\index{pulverización catódica} del metal o de los metales necesarios queda sólo remover la resina.

					\begin{figure}[!ht]
							  \begin{center}
							  \includegraphics[width=\textwidth]{Esquemas/liftoff.pdf}
							  \caption[Esquema del proceso de\textit{ lift-off}]{Esquema del proceso de\textit{ lift-off} en el cual se solubiliza la fotorresina\index{fotorresina} con película metálica encima. 1) Fotorresina transferida en base a un diseño arbitrario, 2) depósito\index{depósito} metálico, 3) disolución de la fotorresina\index{fotorresina} con un solvente adecuado, 4) Diseño completamente trasferido.}
							  \label{esq:liftoff}\index{lift@\textit{lift-off}}
							  \end{center}
							  \end{figure}

		 La bicapa Ti\index{titanio}\textbar Au\index{oro} o Cr\index{cromo}\textbar Au\index{oro} se pulverizó sobre toda la superficie\index{superficie} de la oblea, tanto en las partes donde estaba el silicio\index{silicio} descubierto como en las partes donde quedó la fotorresina\index{fotorresina} sin revelar. 
			
		 De esta forma, al estar el metal sobre la resina, disolviendo ésta, se desvincula la capa Ti\index{titanio}\textbar Au\index{oro} del sustrato\index{sustrato} y queda completa la transferencia de los diseños\index{transferencia!de los diseños}. 
		 La disolución de la fotoresina se lleva a cabo en acetona\index{acetona} (\textit{Sigma}) dentro de un baño de ultrasonido\index{ultrasonido}(\textit{TESTLAB} Modelo \textit{tb02}) a \SI{22}{\kHz}. En la figura \ref{esq:liftoff} se esquematiza todo el proceso completo.

	\subsection{Modificación superficial}\label{sec:silanizacion}
		
		A lo largo del trabajo surgió la necesidad de mejorar la adherencia\index{adherencia} de las \pdm\space sobre los electrodos de Au\index{electrodo!de Au}. Para lograr ésto, se realizó sobre los electrodos una modificación superficial, de forma de generar puntos de anclaje para promover la adherencia\index{adherencia} del óxido de silicio\index{silicio!oxido de}\index{silicio} sobre la superficie\index{superficie} de los electrodos.
		El proceso consistió en vincular covalentemente una molécula\index{moléculas} a la superficie\index{superficie} de Au\index{oro} y, por otro lado, que ésta misma molécula\index{moléculas} sea parte estructural de las \pdm. Para lograr ésto se preparó una solución \SI{10}{\milli\Molar} de 3-mercaptopropil trimetoxisilano\index{mercaptopropil@3-mercaptopropil trimetoxisilano} (MPTMS) en tolueno (se eligió tolueno de forma de minimizar la hidrólisis\index{hidrolisis@hidrólisis} y condensación\index{condensación} del MPTMS) y se dejo reaccionar durante 2 horas en cristalizador. \cite{Goss1991,Herzog2013} Luego se realiza un enjuague con acetona\index{acetona} y se seca en flujo de N$_2$.

	\subsection{Encapsulado y corte}\label{sec:corte}

		Sobre los electrodos depositados se deposita una resina negativa\index{resina!negativa}, epoxi\index{epoxi} y fotocurable, \textit{SU8-100} de \textit{MicroChemical}\cite{MicrochemicalsTeam2009}. Dicha resina es ópticamente transparente y de alta viscosidad, lo que permite generar capas de hasta \SI{100}{\um} de espesor\index{espesor}. 

		Fue utilizada con un doble propósito, proteger mecánicamente los sensores\index{sensor} y hacer un reservorio o celda con un volumen  $V \approx$ \SI{2}{\ul}, el cual contendrá la solución con los analitos que se desean detectar.  
			%Graficos de rampa de veloci\index{velocidad!rampa de}dades		  
			\begin{figure}[ht]
			 		  \begin{center}
			 		  \includegraphics[width=0.70\textwidth]{Graficos/rotacion_su8.pdf}
			 		  \caption[Parámetros de depósito\index{depósito} para la resina expoxi\index{resina!expoxi}\index{epoxi}]{Esquema de aceleración y velocidad de rotación\index{velocidad!de rotación} para el depósito\index{depósito} de la fotorresina\index{fotorresina} epoxi\index{epoxi} SU8.}
			 		  \label{fig:spin-su8}
			 		  \end{center}
			 		  \end{figure}
	
		Para controlar el espesor\index{espesor} mediante \textit{spin-coating}\index{spin@\textit{spin-coating}} se utilizó el esquema de rotación que de la figura \ref{fig:spin-su8}. Luego se realizó un secado para evaporar solventes a \SI{65}{\celsius} durante \SI{1}{\minute} y \SI{95}{\celsius} durante \SI{10}{\minute}. Seguidamente, se expone al UV\index{UV} a través de la máscara\index{máscara} con una densidad de energía\index{energía!densidad de}\index{energía} de \SI{680}{mJ.cm^{-2}}, para activar los iniciadores de la polimerización\index{polimerización} sólo en las zonas iluminadas. Se realiza un segundo calentamiento gradual de \SI{1}{\minute} a \SI{65}{\celsius} y \SI{12}{\minute} a \SI{95}{\celsius} para incrementar el grado de polimerización\index{polimerización} y finalmente se lleva a cabo el revelado\index{revelado} (revelador para resina SU-8 de \textit{MicroChemical}), el cual requiere un tiempo de \SI{10}{\minute} para disolver completamente las partes que no fueron expuestas a la luz UV. 
		
		Para concluir la fabricación de los sensores\index{sensor} se corta la oblea\index{oblea} en cuadrados de \SI{1x1}{\cm} con el próposito de obtiener así cada dispositivo individual con 6 electrodos de trabajo\index{electrodo!de trabajo} cada uno. El corte se realiza con un disco de carburo de silicio\index{silicio} de  de la marca \textit{Loadpoint} girando a \SI{44000}{\minute^{-1}} y con una velocidad de avance de \SI{1}{\mm}. El mismo fue montado en una cortadora de obleas marca \textit{Laser Optics} ubicada en los laboratorios del INTI\index{INTI}-CMNB (ver figurea \ref{fig:dicer}).
			\begin{figure}[ht]
			 		  \begin{center}
			 		  \includegraphics[width=\textwidth]{Imagenes/dicer.jpg}
			 		  \caption[Cortadora de obleas]{Cortadora de obleas de la marca \textit{Laser Optics}.}
			 		  \label{fig:dicer}
			 		  \end{center}
			 		  \end{figure}

	\subsection{Espectroscopia de fotoelectrones de rayos X\index{rayos Xssh}}

		La técnica de XPS\index{XPS} (del ingles, \textit {X-ray photoelectron spectroscopy}) es una espectroscopia\index{espectroscopia} semi-cuantitativa y de baja resolución espacial que habitualmente se utiliza para estimar la estequiometría, estado químico de oxidación de algún elemento en particular y la estructura electrón\index{electrón}ica de los elementos en superficie\index{superficie}.\cite{siegbahn1956,siegbahn1981}

		Se hizo uso de esta técnica para evaluar estados de oxidación del Au\index{oro} y comprobar difusión\index{difusión} de contaminantes hacia la superficie\index{superficie} de los electrodos.  Los equipos constan de diferentes componentes; una cámara de ultra alto vacio\index{alto@alto vacío} (UHV) con presiones del orden de \SI{1e-9}{mbar} para disminuir la cantidad de contaminantes superficiales y asegurar a los electrones eyectados un camino libre medio lo suficientemente grande como para alcanzar el analizador. La cámara está construida en acero inoxidable y posee ventanas de vidrio\index{vidrio} para poder observar su interior. A ella se acoplan diferentes elementos necesarios para el análisis superficial como la fuente de rayos X\index{rayos Xssh}, el analizador de electrones, el cañón de iones, entre otros.\cite{XPS1978,Corthey2012}

		Las medidas de XPS\index{XPS} realizadas se llevaron a cabo en el Instituto de Investigaciones Fisicoquímicas Teóricas y Aplicadas (INIFTA\index{INIFTA}). Se utilizó una fuente de Mg K$\alpha$ (\textit{XR50, Specs GmbH}) y un analizador hemisférico (\textit{PHOIBOS 100, Specs GmbH}). La presión dentro de la cámara de UHV fue menor a \SI{1e-9}{mbar} mbar. El ángulo entre la fuente de rayos X\index{rayos Xssh} y el eje del analizador está fijado en \ang{54;44;00}. Los valores de sección eficaz de fotoionización están tabulados para esta geometría. Se realizó una calibración de la escala de energía\index{energía} de dos puntos utilizando Au\index{oro} evaporado ($E_B$ de Au$f_{7/2}$ = \SI{84}{\electronvolt}) y Cu ($E_B$ de Cu $2_{p3/2}$ = \SI{932.67}{\electronvolt}).
		
\section{Microscopías}
		
	 En este apartado haremos un breve resumen de los tipos de microscopia\index{microscopía} utilizadas durante la tesis.

	\subsection{Microscopía óptica}

		Se utilizó microscopia\index{microscopía}\index{microscopía}\index{microscopía!óptica} en modo reflexión fundamentalmente para evaluar la superficie\index{superficie} (homogeneidad, fracturas, grietas, etc), tanto de las películas metálicas como de las mesoporosas. También para determinar la calidad de las máscara\index{máscara}s impresas y para establecer los tiempos de revelado\index{revelado} en los procesos fotolitográficos. Se utilizó un microscopio \textit{Olympus} modelo \textit{BX51} configurado tanto para trasmisión como para reflexión. Como fuente de luz el equipo cuenta con lámpara halógena y, en los casos que hizo falta, se intercaló un filtro ultravioleta\index{UV} de forma de no exponer las fotorresina\index{fotorresina}s durante la inspección y evaluación de los tiempos de revelado.
	
	\subsection{Microscopía electrón\index{electrón}ica de barrido (MEB)}\label{sec:SEM}

		La microscopia\index{microscopía}\index{microscopía} electrón\index{electrón}ica de barrido (MEB) nos permitió ver y caracterizar las películas delgadas, ya sean los electrodos o las \pdm. Tamaño de poro, homogeneidad, tamaño de cristales, microfisuras y espesor\index{espesor}es son algunas las características que se pudieron evaluar con esta técnica. Además, el equipo utilizado nos permitió hacer análisis por espectroscopia\index{espectroscopia} de rayos X\index{rayos Xssh} dispersiva en energía\index{energía} (EDS, del inglés \textit{Energy Dispersive Spectroscopy}) y tomar imágenes tanto con electrones secundarios como con electrones retrodifundidos. \cite{Goodhew2000,Watt1997}

		Se utilizó un microscopio de la marca \textit{FEI}, modelo \textit{Helios NanoLab 650} equipado con dos columnas, una de iones de galio\index{galio} y otra de electrones. Dejaremos la explicación de la microsopía de iones de Ga\index{galio} para la siguiente sección. La fuente de la columna de electrones es un emisor tipo FEG (del ingles \textit{Field Emission Gun}) y como instrumental de detección cuenta con detector de electrones secundarios (SE, del ingles \textit{Secondary Electron}), de electrones retrodifundidos (BSD, del ingles \textit{back scatter detector}) y de inmersión (TLD, \textit{Thought Lens Detector}), ver esquema presentado en la figura \ref{fig:sem-fib}. 

		Se utilizaron tensiones de trabajo bajas típicamente entre \SI{1}{\kilo\electronvolt} y \SI{5}{\kilo\electronvolt} e intensidades del orden de los \SI{25}{\pA}. La justificación de estos valores es que al acelerar los electrones con bajas tensiones la penetración en la muestra es pobre. Si bien depende del tipo de material, podemos estimar en base simulaciones de trabajos en la literatura especializada que, para oro\index{oro} o silicio, la penetración con los valores de tensión citados es de unos 50 a \SI{200}{\nm} \cite{Joy1984,Shur2012,Hafner2007}. Por el otro, se utilizo un flujo de electrones también bajo (\SI{25}{\pA}), de manera de evitar el apantallamiento debido a la acumulación de carga superficial en la muestra. Todos las imágenes de MEB en este trabajo incluyen las condiciones experimentales utilizadas en la barra de escala situada debajo de cada una de ellas.

	\subsection{Microscopía con iones de galio\index{galio} focalizados (FIB)}\label{sec:FIB}

		El bombardeo con haz de iones (FIB, del ingles \textit{focused ion beam}) es una técnica que se utiliza fundamentalmente para el análisis de materiales en general y en particular materiales de la industria de la microelectrónica\index{microelectrónica}, más específicamente para análisis de microsistemas (MEMS\index{MEMS}, del ingles \textit{Micro Electro Mechanical Systems}) y circuitos integrados (IC, del ingles \textit{Integrated Circuits}).

			\begin{figure}[ht]
			 		  \begin{center}
			 		  \includegraphics[width=\textwidth]{Imagenes/sem-fib.jpg}
			 		  \caption[Microscopio de doble haz FIB\index{FIB}/SEM]{Equipo de FIB\index{FIB}/SEM utilizados para realizar las observaciones, cortes y caracterizaciones de los sensores. Consta de un microscopio de barrido electrón\index{electrón}ico de alta resolución y de una fuente de galio\index{galio} líquido para realizar, entre otras cosas, cortes en la micro y nanoescala\index{nanoescala}.}
			 		  \label{fig:sem-fib}
			 		  \end{center}
			 		  \end{figure}

		Consiste en el bombardeo de iones de galio\index{galio} para desplazar los átomos de la muestras. El Ga\index{galio}$^{\circ}$ (que se almacena en en un reservorio en la cabeza de la columna) se licua y se ioniza para dar lugar a los iones de Ga\index{galio}${^+}$, los cuales mediante un sistema de lentes magnéticas (similar al usado en MEB)  aceleran y focalizan los iones sobre la muestra. 

		\begin{figure}[ht!]
			 		  \begin{center}
			 		  \includegraphics[width=0.60\textwidth]{Esquemas/sem-fib.pdf}
			 		  \caption[Esquema de las microscopia\index{microscopía}\index{microscopía}s FIB\index{FIB}/SEM]{Esquema donde se muestra la disposición de las columnas de electrones y de átomos de galio\index{galio} del \textit{Helios NanoLab 650} y los principales eventos que ocurren al impactar los haces con la muestras.}
			 		  \label{esq:sem-fib}
			 		  \end{center}
			 		  \end{figure}

		El impacto de los mismos desplaza los átomos de la muestra generando así <<cortes>> sobre la muestra. Previo al impacto se deposita sobre la muestra una delgada \index{película!delgada}capa de Pt\index{platino} ($\sim$\SI{150}{\nm}) para protección de la muestra y generar un borde de corte mas abrupto, ya que la taza de desplazamiento de los átomos de Pt\index{platino} con iones Ga\index{galio}${^+}$ es baja.\cite{Giannuzzi2005,Orloff1996} La técnica es de especial utilidad para examinar secciones transversales de muestras, calcular espesor\index{espesor}es, reconstruir volumenes en 3D, preparar láminas para microscopia\index{microscopía}\index{microscopía} electrón\index{electrón}ica de trasmisión, entre otros tantos ejemplos. El haz de iones se utiliza en la mayoría de los casos a \SI{30}{\kilo\electronvolt}. En el esquema \ref{esq:sem-fib} se muestra como es la disposición de las columnas y en la figura \ref{fig:sem-fib} una foto del equipo utilizado ubicado en los laboratorios del INTI\index{INTI}-CMNB.\cite{Orloff2003,Reyntjens2001}

\section{Mediciones Electroquímicas}\label{sec:medidas_eq}
		
			Las mediciones electroquímica\index{electroquimico}s\index{electroquimico} fueron una parte central de este trabajo. Se utilizaron dos tipos de técnicas voltamperométricas, de corriente continua y de corriente alterna. No solo se hizo uso de ellas como técnicas analíticas sino también como herramienta para establecer parámetros de transporte, concentración dentro y fuera de los poros, calcular constantes de difusión\index{difusión} y e inferir mecanismos de transporte\index{transporte} de las sonda\index{sonda}s a través de la red nanoporosa. 

			Se utilizó, para ambas técnicas la típica configuración de celda de tres electrodos. La celda en sí fue fabricada en acrílico, con un volumen aproximado de \SI{3}{\ml} y con un orificio en la parte inferior. Para sellar contra el sustrato\index{sustrato} y evitar perdidas de la solución, se utilizo un sello de \SI{1}{\mm} de radio, el cual determina el área geométrica utilizada en \SI{3.15}{\mm^{2}}. En la figura \ref{fig:celda} se muestra un esquema del sistema de medición EQ y en la figura \ref{esq:eq} una fotografía de una de las tantas mediciones realizadas. 

			 \begin{figure}[!ht]
			 		  \begin{center}
			 		  \includegraphics[width=\textwidth]{Esquemas/celda.pdf}
			 		  \caption[Configuración de una celda de tres electrodos]{Configuración de de una celda tres electrodos para realizar medidas electroquímica\index{electroquimico}s\index{electroquimico}.}
			 		  \label{fig:celda}
			 		  \end{center}
			 		  \end{figure}

	 		 \begin{figure}[ht!]
			 		  \begin{center}
			 		  \includegraphics[width=\textwidth]{Imagenes/eq.jpg}
			 		  \caption[Equipo para realizar la medidas electroquímica\index{electroquimico}s\index{electroquimico}]{Fotografía del instrumental que se utilizó a lo largo de la tesis para tomar las medidas electroquímica\index{electroquimico}s\index{electroquimico}.}
			 		  \label{esq:eq}
			 		  \end{center}
			 		  \end{figure}	
			 		  
			Las mediciones electroquímica\index{electroquimico}s\index{electroquimico} fueron tomadas con un potencionestato \textit{Teq4}, para las medidas que se necesitaron velocidades de barrido mayores a \SI{1}{\volt\per\second} se uso un \textit{Autolab}, de la firma \textit{Ecochemie}. Como electrodo de referencia\index{electrodo!de referencia} se utilizó un electrodo saturado de calomel (ESC) de la firma \textit{Cole-Parmer} y como contraelectrodo\index{electrodo!contraelectrodo} (CE) se utilizaron indistintamente electrodos de Au\index{electrodo!de Au} depositados por pulverización catódica\index{pulverización catódica} o una pieza de Pt\index{platino} de tamaño adecuado. En la tabla \ref{tabla:eq} se resumen los reactivos utilizados para llevar a cabo las mediciones electroquímica\index{electroquimico}s\index{electroquimico}. 
			
			A continuación se hace una breve reseña sobre las técnicas utilizadas, los parámetros empleados y las configuraciones experimentales.

			 \pagebreak

				%Tabla ractivos EQ
				     \begin{table}[ht!]
			  		  \caption[Reactivos utilizados para las mediciones electroquímica\index{electroquimico}s\index{electroquimico}]{Reactivos y sonda\index{sonda}s electroquímica\index{electroquimico}s\index{electroquimico} utilizados para las mediciones electroquímica\index{electroquimico}s\index{electroquimico}}
			  		   \begin{tabular}{>{\raggedright\arraybackslash}m{4.4cm}>{\centering\arraybackslash}m{1.75cm}>{\centering\arraybackslash}m{2.7cm}>{\raggedright\arraybackslash}m{1.6cm}} 
			  		  \toprule
					  Reactivo \hspace{3cm}Nombre& Marca & Peso Molecular (\si{g.mol^{-1}}) & Función  \\ \midrule
			    	  \ferroCompleto \hspace{3cm} ferrocianuro de potasio\index{ferrocianuro de potasio} & \textit{Sigma} & 422,41  & Sonda \\ \midrule
			    	  \ferriCompleto \hspace{3cm} ferricianuro de potasio\index{ferricianuro de potasio} & \textit{Sigma} & 329,27  & Sonda  \\ \midrule
			  		  \aminorutenioCompleto  \hspace{3cm}  cloruro de hexaaminorutenio& \textit{Aldrich} &  309,61  & Sonda  \\ \midrule
			  		  \raisebox{-.5\height}{\includegraphics[scale=0.4]{Esquemas/Fc.pdf}}  \hspace{3cm} ferroceno metanol\index{ferroceno metanol}   & \textit{Aldrich} &  216,06 & Sonda  \\ \midrule
			  		  \raisebox{-.5\height}{\includegraphics[scale=0.4]{Esquemas/HQ.pdf}} \hspace{3cm} hidroquinona	& \textit{Biopack} & 110.11  & Sonda  \\ \midrule
			  		  H$_2$O \hspace{3cm} agua &  \SI{18}{\mega\ohm\per\cm}  &  18,02 & Solvente \\ \midrule
			  		  KCl  \hspace{3cm} cloruro de potasio   & \textit{Biopack} & 74,56 & Electrolito Soporte \\
 			  		  \bottomrule
			    	  \end{tabular}
			   		  \label{tabla:eq}
			   		  \end{table}

	 \subsection{Voltametría cíclica}
	 		
	 		La voltamperometría cíclica (VC) consiste en variar, de una manera cíclica, el potencial de un electrodo estacionario contra un electrodo de referencia\index{electrodo!de referencia}. Ambos se encuentran inmersos en una solución en reposo y se mide la corriente resultante entre ellos. La señal de excitación es un barrido de potencial lineal con una onda de forma triangular, la cual parte de un potencial E$_1$, evoluciona linealmente hasta un potencial E$_2$ para luego volver a E$_1$. Las velocidades de este barrido pueden variar desde unos cuantos milivolts por segundo hasta cientos de volts por segundo; en nuestro caso se utilizaron velocidades próximas a los \SI{50}{\milli\volt.\second^{-1}}, se escogieron estas velocidades para llevar a cabo experimentos de una duración aceptable pero donde aún predomine la transferencia de carga y no se observe desplazamiento de potenciales para los picos de máxima oxidación y/o reducción.\cite{nicholson1964,Gewirth2004}

	 		Como ya se dijo anteriormente se barre el potencial del electrodo de trabajo\index{electrodo!de trabajo} en dirección de ida y vuelta entre dos valores arbitrarios, E$_1$ y E$_2$. Al\index{aluminio} usar soluciones en base acuosa se debe trabajar en la región de estabilidad electroquímica\index{electroquimico} del H$_2$O, para evitar reducción u oxidación de la misma, que genera H$_2$ u O$_2$ respectivamente. En la gran mayoría de los experimentos presentados en este trabajo se trabajo en un pH\index{pH}$\sim 5$ para el cual el rango de estabilidad del agua es entre \SI{-0.5}{\volt} y \SI{0.7}{\volt}, usando como referencia un electrodo saturado de calomel.\cite{wang2014} 

	 		En la figura \ref{fig:CV_ideal} se muestra la onda triangular de excitación aplicada y la curva obtenida para una sonda\index{sonda} electroquímica\index{electroquimico} idealmente reversible, donde se destacan los parámetros mas importantes.
	 			 \begin{figure}[ht]
			  		  \begin{subfigure}[t]{0.495\textwidth}
			  		  \includegraphics[width=\textwidth]{Graficos/onda-triangular.pdf}
			  		  \end{subfigure}
			  		  \begin{subfigure}[t]{0.495\textwidth}
			  		  \includegraphics[width=\textwidth]{Esquemas/CV-ideal.pdf}
			  		  \end{subfigure}
			  		  \caption[Voltamperometria ideal reversible]{Curva de excitación y voltagrama típico para una especie redox reversible.}
			  		  \label{fig:CV_ideal}
			  		  \end{figure}

	 		Esta técnica se utilizó para evaluar fenómenos de exclusión, permeación\index{permeacion} y preconcentración. También para determinar concentración de las sonda\index{sonda}s electroactivas dentro y fuera de los poros, calcular coeficientes de difusión\index{difusión} y estimar distancias de sitios redox\index{sitio redox\index{sitio redox}} así como chequear accesibilidad\index{accesibilidad} y estructura de las películas delgadas mesoporosas.

	 \subsection{Voltametría cíclica de corriente alterna}

	 		La técnica de voltatametría cíclica\index{voltametria!cíclica} de corriente alterna (VCA) consta en aplicar una oscilación sinusoidal de voltaje a la celda electroquímica\index{electroquimico}. A la onda triangular clásica usada en VC se la perturba, montando sobre ella, una pequeña onda de corriente alterna, en los experimentos presentados en este trabajo la perturbación fue de de \SI{10}{\milli\volt} y la frecuencia de la misma de 1 y \SI{2}{\hertz}. Esta técnica se emplea en conjunto con un analizador de frecuencias para filtrar la componente continua de la alterna, de este modo, ofrece un limite de detección menor e incrementa la sensibilidad respecto de la CV tradicional.\cite{Wi2000,Skoog1995}

	 		En la figura \ref{fig:ACV_ideal} se muestra la onda triangular con la perturbación, y la curva obtenida para una sonda\index{sonda} electroquímica\index{electroquimico} idealmente reversible, luego del filtrado de la componente continua.

	 			 \begin{figure}[ht]
			  		  \begin{subfigure}[t]{0.495\textwidth}
			  		  \includegraphics[width=\textwidth]{Graficos/onda-triangular-sin.pdf}
			  		  \end{subfigure}
			  		  \begin{subfigure}[t]{0.495\textwidth}
			  		  \includegraphics[width=\textwidth]{Graficos/ACV-ideal.pdf}
			  		  \end{subfigure}
			  		  \caption[Voltamperometria ideal reversible]{Curva de excitación y voltagrama típico para una especie redox reversible.}
			  		  \label{fig:ACV_ideal}
			  		  \end{figure}
	 		
	 		El propósito de esta técnica fue el obtener el coeficiente de difusión\index{difusión} de hexaaminorutenio en sistemas porosos y contrastar con otras técnicas de forma de validar dicho coeficiente y los mecanismos de trasporte propuestos. 

	 \subsection{Simulaciones electroquímica\index{electroquimico}s\index{electroquimico}}\label{simulacion}

	 	 Para validar las hipótesis de transporte\index{transporte} planteadas en el capitulo \ref{chap:Electroquimica} se llevaron a cabo simulaciones por computadora para voltametrías cíclicas. Las mismas se hicieron con el software  \textit{COMSOL Multiphysics\textsuperscript\textregistered} (\url{https://www.comsol.com/}) el cual simula las voltametrías utilizando elementos finitos. En la tabla \ref{tabla:simulacion} se resumen las variables utilizados para las simulaciones, sus valores y su descripción.
	 	
	    	\begin{table}[ht!]
	 	    \caption[Parámetros de las simulacinoes]{Parámetros y valores de entrada usadas durante las simulaciones de las voltametrías cíclicas.}
	 	    \begin{tabular}{>{\raggedright\arraybackslash}m{1.4cm}>{\centering\arraybackslash}m{2.8cm}>{\raggedright\arraybackslash}m{6.7cm}} 
	 	    \toprule
	 	    Variable  & 	Valor  &   descripción      \\ \midrule
	 	    $h$  	  &    \SI{200}{nm}	& 	   espesor\index{espesor} de las películas delgadas mesoporosas 	    \\ \midrule
	 	    $C_{\fc}$  & \SI{5}{\milli\Molar}  & concentración de FeOH en solución    \\ \midrule
	 	    $C_{\ru}$ & \SI{1}{\milli\Molar}  & concentración de \ru\space en las películas    \\ \midrule
	 	    $k$ 		   & variable 	 & 	constante de mediación\index{mediacion} redox    \\ \midrule
	 	    $E^\circ_{\ru}$  & \SI{-0.3}{\volt} vs ESC & potencial reducción estandar el \ru \\ \midrule
	 	    $E^\circ_{\fc}$  & \SI{0.3}{\volt} vs ESC & potencial reducción estandar el \fc \\ \midrule
	 	    $D_{\fc}$  & variable & coeficiente de difusión\index{difusión} del \fc\space en el film \\ \midrule
	 	    $D_{e}$  & variable & coeficiente de difusión\index{difusión} por \textit{electron hopping }del \ru\space en el film \\ \midrule
	 	    $\nu$    & \SI{50}{\milli\volt\per\second}  &  velocidad de barrido\index{velocidad!de barrido} \\
	 	     \bottomrule
			\end{tabular}
			\label{tabla:simulacion}
			\end{table} 

%Revisión de LOGS

% B: /home/gustavo/Dropbox/Tesis/Capitulos/02_materiales.tex:56 Overfull --> tabla ok!
% B: /home/gustavo/Dropbox/Tesis/Capitulos/02_materiales.tex:57 Overfull --> tabla ok!
% B: /home/gustavo/Dropbox/Tesis/Capitulos/02_materiales.tex:58 Overfull --> tabla ok!
% B: /home/gustavo/Dropbox/Tesis/Capitulos/02_materiales.tex:268 Overfull --> tabla ok!
% B: /home/gustavo/Dropbox/Tesis/Capitulos/02_materiales.tex:268 Overfull --> tabla ok!
% B: /home/gustavo/Dropbox/Tesis/Capitulos/02_materiales.tex:269 Overfull --> tabla ok!
% B: /home/gustavo/Dropbox/Tesis/Capitulos/02_materiales.tex:270 Overfull --> tabla ok!
% B: /home/gustavo/Dropbox/Tesis/Capitulos/02_materiales.tex:270 Overfull --> tabla ok!
% B: /home/gustavo/Dropbox/Tesis/Capitulos/02_materiales.tex:271 Overfull --> tabla ok!
% B: /home/gustavo/Dropbox/Tesis/Capitulos/02_materiales.tex:275 Overfull --> tabla ok!
% B: /home/gustavo/Dropbox/Tesis/Capitulos/02_materiales.tex:0 Underfull --> Pagina de la foto de la alineadora\index{alineadora}, mucho espacio. Aceptable. OK!
% B: /home/gustavo/Dropbox/Tesis/Capitulos/02_materiales.tex:344 Overfull --> tabla ok!
% B: /home/gustavo/Dropbox/Tesis/Capitulos/02_materiales.tex:504 Overfull --> tabla ok!
% B: /home/gustavo/Dropbox/Tesis/Capitulos/02_materiales.tex:0 Underfull --> Pagina del esquema de la celda --> Aceptable 
% B: /home/gustavo/Dropbox/Tesis/Capitulos/02_materiales.tex:0 Underfull --> Pagina de la tabla de reactivos --> Aceptable
% B: /home/gustavo/Dropbox/Tesis/Capitulos/02_materiales.tex:572 Overfull --> tabla ok!