%Linea Para poder completar automaticamente las citas con el Sublime
%No hace el documento, se puede borrar esta linea si no se usa el Sublime
%------------------------------------------------------------------------------
 \newcommand{\NoBiblioMat}[1]{
 \ifthenelse{\equal{#1}{verdadero}}{}{\bibliography{Referencias/base_bibliografica}}
 \NoBiblioMat{verdadero}}
 %-----------------------------------------------------------------------------

%Formato (Nombre de capitulo largo o corto), nombre del capitulo y estilo de la
%Portada del Capitulo
%------------------------------------------------------------------------------

 %Formato en si, titulo en un solo renglon
 \FormatoCapituloUnaLinea

 %Nombre y etiquete para referir
 \chapter{Materiales, Métodos y Procesos}\label{chap:Materiales}

 %Para que no salga el numero de pagina en la portada del capitulo
 \thispagestyle{empty}
	
 %Resumen del Capitulo en Italica 
 % \noindent\textit{En este capítulo se presenta la descripción experimental de todos los materiales, instrumental y procesos involucrados en la tesis. En la primera sección se detallan los procesos de síntesis de las películas mesoporosas, desde la preparación de los soles\index{sol} hasta la caracterización de las mismas; en la segunda se presentan los procesos de fabricación de los electrodos, desde el diseño a las técnicas de transferencia; la tercera sección describe las técnicas de microscopía\index{microscopía}s utilizadas y la última sección detalla como se llevaron a cabo las mediciones y simulaciones electroquímica\index{electroquimico}s\index{electroquimico}.}


 %Indice de capitulo alineada al borde inferior de la pagina, nueva pagina
 \vfill
 \minitoc
 \newpage

 %-------------------------------------------------------------------------------

\section{Síntesis de películas delgadas mesoporosas}\label{sec:sintesis_mesoporosos}	
	
	 \vspace{-2mm}Las consideraciones teóricas sobre la química sol-gel\index{sol-gel} y el autoensamblado inducido por evaporación\index{autoensamblado inducido por evaporación} (AEIE\index{AEIE}) ya fueron expuestas en el capítulo \ref{chap:Introduccion}. También fueron mencionadas las razones por las cuales se eligió SiO$_2$ como estructura para las películas delgadas mesoporosas y Pluronic F127\index{Pluronic F127}, Brij58\index{Brij58} y CTAB como agente moldeante\index{agente moldeante}. Los procedimientos, métodos y proporciones molares para la preparación de los soles\index{sol} se inspiraron en los trabajos de Angelomé\index{Angelomé}\cite{Angelome2008} y Fuertes\index{Fuertes}\cite{Fuertes2009}. El esquema \ref{esq:peliculas_meso} resume cada una de las etapas de síntesis de las películas las cuales se desarrollarán con detalle las próximas secciones de este capítulo.
	      \vspace*{-0.2cm}
		  \begin{figure}[ht]
			  \begin{center}
			  \includegraphics[width=\textwidth]{Esquemas/Resumen_sintesis_meso.pdf}
			  \caption[Síntesis de películas delgadas mesoporosas]{Diagrama de flujo para las dos rutas sínteticas utlizadas en la síntesis de películas delgadas mesoporosas: alta temperatura (AT, \SI{350}{\celsius}) y baja temperatura (BT, \SI{130}{\celsius}).}
			  \label{esq:peliculas_meso}
			  \end{center}
			  \end{figure}
			  \vspace*{-0.7cm}

	\subsection{Preparación de los soles, reactivos y nomenclatura}\label{sec:soles}
		
			La síntesis y depósito\index{depósito} de las películas delgadas mesoporosas comienzan con la preparación de las soluciones, las cuales deben contener los precursor\index{precursor}es del óxido (o de los óxidos en el caso de películas mixtas), el moldeante de los poros, el solvente adecuado, agua\index{agua} y ácido\index{acido@ácido} clorhídrico\cite{Brinker1990} (cada componente cumple una función específica, tal como se explicó en la sección \ref{sec:mesoporosos}). Los precursor\index{precursor}es utlizados fueron tetraetoxisilano\index{tetraetoxisilano} (TEOS, \textit{Merck}) para las películas de sílice\index{silicio!oxido de} pura, y TEOS combinado con cloruro de circonio(IV) (ZrCl$_4$, \textit{Aldrich}) para las películas mixtas de silicio/circonio. Las condiciones de hidrólisis\index{hidrolisis@hidrólisis}\index{hidrólisis} y condensación\index{condensación} para estos dos reactivos (ya sean solos o combinados) son bien conocidas y llevan a la formación películas delgadas estables y reproducibles de óxidos mesoporosos puros o mixtos\cite{Soler-Illia2004,Crepaldi2002a,Angelome2008}. El surfactante\index{surfactante} es el agente que establece el tamaño de los poros y la simetría del sistema. Para ello se utilizaron tres moldes diferentes: el copolímero de bloque Pluronic F127\index{Pluronic F127} (F127, \textit{Aldrich}), bromuro de hexadeciltrimetilamonio\index{bromuro de hexadeciltrimetilamonio} (CTAB, \textit{Aldrich}) y polioxietileno[20] cetil éter (Brij58,\textit{Aldrich}). Como solvente se utilizó etanol \index{etanol}absoluto (EtOH, \textit{Sigma}). El H$_2$O es el reactivo para la formación del óxido mediante la conexión de los grupos metálicos M(IV). Por último, el HCl\index{acido@ácido!clohídrico} es el encargado de generar el medio ácido\index{acido@ácido} que cataliza la hidrólisis\index{hidrolisis@hidrólisis}\index{hidrólisis} y controla la condensación\index{condensación} del Si(IV) y/o del Zr\index{circonio}(IV). Los reactivos utilizados fueron de calidad proanálisis o superior y el H$_2$O de \SI{18}{\mega\ohm\per\cm} fue obtenida con un equipo \textit{Ultra Clear TWF} de la marca \textit{Siemens}. La nomenclatura, pesos moleculares y estructura química de los reactivos utilizados se pueden consultar en la tabla \ref{tabla:reactivos}.
					
			El preparado de las soluciones se realizó agregando cada reactivo por pesada en balanza analítica. Cada lote de solución fue de aproximadamente \SI{100}{\ml}. Para llegar a este volumen se agregaron, en este orden, \SI{10.417}{\gram} de TEOS, \SI{6.911}{\gram} de etanol \index{etanol}y \SI{0.902}{\gram} de HCl\index{acido@ácido!clohídrico} \SI{2,77e-3}{\Molar}. En el caso de los soles\index{sol} mixtos (Si\textbar Zr\index{circonio} 9:1), se pesaron \SI{9.375}{\gram} de TEOS y \SI{1.165}{\gram} de Zr\index{circonio}Cl$_4$. Esta primera solución, denominada solución de prehidrólisis, se deja envejecer bajo agitación constante durante \SI{48}{\hour} a \SI{25}{\celsius}, con el objetivo de hidrolizar los precursor\index{precursor}es metálicos y mantener un bajo grado de condensación\index{condensación}.\cite{Grosso2001}

				\begin{table}[ht!] 
						  \caption[Reactivos para los soles]{Nomenclatura, estructura, peso molecular y función de las moléculas utilizadas en las soluciones para la síntesis de películas delgadas mesoporosas.} 
				  		  \begin{tabular}{>{\raggedright\arraybackslash}m{2.40cm}>{\centering\arraybackslash}m{4cm}>{\centering\arraybackslash}m{2.35cm}>{\raggedright\arraybackslash}m{1.7cm}} 
				  		  \toprule
						  Nombre Nomenclatura    & Estructura & Peso molecular \si{g.mol^{-1}} & Función\\ \midrule
				      	  tetraetoxisilano\index{tetraetoxisilano} TEOS & \includegraphics[scale=0.5]{Esquemas/teos.pdf} & $208,33$ & precursor\index{precursor} del óxido  \\ \midrule
				      	  \mbox{cloruro de circonio(IV)}  Zr\index{circonio}Cl$_4$ & \includegraphics[scale=0.8]{Esquemas/zrcl4.pdf} & $233.04$ & precursor\index{precursor} del óxido  \\ \midrule
				  		  Pluronic F127\index{Pluronic F127} F127    & \hspace*{-10px} \includegraphics[scale=0.5]{Esquemas/f127.pdf} & \multirow{1}{*}{$13800$}	 & agente moldeante\index{agente moldeante}	 \\ \midrule
				  		  bromuro de hexadeciltrimetilamonio\index{bromuro de hexadeciltrimetilamonio}  CTAB   & \hspace*{1cm} \includegraphics[scale=0.6]{Esquemas/ctab.pdf} & \multirow{1}{*}{$364.48$}	 & agente moldeante\index{agente moldeante}	 \\ \midrule
				  		   polioxietileno[20] cetil éter\hspace{2cm}Brij58   & \hspace*{0.5cm}\includegraphics[scale=0.65]{Esquemas/brij58.pdf} & \multirow{1}{*}{$1124$}	 & agente moldeante\index{agente moldeante}	 \\ \midrule
				  		  \mbox{ácido clorhídrico} HCl\index{acido@ácido!clohídrico}& \includegraphics{Esquemas/hcl.pdf}  & \multirow{1}{*}{$36,46$}   & cataliza la hidrólisis\index{hidrolisis@hidrólisis}\index{hidrólisis} \\ \midrule
				  		  agua\index{agua} \hspace{2cm} H$_2$O  &  \includegraphics{Esquemas/agua.pdf}  & \multirow{1}{*}{$18,02$}   & reactivo de hidrólisis\index{hidrolisis@hidrólisis}\index{hidrólisis} \\ \midrule
				  		  etanol\hspace{0.75cm}EtOH  & \includegraphics{Esquemas/etanol.pdf}  & \multirow{1}{*}{$46,07$}   & solvente \\ 
				  		  \bottomrule
				    	  \end{tabular}
				   		  \label{tabla:reactivos}
					      \end{table}
			\vspace*{-0.6cm}			      
			Una vez envejecida la solución de prehidrólisis\index{prehidrólisis} (ya sea de SiO$_2$ pura o mixta) se pesan  \SI{17.146}{\gram} y se agregan: \SI{80.184}{\gram} de EtOH\index{etanol}, el surfactante\index{surfactante} deseado (\SI{3.246}{\gram} de F127 o \SI{1.822}{\gram} de CTAB o \SI{2.646}{\gram} de Brij58\index{Brij58}) y \SI{7.630}{\gram} de HCl\index{acido@ácido!clohídrico} \SI{5,5e-2}{\Molar}. De esta forma se obtienen aproximadamente \SI{100}{\ml} de un sol\index{sol} con las relaciones molares de la tabla \ref{tabla:soles}. Se conservan en \textit{frezeer} a \SI{-18}{\celsius} y sólo se retiran a la hora de depositar las películas. 

			Para facilitar la lectura se utilizará la siguiente nomenclatura, tanto para los soles\index{sol} como para las películas delgadas mesoporosas que se fabriquen con ellos: 

				\begin{itemize}
			     \item \pdm\space para películas delgadas mesoporosas en general.
			     \item \pdmF\space para \pdm\space de óxido de silicio\index{silicio!oxido de}\index{silicio} estructuradas con F127. 
			     \item \pdmC\space para \pdm\space de óxido de silicio\index{silicio!oxido de}\index{silicio} estructuradas con CTAB.
			     \item \pdmZ\space para las \pdm\space mixtas de óxido de circonio\index{circonio} y silicio\index{silicio} en relación molar $1\!:\!9$ y estructuradas con F127.
			     \item \pdmZB\space para las \pdm\space mixtas de óxido de circonio\index{circonio} y silicio\index{silicio} en relación molar $1\!:\!9$ y estructuradas con Brij58\index{Brij58}. 
      		     \end{itemize}	
			
			Todas las soluciones fueron preparadas indistintamente en el Ce\index{cerio}ntro de Micro y Nanoelectrónica del Bicentenario del Instituto Nacional de Tecnología Industrial (INTI-CMNB) o en la Gerencia Química, Ce\index{cerio}ntro Atómico Constituyentes Comisión Nacional de Energía Atómica (CAC-CNEA). 
					
				\begin{table}[h!]
			  		  \caption[Relación molares de los soles]{Relaciones molares para las soluciones utilizadas.} 
			  		  \begin{tabular}{>{\raggedright\arraybackslash}m{2cm}>{\centering\arraybackslash}m{2cm}>{\centering\arraybackslash}m{1.4cm}>{\centering\arraybackslash}m{1.4cm}>{\centering\arraybackslash}m{1.4cm}>{\centering\arraybackslash}m{1.4cm}}
			  		  \toprule
					  Componente & Prehidrólisis  & \pdmF   & \pdmC  & \pdmZ & \pdmZB \\ \midrule
			      	  TEOS 		  & 1/0,9$^*$	  & 1   	& 1		 & 0,9   & 0,9    \\ %\midrule
			      	  Zr\index{circonio}Cl$_4$	  & -/0,1$^*$	  &	-		& - 	 & 0,1   & 0,1    \\ %\midrule	
			      	  EtOH\index{etanol} 		  & 3			  & 40   	& 40	 & 40    & 40     \\ %\midrule
			      	  F127 		  & -		 	  & 0,0075  & -		 & 0,0075& -      \\ %\midrule
			      	  CTAB 		  & -             & -		& 0,1	 & -     & -      \\ %\midrule
			      	  Brij58\index{Brij58}      & -             & -       & -      & -     & 0,05   \\ %\midrule
			      	  H$_2$O	  & 1			  & 9	  	& 9	     & 9     & 9      \\ %\midrule
			      	  HCl\index{acido@ácido!clohídrico}    	  & 0,00005		  & 0,01   	& 0,01	 & 0,01  & 0,01   \\ 
			      	  \bottomrule
			    	  \end{tabular}\vspace*{2pt}
		    	  	  \footnotesize{$^*$Los números antes y después de la barra son los utilizados en soluciones de prehidrólisis\index{prehidrólisis} para películas de óxido de silicio\index{silicio!oxido de}\index{silicio} y óxidos mixtos de silicio/circonio respectivamente.}
			    	  \label{tabla:soles}
			   		  \end{table}

	\subsection{Depósitos de las películas delgadas mesoporosas}\label{sec:deposito_pdm}

			Las películas mesoporosas utilizadas en esta tesis fueron depositadas en el Laboratorio de Fotolitografía del INTI\index{INTI}-CMNB por \textit{spin-coating}\index{spin@\textit{spin-coating}}. Dicha técnica consiste en 
			dispensar el sol\index{sol} sobre el sustrato, el cual está sujeto a un portamuestras que a su vez se encuentra acoplado a un cabezal rotatorio. Al\space hacer rotar el cabezal, se aplica sobre el líquido una fuerza proporcional a la velocidad angular\index{velocidad!angular}, dispersando el sol\index{sol} para formar un recubrimiento homogéneo sobre la muestra. El espesor\index{espesor} del depósito\index{depósito} está regulado, entre otras variables, por la velocidad angular\index{velocidad!angular} y la viscosidad del sol.

			Como sustrato\index{sustrato} para realizar los depósitos se utilizaron vidrio, silicio, oro\index{oro} sobre silicio, microelectrodo\index{electrodo!microelectrodo}s y sustratos poliméricos como  polimetilmetacrilato (PMMA) y poliestireno\index{poliestireno} de alto impacto (PAI). Cada uno de ellos fue escogido para una función particular (p. ej. sustrato\index{sustrato} para reacciones electroquímica\index{electroquimico}s\index{electroquimico}) o por alguna característica distintiva (p. ej. transparente en el IR). En la tabla \ref{tabla:sustratos} se agrupan los sustratos utilizados y se resumen algunas características y funciones destacadas.

			Las dimensiones de las muestras fueron típicamente de \SI{1x1}{\cm} a \linebreak \SI{2x2}{\cm}, aunque la técnica permite depositar películas continuas de hasta \SI{15}{cm} de diámetro. En la mayoría de los casos, para obtener lotes de 32 o 46 sensores\index{sensor} (dependiendo del diseño), se utilizaron obleas de silicio\index{silicio} de \SI{10}{\cm} de diámetro. Antes de hacer el depósito, el sol\index{sol} se pasa a través de un filtro de jeringa\index{jeringa} de nailon de \SI{0.45}{\um} (\textit{GAMAFIL}) para evitar la presencia de partículas que puedan generar discontinuidades o <<cometa\index{cometa}s>> en los depósitos\cite{Franssila2004}. Luego, para dispensar el sol\index{sol} en el sustrato, se utilizaron pipetas tipo Pasteur\index{Pasteur} o pipetas automáticas dependiendo del volumen requerido, el cual varió de 80 a \SI{100}{\uL.\cm^{-2}}. 

			Las condiciones del laboratorio durante el depósito\index{depósito} se mantuvieron en \SI{25}{\celsius} y a una HR entre 30\% y 50\%. Una vez dispensado el sol, se da comienzo a la rotación que dispersa la solución de manera homogénea sobre el sustrato\index{sustrato} y, a su vez, la evaporación del solvente promueve la formación del cristal líquido\index{cristal líquido} por el mecanismo de AEIE\index{AEIE}\cite{Brinker1999}.

				 \begin{table}[t!]
			  		   \caption[Sustratos utilizados para el depósito\index{depósito} de \pdm]{Sustratos utilizados para el depósito\index{depósito} de \pdm.} 
			  		   \begin{tabular}{>{\raggedright\arraybackslash}m{2.4cm}>{\raggedright\arraybackslash}m{2.5cm}>{\raggedright\arraybackslash}m{2cm}>{\raggedright\arraybackslash}m{3.55cm}} 
			  		   \toprule
					   Sustrato Nomenclatura   & Observaciones  & Limpieza previa$^*$ & Función \\ \midrule
			       	   vidrio\index{vidrio} \hspace{2cm} Vi  &	portaobjetos \textit{BioTraza} & inmersión KOH 40\% & económico para pruebas preliminares de depósito\index{depósito} \\ \midrule
			       	   silicio\hspace{2cm} Si \index{silicio} & Si[100] pulido dopado tipo n  \textit{Addison}& inmersión HF\index{acido@ácido!fluohídrico} 48\% & FTIR, MEB\index{MEB}, FIB\index{FIB}, PEA \\ \midrule
			       	   Au \index{oro}sobre silicio\hspace{2cm} Si$|$Au \index{oro}& depositado por pulverización catódica\index{pulverización catódica}$^\dagger$  & ultrasonido\hspace{1cm}en H$_2$O  & transporte, EQ\\ \midrule
			      	   microelectrodo\index{electrodo!microelectrodo}s \hspace{2cm} $\mu Elec$ & sensores, diseño transferido por fotolitografía\index{fotolitografía}$^\mathsection$  	  &  ultrasonido\hspace{1cm}en H$_2$O  & multisensado, EQ \\ \midrule
			      	   poliméricos         &  PMMA y PAI		  &  ultrasonido\hspace{1cm}en H$_2$O &  demostrador métodos suaves de síntesis\\ 
			      	   \bottomrule
			    	   \end{tabular}\vspace*{2pt}
			    	   \footnotesize{$^*$Ver la sección <<\nameref{sec:limpieza}>>, tabla \ref{tabla:limpieza}, pág. \pageref{sec:limpieza}.}\\
			    	   \footnotesize{$^\dagger$Ver la sección <<\nameref{sec:sputt}>>, pág.\pageref{sec:sputt}.} \\
			    	   \footnotesize{$^\mathsection$Ver la sección <<\nameref{sec:fotolito}>>, pág. \pageref{sec:sputt}.}
			    	   \label{tabla:sustratos}
			   		   \end{table}

			Los mejores resultados se obtienen aplicando al inicio una rampa de veloci\index{velocidad!rampa de}dad moderada para formar un depósito\index{depósito} relativamente grueso y uniforme. Una vez formada esta capa uniforme se acelera hasta la velocidad final para obtener al espesor\index{espesor} final requerido y favorecer la evaporación del solvente mediante la corriente de aire generada sobre el depósito. Bajo estas condiciones el espesor\index{espesor} puede ser calculado por la ecuación \ref{eq:espesor_pdm}\cite{zhang2010}, donde $d(t)$ es el espesor\index{espesor} en función del tiempo, $d_0$ es el espesor\index{espesor} a $t=0$, $\rho$ es la densidad del sol, $\eta$ la viscosidad y $\omega$ la velocidad angular\index{velocidad!angular}, la cual es para los casos tratados en este trabajo la principal variable para ajustar el espesor\index{espesor} de los depósitos.\cite{sahu2009,zhang2010a,emslie1958}

			   		\begin{equation}
						d(t)=\dfrac{d_0}{\sqrt{1+\dfrac{4 \rho \omega^2 d_0^2 t}{3 \eta}}}
						\label{eq:espesor_pdm}
					\end{equation}

			En la mayoría de los caso dicha ecuación es difícil de aplicar en la práctica y en general se establece relación empírica entre el espesor\index{espesor} resultante y las condiciones experimentales.
			En este trabajo se  optimizaron las rampas de velocidad y aceleración para obtener \pdm\space con espesor\index{espesor}es entre 150 y \SI{300}{\nm}\cite{Meyerhofer1978,Hall1998,Brinker1990}. Los esquemas aplicados se muestran en gráfico de la figura \ref{fig:rampa-spin}.			

			 El equipo utilizado fue un \textit{Suss MicroTec Delta 20BM}, el cual consiste en un cabezal rotatorio con control de aceleración (0 a  \SI{1000}{\minute^{-1}.\second^{-1}}) y velocidad angular\index{velocidad!angular} (0 a \SI{10000}{\minute^{-1}}). Posee portamuestras con entrada de vacío para sujetar las muestras y de diferentes tamaños para adaptarse a sustratos de diversos tamaños y geometrías (figura \ref{fig:spin}).   		   

			 	\clearpage
			 	\begin{figure}[th!]
						 \begin{center}
						 \includegraphics[width=0.80\textwidth]{Graficos/rotacion_meso.pdf}
						 \caption[Parámetros de depósito\index{depósito} para las \pdm]{Esquema con las rampas de velocidad de rotación\index{velocidad!de rotación} o velocidad angular\index{velocidad!angular} mas frecuentemente utilizadas para el depósito\index{depósito} de \pdm. A mayor velocidad y mayor tiempo, menor es el espesor\index{espesor} de los depósitos resultantes.}
						 \label{fig:rampa-spin}
						 \end{center}
						 \end{figure}

				\begin{figure}[th!]
					  \begin{center}
					  \includegraphics[width=\textwidth]{Imagenes/Spin.jpg}
					  \caption[Equipo para el depósito\index{depósito} de películas delgadas, \textit{spin-coater}]{\textit{Spin-coater} ubicado en el Laboratorio de Fotolitografía del INTI\index{INTI}-CMNB utilizado para el depósito\index{depósito} de las películas delgadas mesoporosas, Marca \textit{Suss MicroTec}, modelo \textit{Delta 20BM}.}
					  \label{fig:spin}
					  \end{center}
					  \end{figure}
	
	\subsection{Métodos de condensación\index{condensación} y extracción\index{extracción}}\label{sec:cond_y_extr}

		Una vez realizado el depósito, se debe conservar la estructura del cristal líquido\index{cristal líquido} obtenido y evitar el deterioro durante la eliminación\index{eliminación} del surfactante\index{surfactante}. Para ello se estabilizó la película durante \SI{1}{\hour} en cámara de humedad controlada a una HR constante de 50\%. Para mantener dicho valor de humedad se utilizó una solución saturada de Ca(NO$_3$)$_2$.5H$_2$O (\textit{Biopack}). Controlar la presión parcial de agua\index{agua} (P$_{\text{H}_2\text{O}}$) permite optimizar el grado de condensación\index{condensación} del óxido y ayuda a la separación de fases entre el agente moldeante\index{agente moldeante} y el óxido\cite{Crepaldi2003}. El proceso de estabilización y condensación\index{condensación} del óxido continúa con un calentamiento en plancha calefactora (\textit{Cimarec}) una hora a \SI{60}{\celsius} y una hora más a \SI{130}{\celsius}\cite{Crepaldi2003,Crepaldi2002a}. 
				
		Posteriormente a la estabilización de la película se experimentaron varios tratamientos posdepósito para completar el proceso de condensación\index{condensación} de la fase inorgánica y extraer el surfactante\index{surfactante} para dar lugar finalmente a la película mesoporosa. A continuación se enumeran y describen brevemente los métodos ensayados:

				\begin{itemize}

				\item \textit{Calcinación.} Este es el proceso clásico en el cual se somete a la película a una temperatura de \SI{350}{\celsius} durante \SI{2}{\hour} con una rampa de \SI{1}{\celsius.\minute^{-1}} (Horno \textit{Indef 337}). De esta forma se condensa el óxido, se elimina el surfactante\index{surfactante} y se minimiza el daño de la estructura tridimensional de la red mesoporosa\index{película!mesoporosa} \cite{Crepaldi2003}.

				\item \textit{Condensación ácida.} En este método se busca promover la condensación\index{condensación} de la matriz de sílice\index{silicio!oxido de} mediante la exposición de las películas a una atmósfera\index{atmósfera} de vapores de HCl\index{acido@ácido!clohídrico} \cite{Doshi2000a}. El arreglo para tal fin consiste en sujetar las muestras al fondo de un vaso de precipitados y colocarlo invertido sobre un cristalizador con HCl\index{acido@ácido!clohídrico} concentrado (\textit{Biopack}) durante \SI{10}{\minute}. 

				\item \textit{Condensación alcalina.} Al\space igual que el método anterior, se busca promover la condensación\index{condensación} del óxido cambiando las condiciones del entorno químico, en este caso someter las películas a una atmósfera\index{atmósfera} de pH\index{pH} alcalino generada con vapores de NH\index{amoniaco}$_3$ (\textit{Biopack}) \cite{Soler-Illia2012,Soler-Illia2011}. El armado experimental fue igual que el descripto para el método ácido.

				\item \textit{Prolongado a \SI{130}{\celsius}.} Esta estrategia de síntesis involucró dejar las muestras en estufa a \SI{130}{\celsius} durante 7 días con el objetivo de promover la condensación\index{condensación} del óxido.

				\item \textit{Alto vacío.} Este tratamiento consiste en dejar las muestras en una cámara de alto vacío\index{alto@alto vacío} a \SI{1e-5}{\milli\bar} y \SI{130}{\celsius} durante 7 días. Para calentar y llegar al vacío necesario se utilizó la cámara de una soldadura de obleas (\textit{EVG 501 Manual Wafer Bonding System}) la cual fue evacuada por una bomba mecánica y una turbomolecular secuencialmente.

				\end{itemize}
					
		En los casos donde fue necesario realizar la extracción\index{extracción} del surfactante\index{surfactante} sin calcinar, las muestras fueron sometidas a un reflujo\index{reflujo} de 2-propanol\index{propanol@2-propanol} a punto de ebullición (\textit{Biopack}) durante \SI{15}{\minute}. Luego se enjuagaron con H$_2$O acidificada con HCl\index{acido@ácido!clohídrico} a $\text{pH}=2$. El siguiente diagrama de flujo resume y agrupa todos los tratamientos realizados sobre las \pdm, desde el depósito\index{depósito} hasta la extracción\index{extracción} del surfactante\index{surfactante}, incluyendo la nomenclatura utilizada a lo largo de la tesis.
		
				\begin{figure}[ht!]
						  \begin{center}
						  \includegraphics[width=\textwidth]{Esquemas/Resumen_extraccion.pdf}
						  \caption[Tratamientos pos-depósito de \pdm]{Etapas de estabilización y diferentes tratamientos posdepósito utilizados para condensar y extraer el surfactante\index{surfactante} en las \pdm, tanto de óxidos puros como mixtos.}
						  \label{esq:peliculas_meso_tratamientos}
						  \end{center}
						  \end{figure}
    	
	\subsection{Funcionalización de las \pdm}\label{sub:funcionalizaci_n_de_las_pdm}
	
		Se realizaron una serie de reacciones químicas con el propósito de funcionalizar\index{funcionalización} las paredes de los poros de la \pdm\space y con el objetivo de conferir a cada electrodo, de un mismo sensor, alguna característica distintiva, p. ej. aumentar carácter el hidrofóbic\index{hidrofóbico}o o hidrofílico, cambiar el estado de la carga eléctrica superficial, etc. Las funcionalizaciones se llevaron a cabo sobre películas mixtas Si$_{0.9}$Zr$_{0.1}$O$_2$ sintetizadas por el método de alto vacío\index{alto@alto vacío}. Esta elección está fundamentada en los buenos resultados obtenidos para estos sistemas, los cuales se discutirán a lo largo de los capítulos \ref{chap:Mesoporosos} y \ref{chap:Electroquimica}.

		Las moléculas elegidas para incorporar a las \pdmZ\space se exponen en la tabla \ref{tabla:funciones}. Allí se indica la denominación, estructura, concentración y condiciones bajo las cuales se realizaron las reacciones así como la nomenclatura de los sistemas resultantes.

			\begin{table}[b!]
  		   \caption[Moléculas funcionalizantes]{Moléculas utilizadas para funcionalizar\index{funcionalización} los electrodos, condiciones experimentales y nomenclatura.} 
  		   \begin{tabular}{>{\raggedright\arraybackslash}m{2cm}>{\centering\arraybackslash}m{4.2cm}>{\centering\arraybackslash}m{2.25cm}>{\centering\arraybackslash}m{2cm}} 
  		   \toprule
		   Nombre & Estructura & Condiciones & Nomenclatura \\ \midrule
		   
		   dihexadecilfosfato\index{dihexadecilfosfato}   DHDP\index{dihexadecilfosfato}& \includegraphics[scale=0.55]{Esquemas/dhdp.pdf}&\SI{3}{\milli\Molar}\textbar\SI{80}{\celsius}\textbar\SI{20}{h}&\pdmZ$^P_3$\\ \midrule \\[-3mm]
		   
		   \multirow{2}{*}{\shortstack[l]{3-aminopropil\\trietoxisilano\\APTES\index{aminopropil@3-aminopropil trietoxisilano}}}   & \multirow{2}{*}{\includegraphics[scale=0.55]{Esquemas/aptes.pdf}}	  & \SI{1}{\milli\Molar}\textbar\SI{80}{\celsius}\textbar\SI{20}{h}&\pdmZ$^N_1$\\ \cmidrule{3-4}
		   	 & &\hspace*{-1.8mm}\SI{10}{\milli\Molar}\textbar\SI{80}{\celsius}\textbar\SI{20}{h}&\pdmZ$^N_{10}$\\ \\[-3mm] \bottomrule       	   
    	   \end{tabular}
    	   \label{tabla:funciones}
   		   \end{table}	

		Las funcionalizaciones se llevaron a cabo en una región estrecha del sensor, de forma de abarcar un sólo electrodo para cada reacción. Con el fin de delimitar la zona de reacción se utilizó un recipiente de polipropileno\index{polipropileno} con un sello contra la \pdm, evitando derrames laterales. Las condiciones experimentales fueron adaptaciones de las utilizadas por Angelomé\index{Angelomé}\cite{Angelome2008} y Calvo\index{Calvo}\cite{Calvo20210}, utilizando como solvente tetrahidrofurano(\textit{Sintorgan}) para el DHDP\index{dihexadecilfosfato}(\textit{Fluka}) y tolueno\index{tolueno}(\textit{DORWIL}) para el APTES\index{aminopropil@3-aminopropil trietoxisilano}(\textit{Aldrich}). Se normalizó el tiempo y la temperatura de ambas reacciones con el propósito de poder funcionalizar\index{funcionalización} dos o más electrodos en simultáneo. Una vez terminada la reacción se realizó un enjuague con abundante etanol \index{etanol}seguido de secado con flujo de aire o N$_2$. 

	\subsection{Espectroscopía IR}\label{sec:IR}

		La región infrarroja (IR) del  espectro electromagnético\index{espectro electromagnético} puede ser divido en tres zonas, según su número de onda: IR lejano (400 a \SI{10}{\cm^{-1}}), IR medio (4000 a \SI{400}{\cm^{-1}}), e IR cercano (14000 a \SI{4000}{\cm^{-1}}). El infrarrojo medio puede ser usado para estudiar las vibraciones\index{vibración} fundamentales y la estructura roto-vibracional; brinda información acerca de los grupos funcionales orgánicos y la estructura inorgánica a través del análisis de las vibraciones\index{vibración} moleculares.\cite{Atkins2006,Barrow1962,Stuart2004} 
		
		A lo largo de este trabajo se usó esta porción del espectro IR para analizar los resultados de la extracción\index{extracción} de surfactante\index{surfactante} y estructura inorgánica de las \pdm. Las mediciones se llevaron a cabo en la Unidad Técnica de Nanomateriales del Ce\index{cerio}ntro de Investigaciones en Procesos Superficiales del INTI\index{INTI} (INTI-CIEPS). El equipo es un \textit{Thermo Scientific Nicolet 6700 FTIR} que cuenta con un microscopio para poder focalizar el haz en un área de aproximadamente \SI{0.5x0.5}{\mm}. Se utilizó la técnica de espectroscopía\index{espectroscopía} infrarroja por transformada de Fourier (FTIR) tanto en trasmisión como en reflexión y los espectros fueron tomados con el detector MCT/B (\textit{Wide Band mercury cadmium telluride}) que es de 4 a 10 veces más sensible que los detectores estándar para equipos de espectroscopía\index{espectroscopía} FTIR.\cite{Nicholet2007} Las películas destinada a ser caracterizadas por FTIR\index{FTIR} fueron depositadas sobre Si, por ser éste trasparente en una amplia región del IR medio.

	\subsection{Ángulo de contacto}

		La medición del ángulo de contacto\index{angulo@ángulo de contacto} (AC) de un líquido sobre una superficie\index{superficie} permite evaluar la energía\index{energía} superficial ($\gamma$) entre ambos. La teoría vincula el AC con $\gamma$ mediante el análisis del equilibrio químico de tres fases: la fase líquida de la gota, la fase gaseosa del aire y la sólida del sustrato. El valor del AC depende principalmente de la relación que existe entre las fuerzas adhesivas entre el líquido y el sólido y las fuerzas cohesivas del líquido. Se puede, así, cuantificar la mojabilidad\index{mojabilidad} de un líquido en aire, en una determinada superficie\index{superficie}.\cite{findenegg1997} Tomando dos caso extremos, cuando la superficie\index{superficie} interactúa fuertemente con el líquido y se moja, el AC se aproxima a $0^{\circ}$; en cambio si la superficie\index{superficie} y el líquido se repelen, el AC tenderá a $180^{\circ}$. En términos de equilibrio termodinámico, el potencial químico de las tres fases  debe ser igual. Quién dió la primera descripción en términos de energías interfasiales fue Young\index{Young} en 1805\cite{young1805}, donde postuló que la energía\index{energía} superficial líquido-vapor ($\gamma$) por el coseno del angulo de contacto($\theta$) es igual a diferencia de las energías superficiales sólido-líquido $\gamma_{_{SL}}$ y sólido-vapor $\gamma_{_{SV}}$s. Tal relación se la conoce como ecuación de Young\index{Young} (ecuación \ref{eq:young}).

			\begin{equation}
				\gamma\, cos(\theta) = \gamma_{_{SL}} - \gamma_{_{SV}}
				\label{eq:young} 
				\end{equation}

		En este trabajo se utilizaron las medidas de AC entre agua\index{agua} y las superficie\index{superficie}s de las \pdm, para calcular la distribución de los tamaños de poro\index{poro} y cuello\index{cuello de poro} de los sistemas porosos aplicando la ecuación de Kelvin\index{Kelvin}.\cite{Boissiere2005} En la próxima sección se explica en detalle como se estiman dichas distribuciones.
		Las medidas de AC se realizaron en la Gerencia Química, CAC-CNEA con un equipo \textit{Ramé-Hart 290} y los datos fueron recogido con el software \textit{DROPImage}.

	\subsection{Elipsometría}\label{sec:elipso}

		La elipsometría es una técnica de análisis óptico que se basa en el cambio del estado de polarización de la luz\index{polarización!de la luz} que incide sobre una o más películas delgadas soportadas sobre un material reflectivo. Dicho análisis es no destructivo y es útil para la determinación de espesor\index{espesor}es y constantes ópticas (índices de refracción y constante de absorción) de dichas películas.\cite{TompkinsHarlandG.1999,Rothen1945} El parámetro medido es el cociente complejo $\rho$ de la amplitud de la reflexión de los componentes paralelo ($r_p$) y perpendicular ($r_s$) del haz polarizado incidente. Este cociente se expresa como función de los parámetros elipsométricos $\Delta(\lambda\index{longitud de onda})$ y $\Psi(\lambda\index{longitud de onda})$. 

		Debido a que las ecuaciones involucradas en el proceso no poseen resolución analítica, es necesario recurrir a modelos que describan el material para obtener las propiedades de interés, es decir, el índice de refracción\index{indice@índice de refracción} real, $n(\lambda\index{longitud de onda})$, el espesor\index{espesor}, $t$, y el coeficiente de absorción\index{coeficiente de absorción} $k(\lambda\index{longitud de onda})$. Mediante un ajuste iterativo por cuadrados mínimos de $\Delta(\lambda\index{longitud de onda})$ y $\Psi(\lambda\index{longitud de onda})$ (para el cual se proponen valores iniciales para $n$, $k$ y $t$ de la muestra) se minimiza la diferencia entre el modelo y los datos experimentales. De esta forma se extrae el espesor\index{espesor} y el índice de refracción\index{indice@índice de refracción}. \cite{TompkinsHarlandG.1999}

		Cuando se adapta al equipo una cámara, en la que se puede variar la presión parcial de H$_2$O, es posible medir los cambios de las propiedades ópticas (p. ej. espesor\index{espesor} e índice de refracción\index{indice@índice de refracción}) de las \pdm\space durante la adsorción\index{adsorción} y desorción\index{desorción} de H$_2$O. A esta técnica se la conoce con el nombre de porosimetría elipsométrica ambiental (PEA) \cite{Boissiere2005}. La figura \ref{fig:elipso} muestra un esquema de los principales componentes de un elipsómetro con cámara de humedad controlada.

			  \begin{figure}[h]
				\begin{center}
				\includegraphics[width=\textwidth]{Esquemas/Elipso.pdf}
			  	\caption[Esquema de la técncia de elipsoporosimetría\index{elipsoporosimetría ambiental} ambiental]{Esquema de los componentes principales del equipos de elipsometría utilizado para determinar las constantes elipsométricas, $\Delta(\lambda\index{longitud de onda})$ y $\Psi(\lambda\index{longitud de onda})$, de las cuales se obtienen el espesor\index{espesor}, indice de refracción, coeficiente de absorción\index{coeficiente de absorción},  distribución y tamaño de poro\index{distribución!de poro}s y cuello\index{cuello de poro}s de las \pdm.}
			  	\label{fig:elipso}
			  	\end{center}
			  	\end{figure}
		
		El volumen de vapor adsorbido dentro de los poros se determina a partir de la variación de $n(\lambda\index{longitud de onda})$ utilizando aproximaciones de medio efectivo como la de Bruggeman\index{Bruggeman}\cite{Bruggeman1935} o la de Maxwell-Garnett\index{Maxwell-Garnett}\cite{Garnett1906} que son simplificaciones de la ecuación Lorentz-Lorentz\index{Lorentz-Lorentz}  general\cite{TompkinsHarlandG.1999}.
		La aproximación de Bruggeman\index{Bruggeman} considera dos componentes mezclados al azar cuyas fracciones en volumen ($f_i$) y constante dieléctrica\index{constante dieléctrica} ($\mathcal{E}_i$) deben cumplir con la ecuación \ref{eq:bruggeman}, donde $\mathcal{E}_e$ es la constante dieléctrica\index{constante dieléctrica} del material compuesto. 
				
				\begin{equation}
				 f_1\left(\frac{\mathcal{E}_1-\mathcal{E}_e}{\mathcal{E}_1+2\mathcal{E}_e}\right)+
				 f_2\left(\frac{\mathcal{E}_2-\mathcal{E}_e}{\mathcal{E}_2+2\mathcal{E}_e}\right)=0
			     \label{eq:bruggeman}
				 \end{equation}
		
		\vspace*{2mm}El índice de refracción\index{indice@índice de refracción} se define según la ecuación \ref{eq:indice} donde $\mu$ es permeabilidad electromagnética relativa. Para la mayoría de los materiales, y cerca del rango visible, $\mu$ es muy cercana a la unidad, por lo tanto es común aproximar $\mathcal{E}_e=n_e^2$.
		
						\begin{equation}
					 	   n=\sqrt{\mathcal{E}\mu}
					 	   \label{eq:indice}
						\end{equation}
		
		\vspace*{2mm}La aproximación de Maxwell-Garnett\index{Maxwell-Garnett} considera al material compuesto por al menos dos especies, la matriz y la inclusión. En el caso de los óxidos porosos, la matriz es el óxido y el aire o el surfactante\index{surfactante} la inclusión. Se deben satisfacer en este caso las ecuaciones \ref{eq:maxwall1} y \ref{eq:maxwall2}.
				
							\begin{equation}
					 		   	 f_1\left(\frac{\mathcal{E}_1-\mathcal{E}_2}{\mathcal{E}_1+1\mathcal{E}_2}\right)-
					 		   	 \left(\frac{\mathcal{E}_e-\mathcal{E}_2}{\mathcal{E}_e+2\mathcal{E}_2}\right)=0
					 		     \label{eq:maxwall1}
								\end{equation}
						
								\begin{equation}
					 		   	 f_2\left(\frac{\mathcal{E}_2-\mathcal{E}_1}{\mathcal{E}_1+2\mathcal{E}_1}\right)-
					 		   	 \left(\frac{\mathcal{E}_e-\mathcal{E}_1}{\mathcal{E}_e+2\mathcal{E}_1}\right)=0
					 		     \label{eq:maxwall2}
								\end{equation}
		
		\vspace*{2mm}El volumen total ocupado por los poros, V$_p$, y el volumen de agua\index{agua} adsorbido para cada HR, V$_{ads}$, se calcularon aplicando indistintamente dichas aproximaciones (ya que para \pdm\space dan resultados equivalentes) a las constantes dieléctricas\index{constante dieléctrica} (o índices de refracción) medidas del film seco y lleno de agua\index{agua}, luego de la condensación\index{condensación} capilar.\cite{Angelome2008,Fuertes2009,Nano-compuestas2013}. Se construye de esta forma una isoterma\index{isoterma} de adsorción\index{adsorción}/desorción de H$_2$O en función del índice de refracción\index{indice@índice de refracción} (o volumen poroso) de las películas porosas. Los distintos tipos de isoterma\index{isoterma}s para la adsorción\index{adsorción} sobre materiales absorbentes porosos fue clasificada por la IUPAC\index{IUPAC} en ocho grupos (Ia, Ib, II, III, IVa, IVb, V y VI) y seis ciclos de histéresis\index{histéresis} para los tipos IVa y V (H1, H2a, H2b, H3, H4 y H5). \cite{Thommes2015}

		La figura \ref{fig:pea_ej}a muestra un resultado típico para adsorción\index{adsorción} de agua\index{agua} en una película de óxido de silicio\index{silicio!oxido de}\index{silicio} estructurada con CTAB medida por PEA. La curva resultante corresponde a una isoterma\index{isoterma} tipo IVa/H2b. El ciclo de histéresis\index{histéresis} indica la presencia de mesoporos, cuyo llenado se produce por condensación\index{condensación} capilar.\cite{Gregg1967} La gran mayoría de las isoterma\index{isoterma}s obtenidas fueron de este tipo (IVa con ciclo H2b) y, consecuentemente, son las más dicutidas y analizadas a lo largo de este trabajo. 

			\begin{figure}[!ht]
		     	  		\begin{subfigure}[t]{0.491\textwidth}
		     	  		\includegraphics[width=\textwidth]{Graficos/CTAB_M2_Modelo_isoterma.pdf}
						\label{fig:pea_ej1}
						\end{subfigure}
						\begin{subfigure}[t]{0.495\textwidth}
		     	  		\includegraphics[width=\textwidth]{Graficos/CTAB_M2_Modelo_distribucion.pdf}
						\label{fig:pea_ej2}
						\end{subfigure}
						\vspace*{-0.6cm}
						\caption[Isoterma de adsorción\index{adsorción}/desorción tipo IVa, H2b.]{(a) Isoterma de adsorción\index{adsorción}/desorción de agua\index{agua} para un sistema mesoporosos de SiO$_2$ estructurado con CTAB. La misma se clasifica según la IUPAC\index{IUPAC} como tipo IVa con ciclo de histéresis\index{histéresis} H2b; (b) Distribución de tamaño de poro\index{poro} y cuello\index{cuello de poro}.}
						\label{fig:pea_ej}
						\end{figure}			
		
		Se puede calcular a partir de las ramas de adsorción\index{adsorción} y desorción\index{desorción} de la isoterma\index{isoterma} la distribución para los diámetros de poros y cuello\index{cuello de poro}s respectivamente. Los resultados que se obtienen son similares al ejemplo de la figura \ref{fig:pea_ej}b. Para realizar este cálculo debemos recurrir a la ecuación de Kelvin\index{Kelvin} (ec. \ref{eq:kelvin}), que describe el equilibrio líquido-vapor considerando el tamaño de la esfera y la energía\index{energía} superficial, donde R es la constante de los gases, T es la temperatura, P es la presión de vapor\index{presión de vapor}, P$_s$ es la presión de vapor\index{presión de vapor} de saturación, $\gamma$ es la tensión superficial\index{tensión superficial} del líquido, V$_m$ es el volumen molar del líquido y $\theta$ es el ángulo de contacto\index{angulo@ángulo de contacto} sólido-líquido. \cite{Baklanov2000,Boissiere2005,Sing1985} Para poros esféricos la relación $\partial S/ \partial dV$ es proporcional al radio de la esfera, llamado radio de Kelvin\index{Kelvin}.\cite{FernandezPrini2005}
		
			\begin{equation}
			  	 \ln \left(\frac{P}{P_s}\right)=\frac{2\gamma V_m}{RT} \cos{\theta}\frac{\partial S}{\partial V}
			     \label{eq:kelvin}
			 	 \end{equation}					
	
		Todas las medidas fueron tomadas en la Gerencia Química, CAC-CNEA con un elipsómetro espectroscópico marca \textit{SOPRA}, modelo \textit{GES 5E}. El rango espectral del equipo va de 190 a \SI{900}{\nm}, posee una cámara para realizar las mediciones en condiciones de humedad controlada y también permite configuración en modo \textit{micro-spot} que permite reducir el área de medición a una región de aproximadamente \SI{1}{\mm^2}. El modelado de los parámetros se hizo mediante el \textit{software Winelli II} también de la marca \textit{SOPRA}.
			
\section{Microfabricación de los electrodos}
		
	 En esta sección se dará cuenta de los detalles experimentales para la fabricación de los electrodos, los cuales son una parte fundamental de los sensores. Es en la superficie\index{superficie} de los electrodos donde tienen lugar las reacciones de óxido/reducción de los analitos de interés, y donde se depositaron las diferentes películas delgadas mesoporosas. Por estos motivos resulta fundamental contar con un diseño funcional y compacto y, además, controlar los aspectos superficiales tales como la rugosidad\index{rugosidad}, control de impurezas, espesor\index{espesor}, y funcionalización\index{funcionalización} en los casos que sea necesario.

	 Los electrodos fueron enteramente diseñados y fabricados en los laboratorios del CMNB-INTI. 
		
	 Las herramientas y técnicas empleadas para la fabricación son propias del sector de la microelectrónica\index{microelectrónica}: herramientas de \textit{software}\index{software@\textit{software}} tipo CAD, fotolitografía\index{fotolitografía}, pulverización catódica\index{pulverización catódica}, grabado por vía húmeda, \textit{lift-off}\index{lift@\textit{lift-off}}, corte y encapsulado, etc.\cite{Franssila2004,Jaeger2001} Cada uno de estos procesos y metodologías se explicarán en las secciones subsiguientes. 

	 El flujo general de trabajo para la transferencia de los diseños\index{transferencia!de los diseños} en una o más capas se presenta en la figura \ref{esq:micro}.
			
			 \begin{figure}[t!]
			  \begin{center}
			  \includegraphics[width=0.85\textwidth]{Esquemas/Resumen_micro.pdf}
			  \caption[Esquema para la transferencia de los diseños\index{transferencia!de los diseños}]{Diagrama general para la transferencia de los diseños\index{transferencia!de los diseños} y fabricación  de una o más capas. Este esquema contempla el uso de las técnicas de \textit{lift-off }o grabado según se requiera dependiendo de las características de los materiales empleados para esa capa.}
			  \label{esq:micro}
			  \end{center}
			  \end{figure}

	\subsection{Diseño e impresión de las máscara\index{máscara}s}\label{sec:impresion_mascaras}

		El primer paso necesario en la fabricación de los sensores\index{sensor} es el diseño. Como todo diseño en microelectrónica\index{microelectrónica}, se diagramó en función de las tecnologías disponibles, de la calidad de las máscara\index{máscara}s y de la aplicación final en la cual se emplearán. Todos estos aspectos ya fueron expuestos en el capítulo \ref{chap:Introduccion}, por lo que aquí nos remitiremos a describir los detalles técnicos.

		Los diseños fueron realizados para obleas de \SI{10}{\cm} de diámetro. El primer diseño se mandó a imprimir en filmina\index{filmina} de \SI{13x13}{\cm} en una filmadora de películas \textit{Agfa Accuset 1000}, a una resolución de \SI{3600}{dpi}, perteneciente a la firma $Imacrom$. Esto ha permitido obtener resoluciones de línea de \SI{50}{\um}, muy por encima de la resolución de la tecnología de la cual disponemos (transferencia por UV, $\lambda\index{longitud de onda}=365nm$). El segundo diseño, más completo e integrado, también fue diagramado para obleas de \SI{10}{\cm} de diámetro. Éste contempló la integración del contraelectrodo\index{electrodo!contraelectrodo} y el electrodo de referencia\index{electrodo!de referencia}, además de incluir 6 electrodos de trabajo\index{electrodo!de trabajo}. Las máscara\index{máscara}s correspondientes a éste diseño se mandaron a imprimir en filmina\index{filmina}s de \SI{13x13}{\cm} a la empresa \textit{International Phototool Company} a una resolución de \SI{48000}{dpi}, logrando mejor resolución y líneas más definidas que en el primer diseño, hasta de \SI{7}{\um}. Todos los diseños se llevaron a cabo con el \textit{software CAD electric}\index{software@\textit{software}\index{software@\textit{software}}!\textit{electric}} (\url{http://www.staticfreesoft.com/productsFree.html}) de licencia pública general de GNU, \url{https://www.gnu.org/licenses/gpl.html}.

		En las imágenes de la figura \ref{fig:mask_ejemplos} se exponen algunos de los muchos diseños con los cuales se fabricaron y se probaron funcionalmente los multisensor\index{multisensor}es. En el capítulo \ref{chap:Microfabricacion} se expondrán las consideraciones que se tuvieron en cuenta a la hora de confeccionar dichos diseños y se discutirán los resultados que se obtuvieron con ellos.

			\begin{figure}[t!]
					  \begin{center}
			  		  \includegraphics[width=\textwidth]{Imagenes/disenos-capitulo2.pdf}
			  		  \vspace*{-7mm}
			  		  \caption{Algunos ejemplos de los diseños de los multisensor\index{multisensor}es que se utilizaron a lo largo de este trabajo de tesis. Cada diseño corresponde a un dispositivo que forma parte de un lote de fabricación de al menos 40 unidades.}
			  		  \label{fig:mask_ejemplos}
			  		  \end{center}
			  		  \end{figure}
				
	\subsection{Limpieza de los sustratos}\label{sec:limpieza}
			
			Una vez terminado el diseño, comienza la etapa de transferencia del mismo. El primer paso es la limpieza de los sustratos para evitar problemas de adherencia\index{adherencia} y eliminar impurezas superficiales adsorbidas. 

			\begin{table}[b!]
					  \caption[Soluciones para la limpieza de los sustratos]{Soluciones utilizadas para hacer la limpieza antes de realizar cualquier proceso de fotolitografía\index{fotolitografía} o pulverización catódica\index{pulverización catódica}.\cite{Franssila2004,Kern1990}}
			  		  \begin{tabular}{>{\raggedright\arraybackslash}m{1.02cm}>{\centering\arraybackslash}m{2.8cm}>{\centering\arraybackslash}m{1.9cm}>{\centering\arraybackslash}m{1.9cm}>{\raggedright\arraybackslash}m{2.4cm}} 
			  		  \toprule
					  Nombre  & Composición &  Proporciones & Condiciones & Blanco \\ \midrule
			      	  KOH$^*$ & KOH:H$_2$O 	&    40\%p/v    &  \SI{25}{\celsius}/\SI{10}{\minute}  &  residuos orgánicos \\  \midrule
			      	  SC1$^\dagger$ &	H$_2$O:H$_2$O$_2$:NH\index{amoniaco}$_4$OH & 5:1:1 & \SI{80}{\celsius}/\SI{10}{\minute} & residuos orgánicos  \\ \midrule
			      	  SC2 &	H$_2$O:H$_2$O$_2$:HCl\index{acido@ácido!clohídrico} & 6:1:1 & \SI{80}{\celsius}/\SI{10}{\minute}   &  residuos iónico\index{iónico}s y metálicos \\ \midrule
			      	  HF\index{acido@ácido!fluohídrico}  &	H$_2$O:HF & 50:1 & \SI{25}{\celsius}/\SI{2}{\minute} & óxido de silicio\index{silicio!oxido de}\index{silicio} \\ \midrule
			      	  iPOH    &	  (CH$_3)_2$CHOH &  puro$^\mathsection$      &  enjuague & residuos grasos \\ \midrule
			      	  H$_2$O & H$_2$O desionizada & puro$^\ddagger$  &  enjuague  & desorción\index{desorción} de partículas \\ \midrule
			      	  Piraña &  H$_2$SO$_4$:H$_2$O$_2$ & 2:1 & \SI{25}{\celsius}/\SI{10}{\minute}  & residuos orgánicos  \\
			      	  \bottomrule
			    	  \end{tabular}
			    	  \footnotesize{$^*$}No apta para silicio, reacciona formando Si(OH)$_4$ y liberando H$_2$. \\
				      \footnotesize{$^\dagger$}Crece una capa de SiO$_2$ de 10 a \SI{15}{\angstrom} de espesor\index{espesor}. \\
				      \footnotesize{$^\mathsection$}Grado analítico o superior. \\
			    	  \footnotesize{$^\ddagger$}Resistividad de \SI{18}{\mega\ohm\per\cm} o mayor.
			    	  \label{tabla:limpieza}
			   		  \end{table}
			
							
			La tabla \ref{tabla:limpieza} resume cuales fueron las soluciones utilizadas para limpieza, su composición y cuál es la finalidad de cada una. Al\space finalizar cada etapa de limpieza siempre se hace un lavado con H$_2$O DI seguido de un secado con aire o N$_2$. 

			El porqué de los materiales elegidos para usar de sustratos ya fueron discutidos en el capítulo \ref{chap:Introduccion}, aquí solo se mencionan los protocolos de limpieza\cite{Franssila2004,Kern1990} utilizados para cada uno de ellos:

				\begin{itemize}
					\item{Vidrio: KOH}
					\item{Silicio: SC1, SC2, HF\index{acido@ácido!fluohídrico} o piraña\index{piraña} según el caso}
					\item{Sustratos poliméricos: ipOH}
				\end{itemize}

    \subsection{Transferencia de los diseños por fotolitografía\index{fotolitografía}}\label{sec:fotolito}

		La transferencia de los diseños\index{transferencia!de los diseños} se realizó por fotolitografía\index{fotolitografía}, técnica que también se conoce con los nombres de litografía\index{litografía} óptica o litografía\index{litografía} ultravioleta\index{UV} (UV). La técnica consiste en depositar una resina fotosensible sobre un sustrato, irradiar con luz UV\index{UV} de $\lambda\index{longitud de onda}\!=$\SI{365}{nm} a través de una máscara\index{máscara} y por último revelar la fotorresina\index{fotorresina}. Dependiendo de si la misma es negativa, positiva o de doble exposición, se disolverá la parte expuesta (positiva) o la no expuesta a la luz (negativa). \cite{Jaeger2001,Franssila2004,Mack2007,Mack2006}
				

			  \begin{figure}[b!]
			  \begin{center}
			  \includegraphics[width=0.70\textwidth]{Esquemas/fotolito.pdf}
			  \caption[Esquema fotolitografía\index{fotolitografía}]{Proceso de fotolitografía\index{fotolitografía} para una resina de doble exposición\index{resina!de doble exposición}. 1) Depósito de la resina, 2) calentamiento suave, mejora la adherencia\index{adherencia} y evapora solventes, 3) 1$^a$ exposición, 4) calentamiento para invertir la polaridad de la resina, 5) 2$^a$ exposición sin máscara\index{máscara}, 6) revelado, nótese el perfil invertido, especialmente útil para aplicar en procesos de\textit{ lift-off}.}
			  \label{esq:fotolito}
			  \end{center}
			  \end{figure}	

		Antes de depositar la fotorresina\index{fotorresina} se calienta el sustrato\index{sustrato} hasta \SI{120}{\celsius} con el objetivo de desorber H$_2$O. Los depósitos de las resinas se realizaron mediante \textit{spin-coating}\index{spin@\textit{spin-coating}} con el equipo descrito en la sección \ref{sec:deposito_pdm}, pág. \pageref{sec:deposito_pdm}. Para cubrir una oblea\index{oblea} completa de \SI{10}{\cm} de diámetro se necesitan colocar un mínimo de \SI{5}{\ml} de fotorresina\index{fotorresina} \textit{TI35E image reversal} de la marca \textit{Microchemicals}, la cual es de doble exposición, especialmente elegida por formar un perfil negativo, particularmente útil para el proceso \textit{lift-off}\index{lift@\textit{lift-off}} el cúal es tratado en detalle en la sección \ref{sec:liffff}.\cite{MicrochemicalsTeam2009} 
	
		El esquema de una típica transferencia con resina de doble exposición\index{resina!de doble exposición} se ilustra en la figura \ref{esq:fotolito}. El proceso comienza con el depósito\index{depósito} de la resina mediante \textit{spin-coating}\index{spin@\textit{spin-coating}} a una velocidad final de \SI{4000}{\minute^{-1}} durante \SI{40}{\second} y con una aceleración de \SI{400}{\minute^{-1}.\second^{-1}}. Bajo estas condiciones se obtuvo un espesor\index{espesor} final de aproximadamente \SI{4}{\um}. Luego se realizó un calentamiento durante \SI{2}{\minute} a \SI{95}{\celsius} para evaporar el exceso de solvente y promover la adherencia\index{adherencia} de la resina al sustrato. Seguidamente se cargó el sustrato\index{sustrato} y la máscara\index{máscara} en la alineadora\index{alineadora} de máscara\index{máscara}s (\textit{EVG 620}, figura \ref{fig:alineadora}), la cual cuenta con un microscopio incorporado para hacer la alineación máscara\index{máscara}/sustrato y una lámpara de Hg para el sistema de irradiación UV\index{UV} ($\lambda\index{longitud de onda}=$\SI{365}{\nm}). 
				
		Después de alinear, se realizó la primera exposición con una densidad de energía\index{energía!densidad de}\index{energía} de \SI{140}{mJ.\cm^{-2}} y se dejó reposar \SI{10}{\minute} para dar tiempo a la difusión\index{difusión} de N$_2$ liberado durante la reacción. Se realizó el calentamiento necesario para invertir el perfil (las zonas expuestas polimerizan volviéndose inerte al solvente) de la resina a una temperatura de \SI{120}{\celsius} durante \SI{2}{\minute}  y se expuso por segunda vez a una densidad de energía\index{energía!densidad de}\index{energía} de \SI{540}{mJ.cm^{-2}}, esta vez sin máscara\index{máscara}. En esta segunda exposición las partes polimerizadas no se afectan, mientras las no expuestas en la primera iluminación se vuelven solubles en el medio revelador. Para finalizar, se hizo el revelado\index{revelado} sumergiendo la oblea\index{oblea} en un cristalizador con una solución de revelador\index{revelador} \textit{AZ General} (\textit{Microchemicals}) y H$_2$O 1:1. La evolución del revelado\index{revelado} se siguió mediante microscopía\index{microscopía} óptica\index{microscopía!óptica} y se determinó el tiempo óptimo de inmersión en aproximadamente unos \SI{7}{\minute}, dependiendo del espesor\index{espesor} de la fotoresina. De esta forma los diseños quedaron completamente transferidos.
				
			  \begin{figure}[h!]
			  \begin{center}
			  \includegraphics[width=\textwidth]{Imagenes/alineadora.jpg}
			  \caption[Alineadora de máscara\index{máscara}s]{Alineadora de máscara\index{máscara}s \textit{EVG 620} semiautomática de doble cara, con lámpara de Hg de \SI{350}{W}  y capacidad para obleas de hasta \SI{150}{\mm} .}
			  \label{fig:alineadora}
			  \end{center}
			  \end{figure}

	\subsection{Depósito de películas delgadas metálicas}\label{sec:sputt}

			En esta sección se describe el proceso de fabricación de las películas delgadas de Au, cuya función es ser usadas como electrodos en los sensores. Las mismas se depositaron utilizando técnica de pulverización catódica\index{pulverización catódica}, la cual es comúnmente conocida por su nombre en inglés, \textit{sputtering}\index{sputtering@\textit{sputtering}}\cite{sigmund1968}. Los fundamentos básicos de la técnica se discutieron en el capítulo \ref{chap:Introduccion}, pág. \pageref{sec:microfabricacion}.
							
			Los sustratos utilizados para depositar los electrodos fueron principalmente obleas de silicio\index{silicio} monocristalinas (vírgenes o fotolitografiadas) y portaobjetos de vidrio. Estos soportes fueron elegidos debido a la baja rugosidad\index{rugosidad} de su superficie\index{superficie} y por ser materiales que pueden ser sometidos a temperaturas altas, en particular \SI{350}{\celsius} que es la temperatura de calcinación\index{calcinación} para la ruta de síntesis clásica de óxidos mesoporosos. Previo a realizar el depósito, los sustratos fueron tratados con los procesos de limpieza descritos en la sección \ref{sec:limpieza}, pág. \pageref{sec:limpieza}, y una vez dentro de la cámara se realizó una limpieza por plasma\index{plasma} para promover una mayor adherencia\index{adherencia} del depósito\index{depósito} al sustrato.

			Cabe destacar que si se trabaja sobre obleas de silicio, estas tienen que estar recubiertas con una capa dieléctrica para que no haya fugas eléctricas a través del silicio. A lo largo de este tesis se utilizó indistintamente obleas que ya venían con capa aislante u obleas a las cuales se le depositó una película delgada \index{película!delgada}de SiO$_2$, también mediante pulverización catódica\index{pulverización catódica}.

			Para promover la adherencia\index{adherencia} del Au, se deposita una capa de al menos \SI{20}{\nm} de espesor\index{espesor}, indistintamente de Ti\index{titanio} o Cr. Sin ésta capa, el Au \index{oro}no adhiere sobre superficie\index{superficie}s no metálicas\cite{Hieber1976}. Una vez depositada esta capa de adherente y sin romper el vacío de la cámara del equipo, se depositan un mínimo \SI{150}{\nm} de Au, para lograr un electrodo mecánicamente robusto y con buenas propiedades de conducción eléctrica. En los casos que se depositó una capa dieléctrica de SiO$_2$ se utilizó la fuente de radiofrecuencia\index{radiofrecuencia} (RF) a potencia constante, P=\SI{400}{W}. Mientras que los depósitos de las películas metálicas se realizaron todos con la fuente de corriente directa (DC) también configurada a P=\SI{400}{W}, dejando la tensión y la corriente libre, parámetros que dependen a su vez del vacío en la cámara, de la distancia entre el cátado\index{cátodo} y el ánodo\index{anodo @ ánodo} y el caudal de argón\index{argón}. 

				\begin{table}[b!]
		  		\caption[Parámetros de depósito\index{depósito} películas metálicas]{Parámetros de depósito\index{depósito} de las distintas películas delgadas metálicas para su uso como electrodos de trabajo\index{electrodo!de trabajo}.}
		  		\begin{tabular*}{\textwidth}{l @{\extracolsep{\fill}} ccccccc} 
		  		\toprule
		    	 Depósito &\hspace*{3mm}P$_{_{\text{DC}}}$(W) & T(V)  &  I(A)   & $p$(mbar) & Q$_{Ar}$(sccm) & $\nu$(\si{\nm. min^{-1}}) \\
		    	 		\midrule
		  		 Ti\index{titanio} 	 & $400$ & $750$ & $0.53$ & \num{1.70e-3} & $5$ & $50$ \\
		  		 Cr \index{cromo}	 & $400$ & $453$ & $0.84$ & \num{1.70e-3} & $5$ & $55$ \\
		  		 Au \index{oro}	 & $400$ & $679$ & $0.56$ & \num{1.35e-3} & $5$ & $44$ \\
		    	 \bottomrule
		    	 \end{tabular*}
		   		\label{tabla:sputt1}
		   		\end{table}
		   		
		  		\begin{table}[b!]
		  		\caption[Parámetros de depósito\index{depósito} películas dieléctricas]{Parámetros de depósito\index{depósito} utilizado para el depósito\index{depósito} de SiO$_2$.}
		  		\begin{tabular*}{\textwidth}{l @{\extracolsep{\fill}} cccccc} 
		  		 		\toprule
		       	Depósito  &\hspace*{3mm} P$_{_{\text{RF}}}$(W)  &P$_{\text{ref}}$(W)  &$p$(mbar) & Q$_{Ar}$(sccm) &$\nu$(\si{\nm. min^{-1}})\\
		    	 		\midrule
		  		 SiO$_2$  & $400$ & $23$ & \num{1.23e-2} & $80$ & $1.18$ \\
		  		 Limpieza\space & 150   & 3    & \num{2.04e-3} & 10   & -      \\
		  		\bottomrule
		  		\end{tabular*}
		   		\label{tabla:sputt2}
		   		\end{table}
			
			Las condiciones óptimas de depósito\index{depósito} de cada una de las capas se detallan en las tablas \ref{tabla:sputt1} y \ref{tabla:sputt2}. La primera contienen las condiciones para las películas metálicas y la segunda para la película de SiO$_2$ y para el proceso de limpieza por plasma\index{plasma} previo a los depósitos. Manteniendo las condiciones experimentales constantes se construyó, para cada material, una curva de calibración del espesor\index{espesor} en función del tiempo de depósito. De esta forma se obtuvieron las velocidades de depósito\index{depósito} que figuran en la última columna en las tablas \ref{tabla:sputt1} y \ref{tabla:sputt2}, esencial para controlar el espesor\index{espesor} de cada películas. Las mediciones de espesor\index{espesor}es se realizaron mediante microscopía\index{microscopía} FIB\index{FIB}, técnica que se describe en la sección \ref{sec:micros}.

		  	Todos los depósitos fueron realizados en el equipo de \textit{sputtering}\index{sputtering@\textit{sputtering}} del INTI\index{INTI}-CMNB. El mismo cuenta, entre sus principales capacidades, con una fuente DC (hasta \SI{1.5}{\kW}) una fuente de RF (\SI{600}{W} a \SI{13.56}{\MHz}), posibilidad de depositar 3 materiales consecutivamente y capacidad para colocar sustratos de hasta \SI{150}{\mm} de diámetro. El mismo es de la marca \textit{Boc Edwards}. En la figura \ref{fig:sputt} se muestra el equipo y un detalle al momento de hacer los depósitos.


		   		  \begin{figure}[h!]
				  \begin{center}
				  \includegraphics[width=\textwidth]{Imagenes/sputt.jpg}
				  \caption[Equipo para depósito\index{depósito} de películas delgadas, \textit{sputtering}\index{sputtering@\textit{sputtering}}]{Foto del instrumental utilizado para realizar los depósitos bicapa Ti\index{titanio}\textbar Au \index{oro}o Cr\textbar Au.(A) El equipo \textit{BOC Edwards} completo donde se ve el gabinete de control y la cámara de vacío, (B) foto a través de la ventana al momento de realizar un deposito de Cr \index{cromo}y (C) foto a través de la ventana al momento de realizar un deposito de Au.}
				  \label{fig:sputt}
				  \end{center}
				  \end{figure}

	\subsection{Proceso de\textit{ lift-off}}\label{sec:liffff}

   	     Una vez finalizados los procesos de fotolitografía\index{fotolitografía} y pulverización de cada una de las capas es necesario remover el excedente de material. Este proceso de remoción, que se explica aquí, se conoce por su nombre en inglés \textit{lift-off}\index{lift@\textit{lift-off}}.

		 La bicapa Ti\index{titanio}\textbar Au \index{oro}o Cr\textbar Au \index{oro}se pulverizó sobre toda la superficie\index{superficie} de la oblea, tanto en las partes donde estaba el silicio\index{silicio} descubierto como en las partes donde quedó la fotorresina\index{fotorresina} sin revelar. 
		
		 De esta forma, al estar el metal sobre la resina, disolviendo ésta, se desvincula la capa Ti\index{titanio}\textbar Au \index{oro}del sustrato\index{sustrato} y queda completa la transferencia de los diseños\index{transferencia!de los diseños}. 
		 La disolución de la fotoresina se lleva a cabo en acetona\index{acetona} (\textit{Sigma}) dentro de un baño de ultrasonido \index{ultrasonido}(\textit{TESTLAB} Modelo \textit{tb02}) a \SI{22}{\kHz}. En la figura \ref{esq:liftoff} se esquematiza todo el proceso completo.\vspace*{-5mm}

		 	\begin{figure}[h!]
			  \begin{center}
			  \includegraphics[width=\textwidth]{Esquemas/liftoff.pdf}
			  \caption[Esquema del proceso de\textit{ lift-off}]{Esquema del proceso de\textit{ lift-off} en el cual se decapa la fotorresina\index{fotorresina}. 1) Fotorresina transferida en base a un diseño, 2) depósito\index{depósito} metálico, 3) disolución de la fotorresina\index{fotorresina} con un solvente adecuado, 4) diseño completamente trasferido.}
			  \label{esq:liftoff}\index{lift@\textit{lift-off}}
			  \end{center}
			  \end{figure}

	\vspace*{-4mm}\subsection{Modificación superficial}\label{sec:silanizacion}
		
		A lo largo del trabajo surgió la necesidad de mejorar la adherencia\index{adherencia} de las \pdm\space sobre los electrodos de Au\index{electrodo!de Au}. Para lograr ésto, se realizó sobre los electrodos una modificación superficial, de forma de generar puntos de anclaje para promover la adherencia\index{adherencia} del óxido de silicio\index{silicio!oxido de}\index{silicio} sobre la superficie\index{superficie} de los electrodos.
		El proceso consistió en vincular covalentemente una molécula\index{moléculas} a la superficie\index{superficie} de Au \index{oro}y, por otro lado, que ésta misma molécula\index{moléculas} sea parte estructural de las \pdm. Para lograr ésto se preparó una solución \SI{10}{\milli\Molar} de 3-mercaptopropil trimetoxisilano\index{mercaptopropil@3-mercaptopropil trimetoxisilano} (MPTMS) en tolueno\index{tolueno} (se eligió tolueno\index{tolueno} de forma de minimizar la hidrólisis\index{hidrolisis@hidrólisis}\index{hidrólisis} y condensación\index{condensación} del MPTMS) y se dejó reaccionar durante 2 horas en cristalizador cubierto con un vidrio\index{vidrio} de reloj. \cite{Goss1991,Herzog2013} Luego se realizó un enjuague con acetona\index{acetona} y se secó en flujo de N$_2$.

	\subsection{Encapsulado y corte}\label{sec:corte}

		Sobre los electrodos depositados se deposita una resina negativa\index{resina!negativa}, epoxi\index{epoxi} y fotocurable, \textit{SU8-100} de \textit{MicroChemical}\cite{MicrochemicalsTeam2009}. Dicha resina es ópticamente transparente y de alta viscosidad, lo que permite generar capas de hasta \SI{100}{\um} de espesor\index{espesor}. 

		Fue utilizada con un doble propósito: proteger mecánicamente los sensores\index{sensor} y hacer un reservorio o celda con un volumen  $V \approx$ \SI{2}{\ul}, el cual contendrá la solución con los analitos que se desean detectar.  
			
					\begin{figure}[hb!]
			 		  \begin{center}
			 		  \includegraphics[width=0.73\textwidth]{Graficos/rotacion_su8.pdf}
			 		  \caption[Parámetros de depósito\index{depósito} para la resina epoxi\index{resina!epoxi}\index{epoxi}]{Esquema de aceleración y velocidad de rotación\index{velocidad!de rotación} para el depósito\index{depósito} de la fotorresina\index{fotorresina} epoxi\index{epoxi} SU8.}
			 		  \label{fig:spin-su8}
			 		  \vspace*{-2mm}
			 		  \end{center}
			 		  \end{figure}

			 		  \begin{figure}[bh!]
			 		  \begin{center}
			 		  \includegraphics[width=\textwidth]{Imagenes/dicer.jpg}
			 		  \caption[Cortadora de obleas]{Cortadora de obleas de la marca \textit{Laser Optics}, ubicada en los laboratorios del INTI\index{INTI}-CMNB}
			 		  \label{fig:dicer}
			 		  \end{center}
			 		  \end{figure}

		Para controlar el espesor\index{espesor} mediante \textit{spin-coating}\index{spin@\textit{spin-coating}} se utilizó el esquema de rotación que se presenta en la figura \ref{fig:spin-su8}. Luego se realizó un secado para evaporar solventes a \SI{65}{\celsius} durante \SI{1}{\minute} y \SI{95}{\celsius} durante \SI{10}{\minute}. Seguidamente, se expone al UV\index{UV} a través de la máscara\index{máscara} con una densidad de energía\index{energía!densidad de}\index{energía} de \SI{680}{mJ.cm^{-2}} para activar los iniciadores de la polimerización\index{polimerización} sólo en las zonas iluminadas. Se realiza un segundo calentamiento gradual de \SI{1}{\minute} a \SI{65}{\celsius} y \SI{12}{\minute} a \SI{95}{\celsius} para incrementar el grado de polimerización\index{polimerización} y finalmente se lleva a cabo el revelado\index{revelado} (revelador para resina SU-8 de \textit{MicroChemical}), el cual requiere un tiempo de \SI{10}{\minute} para disolver completamente las partes que no fueron expuestas a la luz UV. 
		
		Para concluir la fabricación de los sensores\index{sensor} se corta la oblea\index{oblea} en cuadrados de \SI{1x1}{\cm} con el propósito de obtener así cada dispositivo individual con 6 electrodos de trabajo\index{electrodo!de trabajo} cada uno. El corte se realiza con un disco de carburo de silicio\index{silicio} de la marca \textit{Loadpoint} a con una velocidad de rotación\index{velocidad!de rotación} de \SI{44000}{\minute^{-1}} y una de avance de \SI{0.5}{\mm\per\second}. El mismo fue montado en una cortadora de obleas marca \textit{Laser Optics} ubicada en los laboratorios del INTI\index{INTI}-CMNB (ver figura \ref{fig:dicer}).

	\subsection{Espectroscopía de fotoelectrones de rayos X\index{rayos X}}

		La técnica de XPS\index{XPS} (del inglés, \textit {X-ray photoelectron spectroscopy}) es una espectroscopía\index{espectroscopía} semicuantitativa y de baja resolución espacial que habitualmente se utiliza para caracterizar la estequiometría, estado de oxidación de algún elemento en particular y la estructura electrón\index{electrón}ica de los elementos en superficie\index{superficie}.\cite{siegbahn1956,siegbahn1981}

		Se hizo uso de esta técnica para evaluar estados de oxidación del Au \index{oro}y comprobar difusión\index{difusión} de contaminantes hacia la superficie\index{superficie} de los electrodos.  Los equipos constan de diferentes componentes; una cámara de ultra alto vacío\index{alto@alto vacío} (UHV) con presiones del orden de \SI{1e-9}{mbar} para disminuir la cantidad de contaminantes superficiales y asegurar a los electrones eyectados un camino libre medio lo suficientemente grande como para alcanzar el analizador. La cámara está construida en acero inoxidable y posee ventanas de vidrio\index{vidrio} para poder observar su interior. A ella se acoplan diferentes elementos necesarios para el análisis superficial como la fuente de rayos X\index{rayos X}, el analizador de electrones, el cañón de iones, entre otros.\cite{XPS1978,Corthey2012}

		Las medidas de XPS\index{XPS} se llevaron a cabo en el Instituto de Investigaciones Fisicoquímicas Teóricas y Aplicadas (INIFTA\index{INIFTA}). Se utilizó una fuente de Mg K$\alpha$ (\textit{XR50, Specs GmbH}) y un analizador hemisferio (\textit{PHOIBOS 100, Specs GmbH}). La presión dentro de la cámara de UHV fue menor a \SI{1e-9}{mbar}. El ángulo entre la fuente de rayos X\index{rayos X} y el eje del analizador está fijado en \ang{54;44;00}. Los valores de sección eficaz de fotoionización están tabulados para esta geometría. Se realizó una calibración de la escala de energía\index{energía} de dos puntos utilizando Au \index{oro}evaporado ($E_B$ de Au$f_{7/2}$ = \SI{84}{\electronvolt}) y Cu ($E_B$ de Cu $2_{p3/2}$ = \SI{932.67}{\electronvolt}).
		
\vspace*{-0.11cm}\section{Microscopías}\label{sec:micros}
		
	 En este apartado haremos un breve resumen de los diferentes técnicas de microscopía\index{microscopía}s utilizadas durante la tesis.

	\vspace*{-0.12cm}\subsection{Microscopía óptica}

		Se utilizó microscopía\index{microscopía} óptica\index{microscopía!óptica} en modo reflexión, fundamentalmente para evaluar la superficie\index{superficie} (homogeneidad, fracturas, grietas, etc), tanto de las películas metálicas como de las mesoporosas. También para determinar la calidad de las máscara\index{máscara}s impresas y para establecer los tiempos de revelado\index{revelado} en los procesos fotolitográficos. Se hizo uso de un microscopio \textit{Olympus} modelo \textit{BX51} configurado tanto para trasmisión como para reflexión. Como fuente de luz el equipo cuenta con lámpara halógena y, en los casos necesarios, se intercaló un filtro ultravioleta\index{UV} de forma de no exponer las fotorresina\index{fotorresina}s durante la inspección y evaluación de los tiempos de revelado.
	
	\vspace*{-0.12cm}\subsection{Microscopía electrón\index{electrón}ica de barrido (MEB)}\label{sec:SEM}

		La microscopía\index{microscopía} electrón\index{electrón}ica de barrido (MEB) permitió ver y caracterizar las películas delgadas, ya sean los electrodos o las \pdm. Tamaño de poro, homogeneidad, tamaño de cristales, microfisuras y espesor\index{espesor}es son algunas las características que se pudieron evaluar con esta técnica. Además, el equipo utilizado nos permitió hacer análisis por espectroscopía\index{espectroscopía} de rayos X\index{rayos X} dispersiva en energía\index{energía} (EDS, del inglés \textit{Energy Dispersive Spectroscopy}) y tomar imágenes tanto de electrones secundarios como de electrones retrodifundidos. \cite{Goodhew2000,Watt1997}


			\begin{figure}[b!]
			 		  \begin{center}
			 		  \includegraphics[width=\textwidth]{Imagenes/sem-fib.jpg}
			 		  \caption[Microscopio de doble haz FIB\index{FIB}/SEM]{Equipo de FIB\index{FIB}/SEM utilizados para realizar las observaciones, cortes y caracterizaciones de los sensores. Consta de un microscopio de barrido electrón\index{electrón}ico de alta resolución y de una fuente de galio\index{galio} líquido para realizar, entre otras cosas, cortes en la micro y nanoescala\index{nanoescala}.}
			 		  \label{fig:sem-fib}
			 		  \end{center}
			 		  \vspace*{-2mm}
			 		  \end{figure}


		Todas las mediciones e imágenes fueron realizadas con un microscopio de doble haz de la marca \textit{FEI}, modelo \textit{Helios NanoLab 650} equipado con dos columnas, una de iones de galio\index{galio} y otra de electrones. Los iones producen cortes nanométricos y los electrones generan imágenes de alta resolución. La figura \ref{fig:sem-fib} muestra una fotografía del Laboratorio de Microscopía FIB\index{FIB} del CMNB-INTI donde se encuentra instalado el equipo. La fuente de la columna de electrones es un emisor tipo FEG (del inglés \textit{Field Emission Gun}) y como instrumental de detección cuenta con detector de electrones secundarios (SE, del inglés \textit{Secondary Electron}), de electrones retrodifundidos (BSD, del inglés \textit{back scatter detector}) y de inmersión (TLD, \textit{Thought Lens Detector}), ver esquema presentado en la figura \ref{esq:sem-fib}. 

		Se utilizaron tensiones de trabajo bajas, típicamente entre \SI{1}{\kilo\electronvolt} y \SI{5}{\kilo\electronvolt} e intensidades del orden de los \SI{25}{\pA}. La justificación de estos valores es que al acelerar los electrones con bajas tensiones, la penetración en la muestra es pobre. Si\space bien depende del tipo de material, podemos estimar en base simulaciones de trabajos en la literatura especializada que, para oro\index{oro} o silicio, la penetración con los valores de tensión citados es de unos 50 a \SI{200}{\nm} \cite{Joy1984,Shur2012,Hafner2007}. Por el otro, se utilizó un flujo de electrones también bajo (\SI{25}{\pA}), de manera de evitar el apantallamiento debido a la acumulación de carga superficial en la muestra. Todos las imágenes de MEB\index{MEB} en este trabajo incluyen las condiciones experimentales utilizadas en la barra de información situada debajo de cada una de ellas. Cuando fué posible se realizaron sobre las microscopía\index{microscopía}s transformadas rápidas de Fourier (FFT, del ingles \textit{fast Fourier transform}) para inferir la estructura del arreglo de poros. Las mediciones de diámetro de poro\index{poro} presentadas son un promedio sobre una población con $N=100$. Tanto  medidas como las FFT fueron realizadas con el software de código abierto ImageJ\index{ImageJ} (\url{https://imagej.net/})\cite{ImageJ2012}.	
					
	\subsection{Microscopía con iones de galio\index{galio} focalizados (FIB)}\label{sec:FIB}

		El bombardeo con haz focalizado de iones de galio\index{galio} (FIB, del inglés \textit{focused ion beam}) es una técnica que se utiliza fundamentalmente para el análisis de materiales, y en particular para materiales propios de la industria de la microelectrónica\index{microelectrónica}, más específicamente para análisis de microsistemas (MEMS\index{MEMS}, del inglés \textit{Micro Electro Mechanical Systems}) y circuitos integrados (IC, del inglés \textit{Integrated Circuits}). 

		Consiste en el bombardeo de la muestra con iones de galio\index{galio} para desplazar los átomos de la misma. El Ga\index{galio}$^{\circ}$ (que se almacena en un	reservorio en la cabeza de la columna) se licua y se ioniza para dar lugar a los iones de Ga\index{galio}${^+}$, los cuales mediante un sistema de lentes electromagnéticas (similar al usado en MEB\index{MEB}) son acelerados y focalizados sobre la muestra. 

		\begin{figure}[b!]
			  		  \begin{subfigure}[t]{\textwidth}
			  		  \centering\includegraphics[trim=0 0 0 0.5cm,width=0.63\textwidth]{Esquemas/sem-fib.pdf}
			  		  \end{subfigure}
			  		  \begin{subfigure}[t]{0.498\textwidth}
			  		  \includegraphics[width=\textwidth]{Imagenes/FIB-panel-b.jpg}
			  		  \end{subfigure}
			  		  \begin{subfigure}[t]{0.498\textwidth}
			  		  \includegraphics[width=\textwidth]{Imagenes/FIB-panel-c.jpg}
			  		  \end{subfigure}
			  		  \caption[Esquema de las microscopias FIB\index{FIB}/SEM]{(a) Esquema donde se muestra la disposición de las columnas de electrones y de átomos de galio\index{galio} del \textit{Helios NanoLab 650} y los principales eventos que ocurren al impactar los haces sobre la muestra.(b) Corte en sección trasversal de un dispositivo mesoporoso multicapa. (c) Imagen de alta resolución de la sección transversal enmarcada en (b).}
			  		  \label{esq:sem-fib}
			  		  \end{figure}

		El impacto de los iones Ga\index{galio}${^+}$ desplaza los átomos de la muestra generando así <<cortes>> sobre la misma. Previo al tratamiento, sin romper vacío y dentro de la cámara del FIB\index{FIB}, se deposita sobre la muestra una delgada \index{película!delgada}capa de Pt\index{platino} ($\sim$\SI{150}{\nm}). La misma actúa como protección de la muestra y para generar un borde de corte más abrupto, ya que la tasa de desplazamiento de los átomos de Pt\index{platino} con iones Ga\index{galio}${^+}$ es extremadamente baja.\cite{Giannuzzi2005,Orloff1996} 

		\pagebreak La técnica es de especial utilidad para examinar secciones transversales de muestras, calcular espesor\index{espesor}es, reconstruir volúmenes en 3D, preparar láminas para microscopía\index{microscopía} electrón\index{electrón}ica de trasmisión, entre otros tantos ejemplos. El diagrama de la figura \ref{esq:sem-fib}a representa la disposición espacial de las columnas (de iones y electrones), detectores y muestra del equipo. En en el panel (b) de la misma figura se muestra una imagen de electrones secundarios, rotada \SI{52}{\degree}, de un corte realizado para examinar la sección transversal de un dispositivo. El panel (c) es una imagen de alta resolución de la sección remarcada del panel (b). La muestra es de un\index{cristal fotónico} compuesto por capas alternadas de películas delgadas mesoporosos de Ti\index{titanio}O$_2$ y SiO$_2$, estructurados con F127 y CTAB respectivamente.\cite{Gimenez2017}

		Prácticamente en todos los casos se requiere una aceleración de iones de \SI{30}{\kilo\electronvolt} para que sea efectiva la transferencia de momento. El flujo de iones varía de acuerdo a una realción de compromiso entre tiempo y calidad de corte, siendo las corriente mas utilizadas entre \SI{2.5}{\nano\ampere} y \SI{40}{\pico\ampere} \cite{Orloff2003,Reyntjens2001}.

\section{Mediciones electroquímica\index{electroquimico}s\index{electroquimico}}\label{sec:medidas_eq}
			
			Las mediciones electroquímica\index{electroquimico}s\index{electroquimico} constituyen una parte central de este trabajo. Se hizo uso de dos tipos de técnicas voltamétricas: de corriente continua\index{voltametría!de corriente continua} y de corriente alterna, así cómo simulaciones de voltametrías por el método de elementos finitos\index{metodo de elementos finitos@método de elementos finitos}. En las próximas secciones se introducen algunos principios teóricos sobre las técnicas empleadas y cuál fue su relevancia para el presente trabajo.

			En todos los casos (tanto para los experimentales como los simulados) se utilizó la configuración típica para celda de tres electrodos.\cite{Wi2000} El electrodo de trabajo\index{electrodo!de trabajo} (ET) es donde ocurre la reacción electroquímica\index{electroquimico}, para lo cúal se le aplica un potencial variable respecto de otro electrodo de potencial fijo, denominado electrodo de referencia\index{electrodo!de referencia} (ER). La corriente resultante de la reacción química circula entre el electrodo de trabajo\index{electrodo!de trabajo} y el electrodo auxiliar o contraelectrodo\index{electrodo!contraelectrodo} (CE). La variación del potencial aplicado y la medición de la corriente se controlan mediante un potenciostato\index{potenciostato}. En la figura \ref{fig:eq-circuito} se presenta el esquema eléctrico empleado para tomar los voltagrama\index{voltagrama}s. 

				\begin{figure}[b!]
			 		  \begin{center}
			 		  \includegraphics[width=0.85\textwidth]{Esquemas/esquema-eq.pdf}
			 		  \caption[Circuito eléctrico EQ]{Esquema eléctrico correspondiente al armado experimental utilizado para realizar las voltametrías cíclicas.}
			 		  \label{fig:eq-circuito}
			 		  \end{center}
			 		  \end{figure}
		
	 \subsection{Voltametría cíclica de corriente continua\index{voltametría!de corriente continua}}\label{voltamettriaa-rruu}
	 		
	 		La voltametría cíclica \index{voltametría!cíclica}(VC) consiste en variar, de una manera cíclica, el potencial de un electrodo de trabajo\index{electrodo!de trabajo} respecto de un electrodo de referencia\index{electrodo!de referencia}. Ambos se encuentran inmersos en una solución en reposo y se mide la corriente resultante que circula por el electrodo de trabajo\index{electrodo!de trabajo}. La señal de excitación es un barrido de potencial lineal con una onda de forma triangular, la cual parte de un potencial E$_1$, evoluciona linealmente en el tiempo hasta un potencial E$_2$ para luego volver a E$_1$ (figura \ref{fig:CV_ideal}). Las velocidades de este barrido pueden variar desde menos de \SI{1}{\milli\volt.\second^{-1}} hasta cientos de volts por segundo. En este trabajo en particular se utilizaron principalemente velocidades próximas a los \SI{50}{\milli\volt.\second^{-1}}, se escogieron estas velocidades para llevar a cabo experimentos de una duración aceptable y evitar el desplazamiento de potenciales para los picos oxidación y/o reducción, debido a limitaciones en la transferencia de carga por altas velocidades de barrido. \cite{nicholson1964,Gewirth2004}

	 		Como ya se dijo anteriormente, se barre el potencial del electrodo de trabajo\index{electrodo!de trabajo} en dirección de ida y vuelta entre dos valores arbitrarios, E$_1$ y E$_2$. Al\space usar soluciones en base acuosa se debe trabajar en la región de estabilidad electroquímica\index{electroquimico} del H$_2$O, para evitar reducción u oxidación de la misma, que genera H$_2$ u O$_2$ respectivamente. En la gran mayoría de los experimentos presentados en este trabajo se trabajó a un pH\index{pH}$\sim 5$, para el cual el rango de estabilidad del agua\index{agua} es entre \SI{-0.5}{\volt} y \SI{0.7}{\volt}, usando como referencia un electrodo de calomel\index{electrodo!de calomel} saturado.\cite{wang2014} 

			En la figura \ref{fig:CV_ideal} se muestra la onda triangular de excitación aplicada y la curva obtenida para una sonda\index{sonda} electroquímica\index{electroquimico} idealmente reversible, donde se destacan los parámetros más importantes. 

	 			 \begin{figure}[h!]
			  		  \begin{subfigure}[t]{0.495\textwidth}
			  		  \includegraphics[width=\textwidth]{Graficos/onda-triangular.pdf}
			  		  \end{subfigure}
			  		  \begin{subfigure}[t]{0.495\textwidth}
			  		  \includegraphics[width=\textwidth]{Esquemas/CV_ideal.pdf}
			  		  \end{subfigure}
			  		  \caption[Voltamperometría ideal reversible]{Curva de excitación y voltagrama\index{voltagrama} típico para una especie rédox reversible. Se destaca en el voltagrama\index{voltagrama} (derecha) los típicos parámetros que se extraen de la curva, picos máximos de corriente anódica y catódica 
			  		  sus correspondientes potenciales.}
			  		  \label{fig:CV_ideal}
			  		  \end{figure}

			\vspace*{4mm}La diferencia de potencial entre los picos catódico y anódico es fruto de una combinación entre los diferentes coeficientes de difusión\index{difusión} del par rédox y el sobrepotencial necesario para transferir electrones entre la superficie\index{superficie} del electrodo y el analito. Para un par rédox reversible, esta diferencia de potencial está relacionada directamente con el número de electrones que interviene en la reacción según la ecuación \ref{ec:deltadepico}.

			\begin{equation}	
			E_{pa} - E_{pc} = 2.218 \, \frac{RT}{nF}
			\label{ec:deltadepico}
			\end{equation}

			De este modo, a \SI{25}{\celsius}, para cuplas rédox en que la transferencia involucre un único electrón\index{electrón} ($n=1$), la separación teórica debería ser aproximadamente de \SI{57}{\milli\volt}, independiente de la concentración o la velocidad de barrido\index{velocidad!de barrido}. A mayor separación de picos se tendrá un proceso cada vez más irreversible mientras que separaciones menores a \SI{57}{\milli\volt} sugieren sitios rédox\index{sitios rédox} anclados o sonda\index{sonda}s adsorbidas.\cite{Wi2000,Bockris1974,Gewirth2004}  
	 		
	 		Esta técnica se utilizó para evaluar fenómenos de exclusión, permeación\index{permeación} y preconcentración. También para determinar concentración de las sonda\index{sonda}s electroactivas dentro y fuera de los poros, calcular coeficientes de difusión\index{difusión} y estimar distancias de sitios rédox\index{sitios rédox} así como chequear accesibilidad\index{accesibilidad} y estructura de las películas delgadas mesoporosas.

	 \subsection{Voltametría cíclica de corriente alterna}

	 		La técnica de voltametría cíclica \index{voltametría!cíclica}de corriente alterna (VCA) consta en aplicar una oscilación sinusoidal de potencial a la celda electroquímica\index{electroquimico}. A la onda triangular clásica usada en VC se le suma, montando sobre ella, una pequeña onda de corriente alterna. En los experimentos presentados en este trabajo la amplitud de la onda adicionada fue de de \SI{10}{\milli\volt} y la frecuencia de la misma de 1 y \SI{2}{\hertz}. Esta técnica se emplea en conjunto con un analizador de frecuencias para filtrar la componente continua de la alterna, de este modo, ofrece un límite de detección menor e incrementa la sensibilidad respecto de la CV tradicional.\cite{Wi2000,Skoog1995}

	 		En la figura \ref{fig:ACV_ideal} se muestra la onda triangular con la perturbación, y la curva obtenida para una sonda\index{sonda} electroquímica\index{electroquimico} idealmente reversible, luego del filtrado de la componente continua.

	 		El propósito de esta técnica fue el obtener el coeficiente de difusión\index{difusión} de hexaaminorutenio en sistemas porosos y contrastar con otras técnicas de forma de validar dicho coeficiente y los mecanismos de transporte\index{transporte} propuestos. 

	 			 \begin{figure}[hb!]
			  		  \begin{subfigure}[t]{0.495\textwidth}
			  		  \includegraphics[width=\textwidth]{Graficos/onda-triangular-sin.pdf}
			  		  \end{subfigure}
			  		  \begin{subfigure}[t]{0.495\textwidth}
			  		  \includegraphics[width=\textwidth]{Graficos/ACV-ideal.pdf}
			  		  \end{subfigure}
			  		  \caption[Voltamperometria ideal reversible]{Curva de excitación utilizada en un voltametría de corriente alterna\index{voltametría!de corriente alterna} y el típico voltagrama\index{voltagrama} que se obtiene para una especie rédox reversible.}
			  		  \label{fig:ACV_ideal}
			  		  \end{figure}
			
	 \subsection{Simulaciones por el método de elementos finitos\index{metodo de elementos finitos@método de elementos finitos}}\label{simulacion}

	 	 Con la finalidad de validar las hipótesis de transporte\index{transporte} planteadas en el capítulo \ref{chap:Electroquimica} y establecer las condiciones en las que se pueden o no observar fenómenos de mediación\index{mediacion rédox} electroquímica\index{electroquimico}, se llevaron a cabo simulaciones numéricas por computadora de algunos experimentos relevantes para tal fin.

	 	 Se optó para realizar las simulaciones el método de elementos finitos\index{metodo de elementos finitos@método de elementos finitos} (MEF). El MEF es un método de cálculo numérico, especialmente orientado a la resolución de ecuaciones diferenciales, de amplia difusión\index{difusión} y para el cuál existen una gran cantidad programas con módulos preprogramados orientados para diversas aplicaciones (problemas mecánicos, físicos, químicos, biológicos, etc.). En términos matemáticos, el MEF es una técnica numérica para la resolución de problemas descriptos como un conjunto de ecuaciones diferenciales parciales. Comúnmente la ecuación básica que se necesita resolver para simular fenómenos electroquímico\index{electroquimico}s es la ecuación de difusión\index{difusión}, la cual relaciona la concentración $c$ con el tiempo $t$ y la distancia al electrodo $x$, dada por el coeficiente de difusión\index{difusión} $D$.\cite{Britz2005,Nann2003} Se trata de un ecuación de segundo orden en derivadas parciales y se la conoce como la segunda ley de Fick\index{Fick}.\cite{fick1855}

	 	 \begin{equation}
	 	 	\frac{\partial c}{\partial t}=D\frac{\partial^2 c}{\partial x^2	}
	 	 	\label{eq:fick}
	 	 \end{equation}

	 	 Esta ecuación es la base para simular los fenómenos de transporte, a la cual se le suman variables y complicaciones derivadas de las condiciones de contorno de cada sistema que se quiera simular. Convección, reacciones químicos, fenómenos de adsorción\index{adsorción} son sólo algunas de estas complicaciones que producen cambios en el perfil de concentraciones y en la difusión\index{difusión} en sí, agregando complejidad al sistema de ecuaciones diferenciales.  


	 	 		\begin{figure}[b!]
			 		  \begin{center}
			 		  \includegraphics[width=0.80\textwidth]{Esquemas/modelo-simulacion.pdf}
			 		  \caption[Modelos para simulación\index{simulacion} por MEF]{Esquema del modelo utilizado para las simulaciones por el MEF y las variables que se emplearon para evaluar fenómenos de transporte\index{transporte} y mediación\index{mediacion rédox} rédox.}
			 		  \label{esq:simulacion}
			 		  \end{center}
			 		  \end{figure}


	 	 El modelo aplicado en este trabajo utilizaron condiciones de contorno de equilibrio en la superficie\index{superficie} del electrodo, descriptas por la ecuación de Nernst\index{Nernst}. Dichas condiciones se aplicaron tanto para sistemas con una o más sonda\index{sonda}s en solución cómo para sistemas más complejos donde se pueden observar fenómenos de mediación\index{mediacion rédox} rédox o permeación, y son válidas siempre y cuando la cinética de electrodo sea suficientemente rápida. Todos los procesos difusivos, ya sean dentro de la película mesoporosa\index{película!mesoporosa} o en la solución, fueron descriptos según la segunda ley de Fick\index{Fick} (ec. \ref{eq:fick}). Los coeficientes de cada especie fueron variando en función las condiciones de cada experimento simulado. En los caso en los que se simuló una mediación\index{mediacion rédox} rédox, la reacción entre la sonda\index{sonda} y el mediador fue descripta como una reacción bimolecular de orden uno en cada especie, representando a las moléculas utilizadas en los experimentos que fueron \aminorutenio\space (\ru) y ferroceno metanol\index{ferroceno metanol} (\fc) como mediador y sonda\index{sonda} respectivamente. En ningún caso se considera migración, lo cual es válido siempre y cuando la concentración del electrolito soporte sea suficientemente alta. El modelo asume que el sistema es homogéneo (tanto en la solución como en la película) en planos paralelos al electrodo y por lo tanto solo existen gradientes en la dirección normal al electrodo. 

	 	 La figura \ref{esq:simulacion} es un diagrama esquemático representativo del modelo con las variables químicas y físicas que se tuvieron en cuenta en el caso de un sistema que presenta simultáneamente mediación\index{mediacion rédox} rédox y permeación\index{permeación} a través de la película delgada \index{película!delgada}mesoporosa. Las moléculas (\ru\space como mediador y \fc\space como sonda\index{sonda}) y concentraciones utilizadas son reflejo de los experimentos en laboratorio cuyos resultados se discutirá en el capítulo \ref{chap:Electroquimica}.  En la tabla \ref{tabla:simulacion} se resumen las condiciones de contorno impuestas para las simulaciones, los valores de las variables utilizadas y la descripción de cada una de ellas.
		
	 	 Las simulaciones se realizaron en colaboración con el Dr. Tagliasucchi del Instituto de Química Física de los Materiales, Medio Ambiente y Energía\linebreak  (INQUIMAE\index{INQUIMAE}) con el software \textit{COMSOL Multiphysics\textsuperscript\textregistered} (\url{https://www.comsol.com/}) el cual cuenta con un módulo electroquímico\index{electroquimico} (\url{https://www.comsol.com/electrochemistry-module}).

	    	\begin{table}[hb!]
	 	    \caption[Parámetros de las simulaciones]{Parámetros y valores de entrada usadas durante las simulaciones de las voltametrías cíclicas.}
	 	    \begin{tabular}{>{\raggedright\arraybackslash}m{1.4cm}>{\centering\arraybackslash}m{2.8cm}>{\raggedright\arraybackslash}m{6.7cm}} 
	 	    \toprule
	 	    Variable  & 	Valor  &   descripción      \\ \midrule
	 	    $d$  	  &  \hspace{-5mm} \SI{200}{nm}	& 	   espesor\index{espesor} de las películas delgadas mesoporosas 	    \\ \midrule
	 	    $C_{\fc}$  & \SI{5}{\milli\Molar}  & concentración de FeOH en solución    \\ \midrule
	 	    $C_{\ru}$ & \SI{1}{\Molar}  & concentración de \ru\space en las películas    \\ \midrule
	 	    $C_{KCl}$ & \hspace{-5mm}  \SI{100}{\milli\Molar}  & concentración de electrolito soporte    \\ \midrule
	 	    $k$ 		   & variable 	 & 	constante de mediación\index{mediacion rédox} rédox    \\ \midrule
	 	    $E^\circ_{\ru}$  & \hspace{-3mm}\SI{-0.3}{\volt} vs ECS & potencial reducción estándar el \ru \\ \midrule
	 	    $E^\circ_{\fc}$  & \SI{0.3}{\volt} vs ECS & potencial reducción estándar el \fc \\ \midrule
	 	    $D_{\fc}^{pel} $  & variable & coeficiente de difusión\index{difusión} del \fc\space en la película \\ \midrule
	 	    $D_{\fc}^{sol} $  & variable & coeficiente de difusión\index{difusión} del \fc\space en la solución \\ \midrule
	 	    $D_{e}$  & variable & coeficiente de difusión\index{difusión} por \textit{electron hopping }del \ru\space en el film \\ \midrule
	 	    $\nu$    & \hspace{3.5mm} \SI{50}{\milli\volt\per\second}  &  velocidad de barrido\index{velocidad!de barrido} \\
	 	     \bottomrule
			\end{tabular}
			\label{tabla:simulacion}
			\end{table}	  
			
	 \subsection{Celdas electroquímica\index{electroquimico}s\index{electroquimico}}\label{sec:celdas_eq}	

			Las mediciones electroquímica\index{electroquimico}s\index{electroquimico} se llevaron a cabo sobre dos ensambles experimentales diferentes. El primero de ellos fue utilizado durante las etapas más tempranas de la tesis, sobre electrodos metálicos no litografiados. Para este armado se preparó una celda fabricada en acrílico, con un volumen aproximado de \SI{3}{\ml}, y un orificio en la parte inferior al cual se le pueden adaptar sellos de polipropileno\index{polipropileno} de diferentes diámetros para evitar fugas de la fase líquida. Se utilizó un sello de \SI{1}{\mm} de radio el cual determinó el área geométrica del electrodo en \SI{3.15}{\mm^{2}}. Una vez puesto en contacto el sello con el electrodo, se llenó la celda por la parte superior con la solución que contiene la sonda\index{sonda} electroquímica\index{electroquimico} y se verificó, en cada una de las mediciones, que no se haya perdidas de líquido. Éste montaje experimental precisó utilizar ER y CE externos. En la figura \ref{fig:celda}a se expone una fotografía de este sistema. 

					 \begin{figure}[t!]
			  		 \begin{subfigure}[t]{0.495\textwidth}
			  		  \includegraphics[width=\textwidth]{Imagenes/EQ1.jpg}
			  		  \end{subfigure}
			  		  \begin{subfigure}[t]{0.495\textwidth}
			  		  \includegraphics[width=\textwidth]{Imagenes/EQ2.jpg}
			  		  \end{subfigure}
			  \caption[Equipo para realizar la medidas electroquímica\index{electroquimico}s\index{electroquimico}]{Fotografía del instrumental que se utilizó a lo largo de la tesis para tomar las medidas electroquímica\index{electroquimico}s\index{electroquimico}. Izquierda: celda de acrílico sobre electrodos de Au\index{electrodo!de Au} \index{oro}recubierto con una \pdm\space utilizando CE y ER externos. Derecha: oblea\index{oblea} con 6 ET por sensor, celda fabricada con resina epoxi\index{resina!epoxi}\index{epoxi} SU8, CE integrado y utilizando un ER externo.}
			  		 \label{fig:celda}
			 		 \end{figure}

             En el segundo ensamble experimental, cuya fotografía se muestran en la figura \ref{fig:celda}b, las mediciones se llevaron a cabo sobre electrodos litografiados, en los cuales se encuentran integrados en los multisensor\index{multisensor}es el contraelectrodo\index{electrodo!contraelectrodo} y un electrodo de pseudoreferencia\index{pseudoreferencia}. En los casos que no fue posible usar la pseudoreferencia\index{pseudoreferencia} se utilizó un electrodo de referencia\index{electrodo!de referencia} externo. En esta configuración el área geométrica quedó determinada por los electrodos, variando desde $0.1$ a \SI{0.01}{\square\mm} según el diseño (consultar sección \ref{sec:diseno}). En todos los voltagrama\index{voltagrama}s presentados en este trabajo, con el objetivo de simplificar las comparaciones, las intensidades fueron normalizados por el área y los potenciales fueron referidos al potencial de un electrodo estándar de calomel\index{electrodo!de calomel} saturado (\SI{0.248}{\volt} contra el electrodo normal de hidrógeno\index{hidrógeno} a \SI{20}{\celsius})\cite{BANUS1941}. 
					 		  
			 Las mediciones electroquímica\index{electroquimico}s\index{electroquimico} fueron tomadas con un potenciostato\index{potenciostato} \textit{Teq4}, o un potenciostato\index{potenciostato} \textit{Autolab} de la firma \textit{Ecochemie} para las medidas que se necesitaron velocidades de barrido mayores a \SI{1}{\volt\per\second}. Como electrodo de referencia\index{electrodo!de referencia} se empleó un electrodo de calomel\index{electrodo!de calomel} saturado (ECS) de la firma \textit{Cole-Parmer} y como contraelectrodo\index{electrodo!contraelectrodo} (CE) se utilizaron indistintamente electrodos de Au\index{electrodo!de Au} \index{oro}depositados por pulverización catódica\index{pulverización catódica} o una pieza de Pt\index{platino} de \SI{2}{\square\cm} de área.  
	
	 \subsection{Sondas electroquímica\index{electroquimico}s\index{electroquimico}}\label{sec:respuesta_sondas_au} 

	 		En esta sección se dará cuenta de la caracterización electroquímica\index{electroquimico} de los electrodos de Au\index{electrodo!de Au} \index{oro}para cada una de las sonda\index{sonda}s utilizadas a lo largo de la tesis. Los multisensor\index{multisensor}es basan su principio de detección en reacciones de óxido/reducción, es por ello que es necesario obtener electrodos de respuesta reproducible, confiable y fabricados por un proceso repetible y escalable. 

			Una vez que los resultados de la fabricación de los sensores\index{sensor} fueron óptimos (consultar capítulo \ref{chap:Microfabricacion}) se evaluó el desempeño electroquímico\index{electroquimico} de los mismos. Se usaron como sonda\index{sonda}s electroquímica\index{electroquimico}s\index{electroquimico} ferrocianuro y ferricianuro de potasio\index{ferricianuro de potasio} (\Ferro\space y \Ferri, \fe), cloruro de hexaaminorutenio(III) (\aminorutenioCompleto, \ru) y ferroceno metanol\index{ferroceno metanol} (\ferroceno, \fc). La elección de estas sonda\index{sonda}s modelo tiene que ver fundamentalmente con la carga neta de cada una de ellas, y con la reversibildiad de los pares rédox. En la tabla \ref{tabla:eq} se resumen las sonda\index{sonda}s electroquímica\index{electroquimico}s\index{electroquimico} utilizadas en las mediciones y los reactivos utilizados para preparar las soluciones. 
			
			     \begin{table}[h!]
			  		  \caption{Reactivos y sonda\index{sonda}s electroquímica\index{electroquimico}s\index{electroquimico} utilizados para las mediciones electroquímica\index{electroquimico}s\index{electroquimico}.}
			  		   \begin{tabular}{>{\raggedright\arraybackslash}m{5.1cm}>{\centering\arraybackslash}m{1.3cm}>{\centering\arraybackslash}m{2.65cm}>{\raggedright\arraybackslash}m{1.4cm}} 
			  		  \toprule
					  Reactivo \hspace{3cm}Nombre& Marca & Peso Molecular (\si{g.mol^{-1}}) & Función  \\ \midrule
			    	  \ferroCompleto \hspace{3cm} ferrocianuro de potasio\index{ferrocianuro de potasio} & \textit{Sigma} & 422,41  & Sonda \\ \midrule
			    	  \ferriCompleto \hspace{3cm} ferricianuro de potasio\index{ferricianuro de potasio} & \textit{Sigma} & 329,27  & Sonda  \\ \midrule
			  		  \aminorutenioCompleto  \hspace{3cm}  cloruro de hexaaminorutenio(III)& \textit{Aldrich} &  309,61  & Sonda  \\ \midrule
			  		  \raisebox{-.5\height}{\includegraphics[scale=0.4]{Esquemas/Fc.pdf}}  \hspace{3cm} ferroceno metanol\index{ferroceno metanol}   & \textit{Aldrich} &  216,06 & Sonda  \\ \midrule
			  		  H$_2$O \hspace{4cm} agua\index{agua} &  \SI{18}{\mega\ohm\per\cm}  &  18,02 & Solvente \\ \midrule
			  		  KCl  \hspace{4cm} cloruro de potasio   & \textit{Biopack} & 74,56 & Electrolito Soporte \\
 			  		  \bottomrule
			    	  \end{tabular}
			   		  \label{tabla:eq}
			   		  \end{table} 		 	
				
		\vspace*{-2mm}\subsubsection{Respuesta de ferrocianuro/ferricianuro de potasio}	 
			 	
		   En electroquímica\index{electroquimico} este par rédox es frecuentemente utilizado para evaluar la calidad de electrodos. Esto se debe a que se trata de un par rédox cuyas especies oxidada y reducida son económicas, fáciles de conseguir, solubles en solución acuosa y se comportan de forma cuasireversible frente al intercambio electrón\index{electrón}ico. La reacción que tiene lugar es la siguiente:
		
			 \begin{equation}
			   \schemestart 
			   Fe(CN)$_6^{4-}$  
			   \arrow{<=>[\scriptsize oxidación][\scriptsize reducción]}[0,1.5] 
			   Fe(CN)$_6^{3-}+e^-$ \schemestop
			   \end{equation}
		
			\vspace*{1mm}

		   Se espera, en la aproximación más simple, que sigan el comportamiento voltamétrico descrito por Randles-Sevcik\index{Randles-Sevcik}, donde la corriente de pico ($i_p$) es proporcional a la concentración ($C$) y a la raíz cuadrada de la velocidad de barrido\index{velocidad!de barrido} $v$ según:
		  
		 	\begin{equation}
			   i_p=0.4463nFAC\left(\frac{nFvD}{RT}\right)^{1/2}
		     	\label{eq:rs2}
			\end{equation}

     		 Con el propósito de corroborar este comportamiento, se realizaron experimentos de VC a diferentes concentraciones de la sonda\index{sonda} (figuras \ref{fig:Fe_a} y  \ref{fig:Fe_b}) y a distintas velocidades de barrido (figuras \ref{fig:Fe_c} y  \ref{fig:Fe_d}). Resultaron de especial utilidad la curva de calibración y la respuesta frente a distintas velocidades de barrido. Para cualquiera de estás velocidades los voltagrama\index{voltagrama}s conservan constantes los valores de $E_p$, indicativo de una óptima trasferencia de carga entre el electrodo y la sonda\index{sonda}. Además se destaca la relación lineal de j$_p$ con $v^{1/2}$ verificando la ecuación de Randles-Sevcik\index{Randles-Sevcik} (ec. \ref{eq:rs2}) y que el sistema responde a un proceso de difusión\index{difusión} lineal semiinfinita, lo cual es esperable  para un experimento en el que el electrodo esta en contacto directo con la solución que contiene el analito electroactivo y el electrolito soporte {(\SI{0,1}{\Molar} KCl, pH\index{pH}=$5.5$)\cite{Wi2000,Pumera2007,Gewirth2004,Villullas2000}.
     		 

     		 \begin{figure}[h!]
	 	    	\begin{subfigure}[t]{0.5\textwidth}
	         	\includegraphics[width=\textwidth]{Graficos/Concentraciones_Fe.pdf}
	        	\caption{Voltametrías cíclicas para la cupla \fe\space a diferentes concentraciones. Todas medidas fueron tomadas a \SI{50}{\milli\volt\per\second}.}
	         	\label{fig:Fe_a}
	         	\end{subfigure}
	         	\vspace*{2mm}
     		 \begin{subfigure}[t]{0.495\textwidth}
	        	\includegraphics[width=\textwidth]{Graficos/Calibracion_Fe.pdf}
	       		\caption{Curva de calibración para distintas concentraciones de la cupla \fe. Valores extraídos de la figura \ref{fig:Fe_a}.}
	         	\label{fig:Fe_b}
	     		\end{subfigure}
 	     	\begin{subfigure}[t]{0.495\textwidth}
         		\includegraphics[width=\textwidth]{Graficos/Velocidades_Fe.pdf}
        	    \caption{Voltametrías cíclicas de una solución \SI{10}{\milli\Molar} de la cupla equimolar \fe\space para diferentes velocidades de barrido.}
        	    \label{fig:Fe_c}
     		 	\end{subfigure}
     		 	\vspace*{2mm}
     	 	\begin{subfigure}[t]{0.495\textwidth}
        		\includegraphics[width=\textwidth]{Graficos/VelocidadesCal_Fe.pdf}
       			\caption[Respuesta a diferentes velocidades de barrido para \fe]{Dependencia de la corriente de pico con la velocidad de barrido\index{velocidad!de barrido} para \fe\space \SI{10}{\milli\Molar}. Valores extraídos de la figura \ref{fig:Fe_c}.}
         		\label{fig:Fe_d}
     			\end{subfigure}
     		 \caption[Respuesta electroquímica\index{electroquimico} para \fe]{(a) Respuesta electroquímica\index{electroquimico} de la cupla equimolar \fe\space para distintas concentraciones, (b) curva de calibración para dichas concentraciones. (c) Variación de la densidad de corriente con la velocidad de barrido\index{velocidad!de barrido}, (d) dependencia de la densidad de corriente de pico con la raíz cuadrada de la velocidad de barrido\index{velocidad!de barrido}. Todos los voltagrama\index{voltagrama}s fueron tomadas con contraelectrodo\index{electrodo!contraelectrodo} de Pt\index{platino}, en \SI{0,1}{\Molar} de KCl como electrolito soporte y utilizando como referencia ECS.}
     		 \label{fig:ferro-ferri-CV}
     		 \end{figure}
		
		\subsubsection{Respuesta del cloruro de hexaaminorutenio(III)}
	
	 	 El cloruro de hexaaminorutenio(III) se utilizó extensamente en este trabajo debido a la reversibilidad del par rédox y a que ambos estados de oxidación tiene carga positiva. Gran parte de la discusión del capítulo \ref{chap:Electroquimica} tiene por eje la adsorción\index{adsorción} de este complejo en las películas delgadas mesoporosas. 

	 	 \pagebreak Esta molécula\index{moléculas} se disocia en solución para formar el complejo \aminorutenio. La reacción rédox que tiene lugar es la siguiente:
	 		 	 	  	
	 		 	 	  		\begin{equation}
	 		 	 	 			\schemestart 
					 			 Ru(NH\index{amoniaco}$_3$)$_6^{3+}+e^-$  
					 			 \arrow{<=>[\scriptsize reducción][\scriptsize oxidación]}[0,1.5] 
					 		 	 Ru(NH\index{amoniaco}$_3$)$_6^{2+}$ \schemestop 
	 		 	 	 		\end{equation}

	 	  El intercambio entre los estados de oxidación Ru$^{3+}$/Ru$^{2+}$ responde a un proceso electroquímico\index{electroquimico} reversible en el cual podemos fácilmente reducir u oxidar el complejo variando el potencial del electrodo de trabajo\index{electrodo!de trabajo}. Habiendo ya comprobado, con el \fe, el buen desempeño de los electrodos respecto de la velocidad de barrido\index{velocidad!de barrido}, se eligió un valor \SI{50}{\milli\volt\per\second} para las voltametrías cíclicas \index{voltametría!cíclica}(de uso frecuente para este tipo de mediciones). Se llevaron a cabo una serie de VC para varias concentraciones de la sonda\index{sonda}, con el objetivo de elaborar la correspondiente curva de calibración para \aminorutenio. Con estas mediciones se verificó nuevamente el buen desempeño EQ de los electrodos y el modelo de difusión\index{difusión} semiinfinita.
		
			 \begin{figure}[ht]
	 	     \begin{subfigure}[t]{0.495\textwidth}
	         	\includegraphics[width=\textwidth]{Graficos/Concentraciones_Ru.pdf}
	        	\caption{Voltametrías cíclicas para \ru\space a diferentes concentraciones.}
	         	\label{fig:Ru_a}
	     		\end{subfigure}
     		 \begin{subfigure}[t]{0.495\textwidth}
	        	\includegraphics[width=\textwidth]{Graficos/Calibracion_Ru.pdf}
	       		\caption{Curva de calibración para la especie \ru. Los valores fueron extraídos de la figura \ref{fig:Ru_a}.}
	         	\label{fig:Ru_b}
	     		\end{subfigure}
	     		\label{rutenio}
	     		\caption[Respuesta electroquímica\index{electroquimico} para \ru]{(a) Voltametrías cíclicas para soluciones de \ru\space de distinta concentración y, (b) curva de calibración para dichas concentraciones. Todos los voltagrama\index{voltagrama}s fueron tomados a \SI{50}{\milli\volt\per\second} con contraelectrodo\index{electrodo!contraelectrodo} de Pt\index{platino} en una solución \SI{0.1}{\milli\Molar} de NaCl y utilizando de referencia ECS.}
	     	 \end{figure}
			 		 	 
		\subsubsection{Respuesta del ferroceno metanol}
 	 	 
 	 	  A diferencia de las sonda\index{sonda}s anteriores, la especie reducida del \fc\space no tiene carga neta, por lo que se puede esperar que no tenga interacciones de tipo electrostáticas con las películas delgadas mesopoporosas, es por ello que resultó especialmente útil para sacar conclusiones y comparar como varía el transporte\index{transporte} en sistemas calcinados y sistemas no calcinados. En el capítulo \ref{chap:Electroquimica} se realizará una discusión detallada sobre este tema. 
 	 	  La reacción de oxidación/reducción para esta molécula\index{moléculas} es la siguiente:
 	 	 			
 	 				 \begin{equation}
 	 	 				\begin{aligned}
 	 	 				\includegraphics[scale=0.75]{Esquemas/Redox-Fc.pdf}
 	 	 				\end{aligned}
 	 	 			 \end{equation}
 	 	  
 	 	 De la misma forma que se realizó para las otras sonda\index{sonda}s, se confeccionó una curva de calibración para distintas concentraciones de \fc\space. En la figura \ref{Fig:Fc} sen presenta la respuesta electroquímica\index{electroquimico} correspondiente sobre electrodos de Au\index{electrodo!de Au}.
 	 				
 	 				%Graficos para el Ferroceno
		 		 \begin{figure}[ht]
		 	      \begin{subfigure}[t]{0.495\textwidth}
		          	\includegraphics[width=\textwidth]{Graficos/Concentraciones_Fc.pdf}
		         	\caption{Voltametrías cíclicas para \fc\space a diferentes concentraciones, \SI{1}{\milli\Molar}, \SI{5}{\milli\Molar} y \SI{10}{\milli\Molar}.}
		          	\label{fig:Fc_a}
		      		\end{subfigure}
		      	 \begin{subfigure}[t]{0.495\textwidth}
		          	\includegraphics[width=\textwidth]{Graficos/Calibracion_Fc.pdf}
		         	\caption{Curva de calibración para la especie \fc. Los valores fueron extraídos de la figura \ref{fig:Fc_a}.}
		          	\label{fig:Fc_b}
		      		\end{subfigure}
		      	 \caption[Respuesta electroquímica\index{electroquimico} para \fc]{(a) Voltametrías cíclicas para soluciones de \fc\space de distinta concentración y, (b) curva de calibración para dichas concentraciones. Todos los voltagrama\index{voltagrama}s fueron tomadas a \SI{50}{\milli\volt\per\second} con contraelectrodo\index{electrodo!contraelectrodo} de Pt\index{platino} en una solución \SI{0.1}{\Molar} de NaCl y utilizando ECS como referencia.}
		      	 \label{Fig:Fc}
	      		 \end{figure}

		\vspace*{4mm}En la tabla \ref{tabla:sondas} se resumen las variables rédox para cada una de las sonda\index{sonda}s modelo utilizadas: 1) estado de carga de los estados reducidos y oxidados y, 2) diferencia de potenciales entre el pico de corriente anódico y catódico. Este último dato resulta útil para  evaluar el grado de reversibilidad de la reacción y la capacidad de transferencia electrón\index{electrón}ica de los electrodos. Durante todo el desarrollo de la tesis se utilizaron estos parámetros con ánimos de comparar resultados sobre transporte\index{transporte} y propiedades permeoselectiva sobre los electrodos desnudos y sobre los electrodos recubiertos con películas delgadas mesoporosas.\cite{koryta1993,Otal2006,longinotti2007}
		  
		  %Tabla resultados EQ
		  \vspace*{4mm}
		     \begin{table}[h!]
		     \renewcommand{\arraystretch}{1.1}
	  		  \caption[Sondas electroquímica\index{electroquimico}s\index{electroquimico}]{Características de las sonda\index{sonda}s electroactivas utilizadas a lo largo de la tesis.}
	  		  \begin{tabular}{>{\raggedright\arraybackslash}m{3cm}>{\centering\arraybackslash}m{2.7cm}>{\centering\arraybackslash}m{2.7cm}>{\centering\arraybackslash}m{2.05cm}}
	  		  \toprule
			  \multirow{2}{*}{Sonda}  	& Carga especie  & Carga especie  & \multirow{2}{*}{$\Delta E_p$(mV)} \\
			     		    & \hspace*{-0.79cm}reducida      & \hspace*{-0.85cm}oxidada  &	\\ \midrule
	    	  \ferroCompleto	& 	$4-$	& $3-$	     			   &  150  \\ \midrule
	    	  \ferriCompleto	&   $3-$	& $4-$	     			   &  150  \\ \midrule
	  		  \aminorutenioCompleto	& 	$2+$	& $3+$					   &  80    \\ \midrule
	  		  \raisebox{-.5\height}{\includegraphics[scale=0.6]{Esquemas/Fc.pdf}}   &  \hspace*{-0.29cm}0 & 1+ &  103 \\   		 
	  		  \bottomrule
	    	  \end{tabular}
	   		  \label{tabla:sondas}
			  \end{table}
				