%Sección de Plugins
%--------------------------------------------------------------------------
   %Idioma, fuentes y entrada de teclado
   \usepackage[spanish,activeacute,es-tabla,es-lcroman]{babel}
   \usepackage[utf8]{inputenc}
   \usepackage{lmodern}
   
   %Paquete para adaptar la Bibliografia al Español
   \usepackage[datename]{babelbib}

   %Para manejar la forma en que se presentan las citas
   \usepackage[superscript]{cite}
       
   % Paquete para gestionar imágenes jpg
   \usepackage{graphicx}

   % Paquete para hacer cuadros
   \usepackage{boxedminipage} 

   %Paquete para colocar imagenes en el medio del texto
   \usepackage{wrapfig}

   %paquete para lenguaje matematico y símbolos
   \usepackage{amsmath,amssymb,amsfonts,latexsym,cancel,esint,pifont}
    \newcommand{\cmark}{\ding{51}} %Tilde de correcto
    \newcommand{\xmark}{\ding{53}} %Cruz de equivocado

   % Paquete para poner colores
   \usepackage[usenames,dvipsnames]{xcolor}

   %Para la justificación, formato y texto de los caption de las figuritas y tablas
   \usepackage{caption}
   
   %Para que me coloque el primer float justo al tope de pagina
    \makeatletter
    \setlength{\@fptop}{0pt}
    \makeatother
    
   %Paquete para poder poner imagenes una al lado de la otra
   \usepackage{subcaption}
   
   %Para hacer graficos cons tikz, con gnuplot y poder exportarlos como pdf
       \usepackage{tikz}
       \usepackage{tikzpagenodes}
     
           %Librerias Tikz Extra
             \usetikzlibrary{shapes,arrows}
             \usetikzlibrary{shapes.arrows, fadings}
             \usetikzlibrary{shadows.blur}
             \usetikzlibrary{shapes.symbols}
             \usetikzlibrary{decorations.pathreplacing}
             \usetikzlibrary{fit}
             \usetikzlibrary{calc}
             \usetikzlibrary{positioning}
             \usetikzlibrary{matrix}

   %Para hacer lineas gruesa y tablas lindas
   \usepackage{booktabs}

   %Para Formatear los capitulos
   \usepackage{titlesec}

   %Herramientas para parches
   %Yo lo use para Cambiar el nombre de la Bibliografia
   \usepackage{etoolbox}
   
   %Para Tablas multicolumnas, multifilas y tablas largas
   \usepackage{multirow,multicol,array} 

   %Para rota Texto
   \usepackage{rotating} 

   %Para los Indices parciales en las portadas de los capitulos
   \usepackage[nohints]{minitoc}

   %Paquete para personalizar encabezados y notas al pie
   \usepackage{fancyhdr}   

   %Para que no numere las paginas en blanco (borrar los header y footnotes)
   \usepackage{emptypage}

   %Para carga Atributos, Propiedades y Titulos al PDF
   \usepackage[hidelinks,colorlinks]{hyperref}

   %Para personalisar listas de items, yo lo use para la lista de siglas.
   \usepackage{enumitem}

   %Para crear lista de abreviaturas
   \usepackage{acro}

   %Para hacer un indice alfabetico
   \usepackage{imakeidx}
   %Muestra el indexado en el costado
   %\usepackage{showidx}
   
   %Para Imprimir el LayOut del documento
   \usepackage{layout}

   %Para hacer quimica
   \usepackage{chemfig}

   %para meter text de relleno
   \usepackage{blindtext}

   %Paquete gestor de unidades
   \usepackage[version-1-compatibility]{siunitx}
   \sisetup{output-decimal-marker = {,}}

   %Para Colocar objetos en el fondo, en mi caso los indicadores laterales
   \usepackage[contents={},opacity=1,scale=1,color=black]{background}
   %--------------------------------------------------------------------------

%Comandos para suprimir parte de los logs
%--------------------------------------------------------------------------
    
    %SUprimir log de Overfull y UNdelfull
    \newcommand{\NoBadBoxesLog}{
      \hfuzz=\maxdimen \vfuzz=\maxdimen 
      \newdimen\hfuzz  \let\vfuzz=\hfuzz
      \hbadness=10000    \vbadness=10000 
      \newcount\hbadness \let\vbadness=\hbadness 
      }
    %\NoBadBoxesLog  
      
   %Suprime el Warning de multiples PDF por pagina
   \pdfsuppresswarningpagegroup=1

%Propiedades del Documento
%--------------------------------------------------------------------------
   \hypersetup{
   pdftitle={Tesis Gustavo Giménez},
   pdfauthor={Gustavo Giménez},
   pdfsubject={Tesis Doctoral}, %subject of the document
   pdfcreator={Tex},
   pdfproducer={PDFLatex},
   pdfkeywords={UBA,nanomateriales,MEMS,NEMS,INTI},
   % %pdftoolbar=false, % toolbar hidden
   pdfmenubar=true, %menubar shown
   pdfhighlight=/O, %effect of clicking on a link
   % pdfpagemode=UseOutlines,%UseNone, %aucun mode de page
   % pdfpagelayout=TwoPageRight,%SinglePage, %ouverture en simple page
   % pdffitwindow=true, %pages ouvertes entierement dans toute la fenetre
   % bookmarksopenlevel=2,
   %    colorlinks=false,       % false: boxed links; true: colored links
   linkcolor=black,          % color of internal links (change box color with linkbordercolor)
   citecolor=grisoscuro,        % color of links to bibliography
   urlcolor=grisoscuro,          % color of external link
   %backref=true
   }
   %--------------------------------------------------------------------------