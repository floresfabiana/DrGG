%%%%%%%%%%%%%%%%%%%%%%%%%%%%%%%%%%%%%%%%%%%%%%%%%%%%%%%%%%%%%%%%%%%
%   Aca van los Formatos Personalizados de todo el documento	  %
%%%%%%%%%%%%%%%%%%%%%%%%%%%%%%%%%%%%%%%%%%%%%%%%%%%%%%%%%%%%%%%%%%%

%Formato General del Documento
%------------------------------------------------------------
 
 %Intercambio los margenes derechos e izquierdos
 %------------------------------------------------------------
 \let\tmp\oddsidemargin
 \let\oddsidemargin\evensidemargin
 \let\evensidemargin\tmp
 \reversemarginpar

 %La class book por default pone un espacio para notas marginales en el lado derecho para las paginas 
 %impares y en el izquierdo para las pares, yo lo invertí para que me quede mas amplio el espacio para 
 %la unión de las paginas en el lomo del libro, la costura.
 

    % Defino los margenes bien personalizados.
    % Zona de la pagina ocupada por el texto
    % Información detallada: http://en.wikibooks.org/wiki/LaTeX/Page_Layout
    % O descomentar la linea  \layout 
	% \layout
 	%------------------------------------------------------------------------------------------------------------------ 
 		% 	    \hoffset = 2 cm                          % OffSet desde el borde izquierdo de la pagina
 		% 		\voffset = 2 cm                          % OffSet desde el borde Superior de la pagina
 		% 		\topmargin = -2.54 cm                    % Desde el borde superior de la pagina hasta el texto + 1 inch
		% 		\oddsidemargin = -2.54 cm                % Desde el borde izquierdo de la pagina hasta el texto + 1 inc
		% 		\headsep =  0 cm                         % Separación entre el encabezado y el texto	
 		% 		\textwidth = 17 cm                       % Ancho del texto y define margen derecho - margen - offset
 		% 		\textheight = 23.5 cm                    % Largo del texto y margen inferior - margen - offset
 		% 		\headheight = 0.1 cm                     % Espacio del cuadro de texto del Encabezado
		% 		\footskip = 1.5 cm                       %Espacio para el pie de pagina 
 		%    	\marginparwidth = 95pt					 %Ancho de las notas al margen
 		% 	    \marginparsep = 25pt					 %Separación margen-texto
 	%------------------------------------------------------------------------------------------------------------------

%Formato de las Cabeceras, los Pie de Pagina y la numeracion
%------------------------------------------------------------
	%Cambio el Header para que no salga la palabra "Capitulo"
	\addto{\captionsspanish}{\def\chaptername{}}

	%Cambio el Color de la Headrule
	\let\headruleORIG\headrule
	\renewcommand{\headrule}{\color{grisoscuro}\headruleORIG}
	
	%Estilo de las Cabeceras y Pie de pagina para los precapitulos
	\fancypagestyle{frontmatter}{
		\fancyhead{}   							%Borro Cabeceras
		\renewcommand{\headrulewidth}{0pt} 		%Borro Headrule							
		\fancyfoot[C]{\thepage}    				%Numero de pagina centrado
		}
	
	%Estilo de las Cabeceras y Pie de pagina para el cuerpo principal
	\fancypagestyle{mainmatter}{
		\fancyfoot{}													   % Borro las paginas al pie
		\renewcommand{\headrulewidth}{1pt}  							   % Borro la Linea del encabecera		
		\fancyfoot[LE,RO]{\color{grisoscuro}\bfseries\thepage}             % Pongo el numero de pagina en todas la paginas en el lado exterior
	 	\fancyhead[LE]{\color{grisoscuro}\bfseries\nouppercase{\leftmark}} % Pongo el capitulo arriba a la derecha en las paginas pares 
	    \fancyhead[RO]{\color{grisoscuro}\bfseries\nouppercase{\rightmark}}% Pongo la sección arriba a la izquierda en las impares
	    }
	
	%Estilo de las Cabeceras y Pie de pagina para los postcapitulos
	\fancypagestyle{backmatter}{
		\fancyhead{}
		\fancyfoot{}										    %Borro Cabeceras
		\renewcommand{\headrulewidth}{0pt} 				        %Borro Headrule							
		\fancyfoot[LE,RO]{\color{grisoscuro}\bfseries\thepage}  % Pongo el numero de pagina en todas la paginas en el lado exterior
		}
 %--------------------------------------------------------------

%Formato de las Etiquetas (Referencias, Capitulos, etc)
%------------------------------------------------------------
 %Cambiar el nombre de Bibliografia por Referencias
 \patchcmd{\thebibliography}{\chapter*{\bibname}}{\chapter*{Referencias}}{}{}
 %-------------------------------------------------------------

%Formato de los Titulos de los capitulos
%------------------------------------------------------------
    %Ubicación Espacial del conjunto "Titulo" de Capitulos
      \titlespacing*{\chapter}  %Seccion a modificar
 			 {-3cm}             %Indentación Izquierda
 			 {-1cm}            %Espacio Anterior
 			 {50pt} 	        %Espacio Posterior
 			 [0cm]	           %Indentación Derecha
    
    %Formato para Todos los Capitulos, solo lo uso para el Indice General
	 \titleformat{\chapter}[display]
			{\Huge\filleft\scshape\color{grisoscuro}}      	 %Formato para etiqueta y texto
	 		{}	
	 		{20pt}  	     %Separación Etiqueta/Texto                
	 		{}               %Codigo anterior
	 		[]   %Codigo Posteior  \titlerule\vspace{2ex}\filright\vspace{2ex}]
	
	%Variables para ubicar el numero de capitulo	
	 \newlength\NumCapX
	 \setlength\NumCapX{-1cm}  %Aumentar es mas para arriba
	 \newlength\NumCapY	
	 \setlength\NumCapY{-2cm}  %Aumentar es mas para el exterior		

	%Formato para el titulo de los Indices
	 \newcommand{\FormatoIndice}{
	 		\titleformat{name=\chapter,numberless}[display]						 %Sin Numerar
			{\Huge\filleft\scshape\color{grisoscuro}}      	 %Formato para etiqueta y texto
	 		{\begin{tikzpicture}[remember picture,overlay] 						 %Cuadrado con el numero
				\node[at={($(\NumCapX,\NumCapY)$)},anchor=center,font=\rmfamily\fontsize{300pt}{0pt}\bf\selectfont\color{grisoscuro},opacity=0.2] (texto) {\textit{I}};
				\node at (-12.16,-2) {\rule{25cm}{0.5mm}};
				\end{tikzpicture}
			}		
	 		{20pt}  	    				 %Separación Etiqueta/Texto                
	 		{\vspace*{-0.15cm}}              %Codigo anterior
	 		[]   				 %Codigo Posteior
	 } 

	%Formato Para los Capitulos Numerados y que el título ocupa una sola linea
 	 \newcommand{\FormatoCapituloUnaLinea}{
 		\titleformat{\chapter}     						     %Seccion a modificar
			[display]					 			 		 %Formato separado en dos renglones separados
			{\Huge\filleft\scshape\color{grisoscuro}}      	 %Cuadrado con el numero
			{\begin{tikzpicture}[remember picture,overlay]
				\node[at={($(\NumCapX,\NumCapY)$)},anchor=center,font=\rmfamily\fontsize{300pt}{0pt}\bf\selectfont\color{grisoscuro},opacity=0.2] (texto) {\textit{\thechapter}};
				\node at (-12.16,-2) {\rule{25cm}{0.5mm}};
			    \end{tikzpicture}
			}
	 		{20pt}  	     %Separación Etiqueta/Texto                
	 		{}               %Codigo anterior
	 		[]   %Codigo Posteior 
	 		}	
 	 		
	%Formato Para los Capitulos Numerados y que el título ocupa dos lineas
 	 \newcommand{\FormatoCapituloDosLineas}{
 		\titleformat{\chapter}     						     %Seccion a modificar
			[display]					 			 		 %Formato separado en dos renglones separados
			{\Huge\filleft\scshape\color{grisoscuro}}      	 %Cuadrado con el numero
			{\begin{tikzpicture}[remember picture,overlay]
				\node[at={($(\NumCapX,\NumCapY)$)},anchor=center,font=\rmfamily\fontsize{300pt}{0pt}\bf\selectfont\color{grisoscuro},opacity=0.2] (texto) {\textit{\thechapter}};
				\node at (-12.16,-2) {\rule{25cm}{0.5mm}};
				\end{tikzpicture}
			}
	 		{20pt}  	     		 %Separación Etiqueta/Texto                
	 		{\vspace*{-1.05cm}}      %Codigo anterior
	 		[]  		 %Codigo Posteior  
	 		}	
	
	%Formato para el titulo de los Anexos
	 \newcommand{\FormatoAnexoA}{
	 		\titleformat{name=\chapter,numberless}[display]						 %Sin Numerar
			{\Huge\filleft\scshape\color{grisoscuro}}      	 %Formato para etiqueta y texto
	 		{\begin{tikzpicture}[remember picture,overlay] 						 %Cuadrado con el numero
				\node[at={($(\NumCapX,\NumCapY)$)},anchor=center,font=\rmfamily\fontsize{300pt}{0pt}\bf\selectfont\color{grisoscuro},opacity=0.2] (texto) {\textit{A}};
				\node at (-12.16,-2) {\rule{25cm}{0.5mm}};
				\end{tikzpicture}
			}		
	 		{20pt}  	    				 %Separación Etiqueta/Texto                
	 		{\vspace*{-1.05cm}}              %Codigo anterior
	 		[]   				 %Codigo Posteior
	 }
	 \newcommand{\FormatoAnexoB}{
	 		\titleformat{name=\chapter,numberless}[display]						 %Sin Numerar
			{\Huge\filleft\scshape\color{grisoscuro}}      	 %Formato para etiqueta y texto
	 		{\begin{tikzpicture}[remember picture,overlay] 						 %Cuadrado con el numero
				\node[at={($(\NumCapX,\NumCapY)$)},anchor=center,font=\rmfamily\fontsize{300pt}{0pt}\bf\selectfont\color{grisoscuro},opacity=0.2] (texto) {\textit{B}};
				\node at (-12.16,-2) {\rule{25cm}{0.5mm}};
				\end{tikzpicture}
			}		
	 		{20pt}  	    				 %Separación Etiqueta/Texto                
	 		{\vspace*{-0.15cm}}              %Codigo anterior
	 		[]   				 %Codigo Posteior
	 }

	  \newcommand{\FormatoAnexoC}{
	 		\titleformat{name=\chapter,numberless}[display]						 %Sin Numerar
			{\Huge\filleft\scshape\color{grisoscuro}}      	 %Formato para etiqueta y texto
	 		{\begin{tikzpicture}[remember picture,overlay] 						 %Cuadrado con el numero
				\node[at={($(\NumCapX,\NumCapY)$)},anchor=center,font=\rmfamily\fontsize{300pt}{0pt}\bf\selectfont\color{grisoscuro},opacity=0.2] (texto) {\textit{C}};
				\node at (-12.16,-2) {\rule{25cm}{0.5mm}};
				\end{tikzpicture}
			}		
	 		{20pt}  	    				 %Separación Etiqueta/Texto                
	 		{\vspace*{-0.15cm}}              %Codigo anterior
	 		[]   				 %Codigo Posteior
	 }

	%Formato para el titulo del Capitulo Referencias
	  \newcommand{\FormatoReferencias}{
	  		\titleformat{name=\chapter,numberless}[display]  %Sin Numerar
		 	{\Huge\filleft\scshape\color{grisoscuro}}      	 %Formato para etiqueta y texto
	  	 	{\begin{tikzpicture}[remember picture,overlay]   %Cuadrado con el numero
		 		\node[at={($(\NumCapX,\NumCapY)$)},anchor=center,font=\rmfamily\fontsize{300pt}{0pt}\bf\selectfont\color{grisoscuro},opacity=0.2] (texto) {\textit{R}};
		 		\node at (-12.16,-2) {\rule{25cm}{0.5mm}};
		 		\end{tikzpicture}
		 	}
	  		{20pt}  	     %Separación Etiqueta/Texto                
	  		{}               %Codigo anterior
	  		[]   %Codigo Posteior  
	  }

	%-------------------------------------------------------------

%Indicador lateral de capitulos
%------------------------------------------------------------
	
	%Definiciones Generales de variables para los loops if
	 \newif\ifMaterialInd
	 \newif\ifMaterialCap
	 \newif\ifMaterialAxUno
	 \newif\ifMaterialAxDos
	 \newif\ifMaterialAxTres
	 \newif\ifMaterialBib
	 \newif\ifMaterialAlf

	%Posiciones Iniciales de los indicadores laterales
	 \newlength\PosVertical
	 \setlength\PosVertical{6.56cm}       %Aumentar es mas para arriba
	 \newlength\PosHorizontal	
	 \setlength\PosHorizontal{10.2cm}  %Aumentar es mas para el exterior

	%Con \AddLabelsInd se agregan los indicadores laterales de los indices
	 \newcommand\AddLabelsInd{
	 \MaterialIndtrue
	 \AddEverypageHook{
	 	\ifMaterialInd
	 		\ifodd\value{page} 
  				\backgroundsetup{
	 				  angle=90, %Ojo!! Invierte las coordenadas!!
	 				  contents={
	 				  \begin{tikzpicture}[remember picture,overlay] 
	 				    \node[at={($(\PosVertical,-\PosHorizontal)$)}] % Las coordenadas son (y,-x)
	 					(cuadrado) {\tikz {\draw[draw=none,fill=grisoscuro] (0,0) rectangle ++(3,1);} };
	 					\node[at={($(cuadrado.center)+(0,0.2)$)},anchor=center,font=\large\filleft\scshape,text=white](texto) {Índice};
	 					\end{tikzpicture}
	 				  }
	 				 }
  			\else
	 				 \backgroundsetup{
	 				  angle=90, %Ojo!! Invierte las coordenadas!!
	 				  contents={
	 				    \begin{tikzpicture}[remember picture,overlay] 
	 				    \node[at={($(\PosVertical,\PosHorizontal)$)}] % Las coordenadas son (y,-x)
	 					(cuadrado) {\tikz {\draw[draw=none,fill=grisoscuro] (0,0) rectangle ++(3,1);} };
	 					\node[at={($(cuadrado.center)+(0,-0.1)$)},anchor=center,font=\large\filleft\scshape,text=white](texto) {Índice};
	 					\end{tikzpicture}
	 				  }
	 				 }
  			\fi
	 	\BgMaterial
	 	\else\relax\fi}
	 	}	 

	%Con \AddLabelsCap se agregan los indicadores laterales de los capitulos
	 \newcommand\AddLabelsCap{
	 \MaterialCaptrue
	 \AddEverypageHook{
	 	\ifMaterialCap
	 		\ifodd\value{page} 
  				\backgroundsetup{
	 				  angle=90, %Ojo!! Invierte las coordenadas!!
	 				  contents={
	 				  \begin{tikzpicture}[remember picture,overlay] 
	 				    \node[at={($(\PosVertical,-\PosHorizontal)-(\thechapter,0)$)}] % Las coordenadas son (y,-x)
	 					(cuadrado) {\tikz {\draw[draw=none,fill=grisoscuro] (0,0) rectangle ++(3,1);} };
	 					\node[at={($(cuadrado.center)+(0,0.2)$)},anchor=center,font=\large\filleft\scshape,text=white](texto) {Capítulo \thechapter};
	 					\end{tikzpicture}
	 				  }
	 				 }
  			\else
	 				 \backgroundsetup{
	 				  angle=90, %Ojo!! Invierte las coordenadas!!
	 				  contents={
	 				    \begin{tikzpicture}[remember picture,overlay] 
	 				    \node[at={{($(\PosVertical,\PosHorizontal)-(\thechapter,0)$)}}] % Las coordenadas son (y,-x)
	 					(cuadrado) {\tikz {\draw[draw=none,fill=grisoscuro] (0,0) rectangle ++(3,1);} };
	 					\node[at={($(cuadrado.center)+(0,-0.1)$)},anchor=center,font=\large\filleft\scshape,text=white](texto) {Capítulo \thechapter};
	 					\end{tikzpicture}
	 				  }
	 				 }
  			\fi
	 	\BgMaterial
	 	\else\relax\fi}
	 	}

	%Con \AddLabelsAxUno se agregan los indicaroes laterales Anexo A
     \newcommand\AddLabelsAxUno{
	 \MaterialAxUnotrue
	 \AddEverypageHook{
	 	\ifMaterialAxUno
	 		\ifodd\value{page}
  				\backgroundsetup{
	 				  angle=90, %Ojo!! Invierte las coordenadas!!
	 				  contents={
	 				  \begin{tikzpicture}[remember picture,overlay] 
	 					\node[at={($(\PosVertical,-\PosHorizontal)-(\thechapter+1,0)$)}] % Las coordenadas son (y,-x)
	 					(cuadrado) {\tikz {\draw[draw=none,fill=grisoscuro] (0,0) rectangle ++(3,1);} };
	 					\node[at={($(cuadrado.center)+(0,0.2)$)},anchor=center,font=\large\filleft\scshape,text=white](texto) {Anexo A};
	 					\end{tikzpicture}
	 				  }
	 				 }
  			\else
	 				 \backgroundsetup{
	 				  angle=90, %Ojo!! Invierte las coordenadas!!
	 				  contents={
	 				    \begin{tikzpicture}[remember picture,overlay] 
	 				    \node[at={($(\PosVertical,\PosHorizontal)-(\thechapter+1,0)$)}]  % Las coordenadas son (y,-x)
	 				    (cuadrado) {\tikz {\draw[draw=none,fill=grisoscuro] (0,0) rectangle ++(3,1);} };
	 				    \node[at={($(cuadrado.center)+(0,-0.1)$)},anchor=center,font=\large\filleft\scshape,text=white] (texto) {Anexo A};
	 					\end{tikzpicture}
	 				  }
	 				 }
  			\fi
	 	\BgMaterial
	 	\else\relax\fi}
	 	}

	%Con \AddLabelsAxDos se agregan los indicaroes laterales Anexo B
	 \newcommand\AddLabelsAxDos{
	 \MaterialAxDostrue
	 \AddEverypageHook{
	 	\ifMaterialAxDos
	 		\ifodd\value{page}
  				\backgroundsetup{
	 				  angle=90, %Ojo!! Invierte las coordenadas!!
	 				  contents={
	 				  \begin{tikzpicture}[remember picture,overlay] 
	 					\node[at={($(\PosVertical,-\PosHorizontal)-(\thechapter+2,0)$)}] % Las coordenadas son (y,-x)
	 					(cuadrado) {\tikz {\draw[draw=none,fill=grisoscuro] (0,0) rectangle ++(3,1);} };
	 					\node[at={($(cuadrado.center)+(0,0.2)$)},anchor=center,font=\large\filleft\scshape,text=white](texto) {Anexo B};
	 					\end{tikzpicture}
	 				  }
	 				 }
  			\else
	 				 \backgroundsetup{
	 				  angle=90, %Ojo!! Invierte las coordenadas!!
	 				  contents={
	 				    \begin{tikzpicture}[remember picture,overlay] 
	 				    \node[at={($(\PosVertical,\PosHorizontal)-(\thechapter+2,0)$)}]  % Las coordenadas son (y,-x)
	 				    (cuadrado) {\tikz {\draw[draw=none,fill=grisoscuro] (0,0) rectangle ++(3,1);} };
	 				    \node[at={($(cuadrado.center)+(0,-0.1)$)},anchor=center,font=\large\filleft\scshape,text=white] (texto) {Anexo B};
	 					\end{tikzpicture}
	 				  }
	 				 }
  			\fi
	 	\BgMaterial
	 	\else\relax\fi}
	 	}
	
	%Con \AddLabelsAxDos se agregan los indicaroes laterales Anexo C
	 \newcommand\AddLabelsAxTres{
	 \MaterialAxTrestrue
	 \AddEverypageHook{
	 	\ifMaterialAxTres
	 		\ifodd\value{page}
  				\backgroundsetup{
	 				  angle=90, %Ojo!! Invierte las coordenadas!!
	 				  contents={
	 				  \begin{tikzpicture}[remember picture,overlay] 
	 					\node[at={($(\PosVertical,-\PosHorizontal)-(\thechapter+3,0)$)}] % Las coordenadas son (y,-x)
	 					(cuadrado) {\tikz {\draw[draw=none,fill=grisoscuro] (0,0) rectangle ++(3,1);} };
	 					\node[at={($(cuadrado.center)+(0,0.2)$)},anchor=center,font=\large\filleft\scshape,text=white](texto) {Anexo C};
	 					\end{tikzpicture}
	 				  }
	 				 }
  			\else
	 				 \backgroundsetup{
	 				  angle=90, %Ojo!! Invierte las coordenadas!!
	 				  contents={
	 				    \begin{tikzpicture}[remember picture,overlay] 
	 				    \node[at={($(\PosVertical,\PosHorizontal)-(\thechapter+3,0)$)}]  % Las coordenadas son (y,-x)
	 				    (cuadrado) {\tikz {\draw[draw=none,fill=grisoscuro] (0,0) rectangle ++(3,1);} };
	 				    \node[at={($(cuadrado.center)+(0,-0.1)$)},anchor=center,font=\large\filleft\scshape,text=white] (texto) {Anexo C};
	 					\end{tikzpicture}
	 				  }
	 				 }
  			\fi
	 	\BgMaterial
	 	\else\relax\fi}
	 	}
	 	
	%Con \AddLabelsBib se agregan los indicaroes laterales para la bibliografia
	 \newcommand\AddLabelsBib{
	 \MaterialBibtrue
	 \AddEverypageHook{
	 	\ifMaterialBib
	 		\ifodd\value{page}
  				\backgroundsetup{
	 				  angle=90, %Ojo!! Invierte las coordenadas!!
	 				  contents={
	 				  \begin{tikzpicture}[remember picture,overlay] 
	 					\node[at={($(\PosVertical,-\PosHorizontal)-(\thechapter+4,0)$)}] % Las coordenadas son (y,-x)
	 					(cuadrado) {\tikz {\draw[draw=none,fill=grisoscuro] (0,0) rectangle ++(3,1);} };
	 					\node[at={($(cuadrado.center)+(0,0.2)$)},anchor=center,font=\large\filleft\scshape,text=white](texto) {Referencias};
	 					\end{tikzpicture}
	 				  }
	 				 }
  			\else
	 				 \backgroundsetup{
	 				  angle=90, %Ojo!! Invierte las coordenadas!!
	 				  contents={
	 				    \begin{tikzpicture}[remember picture,overlay] 
	 				    \node[at={($(\PosVertical,\PosHorizontal)-(\thechapter+4,0)$)}]  % Las coordenadas son (y,-x)
	 				    (cuadrado) {\tikz {\draw[draw=none,fill=grisoscuro] (0,0) rectangle ++(3,1);} };
	 				    \node[at={($(cuadrado.center)+(0,-0.1)$)},anchor=center,font=\large\filleft\scshape,text=white] (texto) {Referencias};
	 					\end{tikzpicture}
	 				  }
	 				 }
  			\fi
	 	\BgMaterial
	 	\else\relax\fi}
	 	}

	%Con \AddLabelsAlf se agregan los indicadores laterales del indice alfabetico
	 \newcommand\AddLabelsAlf{
	 \MaterialAlftrue
	 \AddEverypageHook{
	 	\ifMaterialAlf
	 		\ifodd\value{page} 
  				\backgroundsetup{
	 				  angle=90, %Ojo!! Invierte las coordenadas!!
	 				  contents={
	 				  \begin{tikzpicture}[remember picture,overlay] 
	 				    \node[at={($(\PosVertical,-\PosHorizontal)-(\thechapter+6,0)$)}] % Las coordenadas son (y,-x)
	 					(cuadrado) {\tikz {\draw[draw=none,fill=grisoscuro] (0,0) rectangle ++(4,1);} };
	 					\node[at={($(cuadrado.center)+(0,0.2)$)},anchor=center,font=\large\filleft\scshape,text=white](texto) {Índice Alfabético};
	 					\end{tikzpicture}
	 				  }
	 				 }
  			\else
	 				 \backgroundsetup{
	 				  angle=90, %Ojo!! Invierte las coordenadas!!
	 				  contents={
	 				    \begin{tikzpicture}[remember picture,overlay] 
	 				    \node[at={($(\PosVertical,\PosHorizontal)-(\thechapter+6,0)$)}] % Las coordenadas son (y,-x)
	 					(cuadrado) {\tikz {\draw[draw=none,fill=grisoscuro] (0,0) rectangle ++(4,1);} };
	 					\node[at={($(cuadrado.center)+(0,-0.1)$)},anchor=center,font=\large\filleft\scshape,text=white](texto) {Índice Alfabético};
	 					\end{tikzpicture}
	 				  }
	 				 }
  			\fi
	 	\BgMaterial
	 	\else\relax\fi}
	 	}	 	
	
	%Con estos comandos se borran los indicadores laterales	
	 \newcommand\RemoveLabelsInd{\MaterialIndfalse}
	 \newcommand\RemoveLabelsCap{\MaterialCapfalse}
	 \newcommand\RemoveLabelsAxUno{\MaterialAxUnofalse}
	 \newcommand\RemoveLabelsAxDos{\MaterialAxDosfalse}
	 \newcommand\RemoveLabelsAxTres{\MaterialAxTresfalse}
	 \newcommand\RemoveLabelsBib{\MaterialBibfalse}
	 \newcommand\RemoveLabelsAlf{\MaterialAlffalse}

%Formato de los mini-indices de los capitulos 
%------------------------------------------------------------
  \mtcindent=0pt  %indentacion de los Indices de los capitulos
  \def\mtctitle{Contenido}  %Cambio el Nombre del indice de los capitulos

  \setcounter{tocdepth}{3}
  \setcounter{minitocdepth}{3} 
  \setcounter{secnumdepth}{3}
  \renewcommand{\thesubsubsection}{\thesubsection.\roman{subsubsection}}
 %-------------------------------------------------------------

%Formato de la Bibliografia
%------------------------------------------------------------
 %links externos con misma fuente que el texto
 \urlstyle{same}

%Formato de la lista de siglas, abreviaturas y nomenclaturas
%------------------------------------------------------------
 	\newlength\myitemwidth

 	%Declara el formato de la lista de items
	\setlength\myitemwidth{3cm}
	\newlist{listasiglas}{description}{1}
	\setlist[listasiglas]{
  		labelindent = 0pt ,
  		labelsep    = 0pt ,
 		leftmargin  = \myitemwidth ,
 		labelwidth  = \myitemwidth ,
 		itemindent  = 0pt ,
 		format      = \bfseries ,
 		parsep      = -2pt
		}

	%Ledigo al paquete de acronimos que use la lista formateada	
	\DeclareAcroListStyle{estilosiglas}{list}{list = listasiglas}
  	\acsetup{ list-style = estilosiglas,only-used=false }	

%Formato de del indice alfabetico
%------------------------------------------------------------
	
	%Dos Columnas y tamaño mas pequeño
	\makeindex[columns=2, options= -s Tesis.ist]
	\indexsetup{othercode=\footnotesize}

%Formato de las notas al margen
%------------------------------------------------------------
 \let\oldmarginpar\marginpar
 \renewcommand{\marginpar}[1]{\oldmarginpar{\itshape{\textcolor{red}{#1}}}}
 %-----------------------------------------------------------
 
%Colores Personalizados
%------------------------------------------------------------
  \definecolor{grisoscuro}{gray}{0.3}          %El gris oscuro de los Capitulos
  \definecolor{marron}{rgb}{0.647,0.165,0.165} %Los use en los graficos de VC del Au INTI contra Au CNEA
  \definecolor{rojo}{rgb}{1,0,0} 			   %Los use en los graficos de VC del Au INTI contra Au CNEA	
  \definecolor{oliva}{rgb}{0.627,0.502,0.125}  %Los use en los graficos de VC del Au INTI contra Au CNEA
 %------------------------------------------------------------

%Formato de los caption para las tablas y figuras y ecuaciones
%------------------------------------------------------------
 \captionsetup[table]{singlelinecheck=false,justification=justified ,labelfont=bf,font=small,position=top,skip=0pt}
 \captionsetup[subtable]{labelfont=bf,font=small,position=top,aboveskip=10pt,belowskip=10pt,labelsep=period,labelformat=empty}
 \captionsetup[figure]{singlelinecheck=false,justification=justified,labelfont=bf,font=footnotesize,position=bottom,aboveskip=10pt,belowskip=-10pt}
 \captionsetup[subfigure]{singlelinecheck=false,justification=justified,labelfont=bf,font=scriptsize,position=top,aboveskip=0pt,belowskip=10pt,labelsep=period,labelformat = simple,width=.95\textwidth}

%Formato de las viñetas
%------------------------------------------------------------
	\renewcommand{\labelitemi}{$\bullet$}
 %-----------------------------------------------------------

%Abreviaturas para cosas que voy a usar mucho
%------------------------------------------------------------
   
   %Sondas electroquimcias
   \newcommand{\hq}{HQ\index{hidroquinona}}
   \newcommand{\fe}{FeCN\index{ferrocianuro de potasio}\index{ferricianuro de potasio}}
   \newcommand{\ru}{ARu\index{aminorutenio}}
   \newcommand{\fc}{FcOH\index{ferroceno metanol}}  

   %Formula Aminorutenio desarrollada
   \newcommand{\aminorutenio}{[Ru(NH$_3$)$_6]^{3+}$\index{aminorutenio}}
   \newcommand{\aminorutenioCompleto}{Ru(NH$_3$)$_6$Cl{$_3$}\index{aminorutenio}}

   %Formula desarrollada para la cupla Ferro/Ferri
   \newcommand{\ferroferri}{[Fe(CN)$_6]^{4-}$\textsuperscript{/}$^{3-}$\index{ferrocianuro de potasio}\index{ferricianuro de potasio}}
   \newcommand{\ferro}{[Fe(CN)$_6]^{4-}$\index{ferrocianuro de potasio}}
   \newcommand{\ferri}{[Fe(CN)$_6]^{3-}$\index{ferricianuro de potasio}}
   \newcommand{\Ferro}{K$_4$Fe(CN)$_6$\index{ferrocianuro de potasio}}
   \newcommand{\Ferri}{K$_3$Fe(CN)$_6$\index{ferricianuro de potasio}}
   \newcommand{\ferroCompleto}{K$_4$Fe(CN)$_6$.3H$_2$O\index{ferrocianuro de potasio}}
   \newcommand{\ferriCompleto}{K$_3$Fe(CN)$_6$.3H$_2$O\index{ferricianuro de potasio}}
  
   %Formula desarrollada para la Ferroceno
   \newcommand{\ferroceno}{C$_{11}$H$_{12}$OFe\index{ferroceno metanol}}

   %Formula desarrollada para el Hidroquinona
   \newcommand{\hidroquinona}{C$_6$H$_4$O$_2$\index{hidroquinona}}

   %Peliculas Delgadas Mesoporosas de oxido de SIlicio
   \newcommand{\pdm}{PDM\index{película!mesoporosa}}
   \newcommand{\pdmF}{SF\index{película!mesoporosa Si(F127)}}
   \newcommand{\pdmC}{SC\index{película!mesoporosa Si(CTAB)}}
   \newcommand{\pdmZ}{SZF\index{película!mesoporosa Zr/Si(F127)}}
   \newcommand{\pdmZB}{SZB\index{película!mesoporosa Zr/Si(Brij58)}}

%Leyendas para los graficos
%------------------------------------------------------------
	%Las rayas con los colores mas comunes
   	\newbox{\negro}
  		\savebox{\negro}{\raisebox{2pt}{\tikz{\draw[black,solid,line width=0.8pt](0,0) -- (4mm,0);}}}
	\newbox{\marron} 
	  	\savebox{\marron}{\raisebox{2pt}{\tikz{\draw[marron,solid,line width=0.8pt](0,0) -- (4mm,0);}}}
	\newbox{\rojo} 
	  	\savebox{\rojo}{\raisebox{2pt}{\tikz{\draw[rojo,solid,line width=0.8pt](0,0) -- (4mm,0);}}}
	\newbox{\oliva} 
	  	\savebox{\oliva}{\raisebox{2pt}{\tikz{\draw[oliva,solid,line width=0.8pt](0,0) -- (4mm,0);}}}	
	\newbox{\azul} 
	  	\savebox{\azul}{\raisebox{2pt}{\tikz{\draw[blue,solid,line width=0.8pt](0,0) -- (4mm,0);}}}
	\newbox{\gris} 
	  	\savebox{\gris}{\raisebox{2pt}{\tikz{\draw[gray,solid,line width=0.8pt](0,0) -- (4mm,0);}}}
	\newbox{\verde} 
	  	\savebox{\verde}{\raisebox{2pt}{\tikz{\draw[OliveGreen,solid,line width=0.8pt](0,0) -- (4mm,0);}}}
	\newbox{\punteado}
		\savebox{\punteado}{\raisebox{2pt}{\tikz{\draw[black,dashed,line width=0.8pt](0,0) -- (3.2mm,0);}}}
	  	  
%Separación de las sílabas
%------------------------------------------------------------
	\hyphenation{Pasteur Kelvin PMMA Biopack sonda-electrodo hexadeciltrimetilamonio INTI telluride adherencia Fourier}

 %-----------------------------------------------------------

%Seccion de parrafos sin sangria 
%------------------------------------------------------------
	
  \newenvironment{sangria_pers}[1]%
  		{\begin{list}{}%
          {\setlength{\leftmargin}{#1}}%
          \item[]%
  			}
  		{\end{list}}