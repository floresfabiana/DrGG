%Linea Para poder completar automaticamente las citas con el Sublime
%No hace el documento, se puede borrar esta linea si no se usa el Sublime
%------------------------------------------------------------------------------
 \newcommand{\NoBiblioInt}[1]{
 \ifthenelse{\equal{#1}{verdadero}}{}{\bibliography{Referencias/base_bibliografica}}
 \NoBiblioInt{verdadero}}
 %-----------------------------------------------------------------------------

%Formato (Nombre de capitulo largo o corto), nombre del capitulo y estilo de la
%Portada del Capitulo
%------------------------------------------------------------------------------

 %Formato en si, titulo en un solo renglon
 \FormatoCapituloUnaLinea
 
 %Nombre y etiquete para referir
 \chapter{Integración}
 \label{chap:Integracion}

 %Para que no salga el numero de pagina en la portada del capitulo
 \thispagestyle{empty}
	
 %Resumen del Capitulo en Italica
 \noindent\textit{Este capitulo cuaenta coomo se puede intergrar bottom-up, top-down y hacer un dispositivo intergrado, miniaturizado, escalado, industrializable, tecnologicamente compatible, IC, logica, sensores MEMS}

 %Indice de capitulo alineada al borde inferior de la pagina, nueva pagina
 \vfill
 \minitoc
 \newpage
 %-------------------------------------------------------------------------------

\section{Compatibilidad}

Controlar los espesores de cada capa tienen una importancia vital a la hora de hacer la implementación tecnologia, tiempos costos, ya que capas mas delgadas, nos permitirán reducir las dimensiones del diseño, depositar mejor las peliculas mesopororosas. Elevaciones para que no haya escalanos a la hora de depositar, etc.
Aca va todo el problema del Au, curva de calibracion del Sputt, XPS, , etc.
Aca va todo lo de litografia, lift-off, resultados de la fabricación, poblemas de compatibilidad con el Au respecto del Tratamiento Termirco.

\section{integracion con IC}

\section{Diseño de los electrodos}

\section{Posibilidades futuras}